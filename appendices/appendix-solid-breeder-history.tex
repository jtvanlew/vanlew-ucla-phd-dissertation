\chapter{History of solid breeder design}\label{sec:solid-breeder-history}
A design was proposed by Abdou\etal\cite{Abdou1975} in 1975 wherein the plasma would be surrounded by a blanket of nonmobile, solid lithium. To exist in solid form at high temperatures, pure lithium, which has a melting temperature of only about 180~\celsius, must be combined with refractory materials (melting temperatures >1000~\celsius). To date, most parties researching solid breeder blankets have settled on lithium orthosilicate (\lis) or lithium metatitanite (\lit) as candidate ceramics.

As nuclear energy is deposited into the solid breeder, large thermal gradients in the solid lithiated ceramics will induce thermal stresses across large characteristic lengths. Avoiding thermal stress has led to most solid breeder designs implementing packed beds of small, spherical (or near-spherical) pebbles.\cite{Lulewicz2002, Mandal2012a, Tsuchiya1998, Cho2012} Moreover, tritium diffusion and release considerations for solid lithium ceramic support the choice of short characteristic lengths of individual pebbles. From an engineering design standpoint, the choice of packed bed has other desirable characteristics. For instance, the ensemble of small spherical pebbles can be filled into many complex shapes with relatively uniform porosity. The uniform packing of spheres permits a well-distributed flow of purge gas for tritium extraction. 

The advantages of the pebble bed design include ease of uniformly assembling the solid into complex geometries, ease of tritium extraction from the porous bed via an interstitial purge of helium, and with the small size of pebbles being more resilient to thermal stresses than a solid brick of lithiated ceramic.\cite{Casadio2004} 


Lithium oxide had been considered because of its favorable lithium density, among other attractive features, though the reaction of lithium with elemental oxygen is a concern. Pure lithium reacts with oxygen exothermically in reactions such as

\begin{subequations}
\begin{align}
	2\mathrm{Li} + \frac{1}{2}\mathrm{O} &\rightarrow \mathrm{Li}_2\mathrm{O} - 142.75\ \text{kCal/mol}\\
	2\mathrm{Li} + \mathrm{O} &\rightarrow \mathrm{Li}_2\mathrm{O}_2 - 151.9\ \text{kCal/mol}
\end{align}
\end{subequations}

Of primary concern in lithium fires is the peak flame temperature. This will determine, to a large extent, whether many radioactive species become air-borne by vaporization. The flame temperature depends on many variables. Some investigations found it to be about 2500~K which would cause some materials to melt but not vaporize. [cite Abdou's class notes?]