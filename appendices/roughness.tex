\chapter{Surface Roughness}\label{sec:granular-contact-roughness}
Bahrami et al use a joint thermal resistance of the superposition of macroscopic and microscopic influences.\cite{Bahrami2004}

The thermal joint resistance is
\begin{equation}
	R_j = R_s + R_L
\end{equation}
where the subscript $L$ refers to macroscopic variables and $s$ refers to microscopic ones. Bahrami et al used the constriction formulation of Yovanovich et al to express the macroscpic resistance as\cite{Yovanovich1976}
\begin{equation}
	R_L = \frac{(1-a_L/b_L)^{3/2}}{2k_sa_L}
\end{equation}
where $a_L$ is the radius of contact, $b_L$ is the relative radius of curvature (which, in the notation of this thesis is $R^*$). If the contact of the two materials obeys Hertz contact law, then $a_L/b_L \ll 1$ and the above becomes
\begin{equation}\label{eq:macro-thermal-resistance}
	R_L = \frac{1}{2k_sa_L}
\end{equation}
which matches the heat conduction form of Batchelor and O'Brien\cite{Batchelor1977} as used in \cref{eq:batchelor-pebble-conductance}.

To determine the thermal resistance of the asperities in contact, Bahrami et al used a superposition of many cylindrical constrictions inside of the contact area. The result is given in \cite{Bahrami2004} as
\begin{equation}
	R_s^* = \begin{cases}
	\left(\frac{\pi H'b_L^2}{F} \right)^s 										& F_c = 0\\
	(b_L/a_L)^2(H'/P_0)^s(1+s\gamma) 										& F \le F_c\\
	(H'/P_{0,c})^s(1+s\gamma_c)+\left[\frac{\pi H'b_L^2}{(F-F_c)}\right]^s				& F\ge F_c
	\end{cases}
\end{equation}
where $R_s^*$ is a non-dimensional form of the surface roughness thermal resistance, defined as $R_s^* = 1.25\pi b_L^2k_s(m/\sigma)R_s$, $k_s$ is the harmonic mean of contact grains thermal conductivity, $H'$ is the Vicker's hardness value, $F$ is the contact force, $P_0$ is the maximum pressure of contact, $s$ is a parameter based on the hardness constants, $\gamma = 1.5(P_0/P_{0,H})(a_L/a_H)^2-1$, $F_c$ is the critical force where $a_L = b_L$, $\gamma_c$ is the value of $\gamma$ at the critical force, $m$ is the mean absolute surface slope, and $\sigma$ is the rms surface roughness. For Hertzian contact, $\gamma = 0.5$.

Antonetti et al proposed a correlation for mean absolute surface slope as\cite{Antonetti1984}
\begin{equation}\label{eq:m-of-sigma}
	m = 0.125(\sigma\times10^6)^{0.402}
\end{equation}
where the range of applicability of surface roughness is \SI{0.216E-6}{\meter}$\le \sigma <$ \SI{9.6E-6}{\meter}. Thus the term $\sigma/m = 0.031\sigma^{0.598}$

For Hertzian contacts of the non-conforming ceramic materials, $F \ll F_c$, thus we can consider only that case to write
\begin{equation}
	R_s = \frac{(b_L/a_L)^2(H'/P_0)^s(1+s/2)}{1.25\pi b_L^2k_s(0.031\sigma^{0.598})}
\end{equation}
or
\begin{equation}\label{eq:thermal-resistance-pressure}
	R_s = \frac{(H'/P_0)^s(1+s/2)}{1.25\pi a_L^2k_s}\left(0.031\sigma^{0.598}\right)
\end{equation}
For Hertzian contact, the maximum pressure is given by \cref{eq:hertzian-pressure}. It is
\begin{equation*}
	P_0 = \frac{2E^*\delta_n}{\pi a_L}
\end{equation*}
Furthermore, as noted by Bahrami et al, the parameter $s$ is in the range of $0.95\le s\le 0.97$. Therefore it can be approximated as $s=0.96$ here. Thus the thermal resistance of \cref{eq:thermal-resistance-pressure} is rewritten in a simplified form,
% \begin{equation}
% 	R_s = \frac{\pi^{s-1}}{1.25(2^s)}\left(\frac{H'}{E^*}\right)^s\frac{1+s/2}{k_s\delta_n^s}\left(\frac{\sigma}{m}\right)a_L^{s-2}
% \end{equation}
\begin{equation}\label{eq:micro-thermal-resistance}
	R_s = \left(\frac{H'}{E^*\delta_n}\right)^{0.96}\left(\frac{\sigma}{m}\right)\frac{1}{1.720k_sa_L^{1.04}}
\end{equation}

The macroscopic and microscopic thermal resistances given in \cref{eq:macro-thermal-resistance} and \cref{eq:micro-thermal-resistance}, respectively, are combined to give the total joint thermal resistance of
\begin{equation}
	R_j = \left(\frac{H'}{E^*\delta_n}\right)^{0.96}\frac{0.031\sigma^{0.598}}{1.720k_sa_L^{1.04}} + \frac{1}{2k_sa_L}
\end{equation}
and the total thermal conductance between the two grains, $H_j = 1/R_J$, is
\begin{equation}\label{eq:micro-macro-conductance}
	H_j = \left[\left(\frac{H'}{E^*\delta_n}\right)^{0.96}\frac{0.031\sigma^{0.598}}{1.720k_sa_L^{1.04}} + \frac{1}{2k_sa_L}\right]^{-1}
\end{equation}

In the limit of zero roughness, the first term inside the bracket tends to 0 and the conductance is simply the Hertzian solution of a pair of perfectly smooth elastic spheres.

In the LIGGGHTS source code file `fix\_heat\_gran\_conduction.cpp', the macroscopic thermal resistance is incorporated into the heat conductance term. To include the microscopic term, we simply need information on the $H'$, $\sigma$, and $m$. All other terms of \cref{eq:micro-macro-conductance} are available to the code.