\abstract{Ceramic pebble beds as tritium breeders in fusion reactors must endure enormous nuclear energy depositions while satisfying energy and tritium fuel production requirements. To accomplish these feats, pebble beds must maintain ceramic temperatures within relatively-narrow prescribed operating windows while facilitating transport of high quality heat into coolants flowing through structural containers. The ceramic pebble beds exist with meta-stable packing structures that will evolve from thermal transients which generate external and internal forces during long-term operation as tritium breeders. As a consequence, predictive models of solid breeder heat transfer characteristics must contend with transitory packing structures and the changing modes of heat transfer they present. To provide such predictive modeling, microscale numerical models were developed allowing investigation of thermal transport in pebble beds operating in environmental conditions relevant to planned fusion reactors. Furthermore, in this work, specific effort was made to apply the predictive models toward simulating pebble bed thermomechanical responses to the fault condition of crushed individual pebbles.

In this work, the thermal discrete element method (DEM) models motions and heat transfer between individual pebbles in packed bed assemblies. Pebble interaction with slow-moving, interstitial helium purge gas is accomplished by means of two coupling approaches. First, the fluid is considered with a volume-averaged computational fluid dynamic (CFD) method. Volume-averaged models of helium are computationally efficient and provide an overall view of helium influence on heat transfer in solid breeder pebble beds. Second, the lattice-Boltzmann method (LBM) is employed to gain insight into complete fluid flow patterns and conjugate heat transfer. The lattice-Boltzmann method is well-suited to modeling complex porous structures (such as packed beds) due to its inherent parallelizability and simple application of solid-fluid interface boundary conditions on structured grids. 

Several open-source codes have been used as platforms for launching the numerical experiments. The codes provided basic numerical frameworks for, \textit{e.g.} time integration, particle tracking, mesh decomposition, and streaming/colliding operators. However, stock codes are not sufficient for modeling the complex interaction of pebble beds in fusion reactors. The following was contributed to code development: stochastic numerical implementation of `apparent' Young moduli distributions; model of surface roughness effects for heat transfer contact conductance; a mass- and energy-conserving pebble fragmentation algorithm; implementation of the Jeffreson correction to the inter-phase exchange coefficient for moderate-to-high Biot number conditions.

Numerical models to measure effective simulation pebble beds with stagnant interstitial helium are used to validate the numerical models. The DEM and CFD-DEM models first agreed well with experiments of lithium ceramic pebble beds in both vacuum and stagnant helium; results in helium have negligible error compared to uncertainty in experimental data. The predictive capability of the models were then demonstrated with validation against a broad range of non-fusion-type packed bed experimental data as well as the commonly-applied SZB correlation. 
%The predictive models developed in this thesis offer the most complete view of the underlying causes of thermomechanical expressions of packed beds. 

%They have been applied in parametric studies of fusion-relevant representative volume pebble beds to reveal new phenomena that have not yet been considered from standard experimental and modeling efforts. Three main application studies were performed to consider: (i) effective thermal conductivity responses to solid conductivity reductions due to irradiation damage; (ii) changes to temperature distributions in ITER-relevant solid breeders as a function of many variables, including crushing, fragmentation, and resettling; and (iii) complete helium flow fields, tortuosity, conjugate heat transfer, and transverse dispersive conductivity changes in packed beds with crushed pebble fragments. The microscale nature of models developed for this thesis made possible the uncovering of several critical and consequential phenomena.

The predictive models developed in this thesis were used to address several of the most pressing thermomechanical issues for solid breeder ceramic pebble beds: fragmentation of \lit~and gap formation between pebble beds and structural materials. In ITER-like representative volumes, mass re-distribution in pebble beds due to fragmentation was shown to induce subtle changes to local packing fractions yet have the ability to result in macroscopically consequential changes to temperature distributions with volumetric heating. Pebble fragmentation had the largest impact when fragments were significantly smaller than the original pebble (fragments 1/125 the volume of the original) and a comparably large number of pebbles were crushed, 5\%. In this case, maximum pebble bed temperatures increased approximately 20\% compared to the well-packed bed with a nuclear heating rate of \SI{8}{\mega\watt\per\cubic\meter}. Yet when pebbles broke into larger fragments and the amount of broken pebbles was less extensive, the thermomechanical response of pebble beds was significantly more tame; less than 5\% increases in maximum bed temperatures were seen in all pebble beds considered when only 1\% of the pebbles fragmented. Moreover, the models predict that horizontal-style configurations of breeder zones, such as in the EU HCPB design for ITER, produced \textit{no gap} between the upper layer of pebbles and coolant surface even up to 5\% of all pebbles fragmenting during operation. In fact, the configuration’s orientation relative to the gravity vector resulted in slight broadening of temperature profiles, and even slightly lower peak-to-average temperatures than vertical-style configurations, as packing structures evolved due to fragmentation. 

However, recent studies of contact forces as a function of external pressure on the bed indicate that pebble fragmentation to the extent considered in the parametric study of this thesis are unlikely to occur. For small quantities of crushing, the effects on thermomechanics are predicted to be relatively minor compared to other phenomena such as contact creep inside the pebble bed. Based on the results of this thesis, \textit{i.e.} no gap formation in horizontal-style breeders, future modeling of pebble bed thermomechanics should focus on modeling high-performance concepts to increase effective conductivity (\textit{e.g.} mixed beryllide/ceramic volumes) rather than the negative effects of pebble crushing.

The LBM models used in this thesis validated the accuracy of the volume-averaged approach of fluid flow and energy interaction between solid and fluid phases, as adopted in the CFD-DEM model. From LBM results, the laminar nature of low-Reynolds flow in packed beds implies conduction is the dominant mode of heat transfer through the packed bed. The maximum difference between temperature profiles predicted with CFD-DEM and LBM-DEM models was only 6\%, while CFD-DEM simulations were run in considerably less time than the full models of LBM-DEM (hours compared to days). Nevertheless, the small differences between LBM and CFD-DEM results arose due to increases in transverse dispersive conductivity of the fluid which itself suggest larger Reynolds or higher packing fraction pebble beds may increase transverse dispersion and thereby the ability to handle higher heat deposition due to increased effective conductivity. The LBM model of fluid and solid interaction should be used for future studies of this topic.
%The contribution of transverse dispersive conductivity, a heat transfer mechanism not accounted for in CFD-DEM, was shown to be not significant for the range of Reynolds/Peclet numbers of interest for purge gas in solid breeders and the failure percentages in the packed bed configurations considered; the dispersive component of conductivity, even in beds with fragments and higher tortuosity, was only 0.5\% of the total effective thermal conductivity.

}