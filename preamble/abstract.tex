\abstract{Lithium ceramic pebble beds are a proposed form for tritium breeding volumes in fusion reactors. In application, the beds will endure high volumetric energy deposition densities but must maintain within relatively-narrow prescribed temperature operating windows for efficient tritium release, while also providing continued transport of high quality heat into coolants for power production. The ceramic pebble beds, as non-cohesive granular material, exist with metastable packing structures defined by the local equilibrium between normal forces and static friction force chains in the assembly. Differential rates of heating and thermal expansion coefficients between ceramic pebble beds and their structural container induce stresses in the bed volume. If sufficiently large, the stresses overcome the local equilibrium of inter-pebble forces and irreversibly compel the pebble bed into a new metastable packing state. Transport of heat from pebble bed to coolant structure is divided between contact conductance between pebbles in the ensemble and convection with a helium purge gas filling the interporous voids. Thus thermal characteristics of pebble beds are intimately linked with its mechanical ones. As a consequence, predictive models of solid breeder heat transfer characteristics must contend with both flowing interporous fluid as well as transitory packing structures and the changing modes of heat transfer they present. To provide such predictive modeling, microscale numerical models were developed allowing investigation of thermal transport in pebble beds operating in environmental conditions relevant to planned fusion reactors. Specific effort was made to apply the predictive models toward simulating pebble bed thermomechanical responses to the fault condition of crushed individual pebbles.

In this work, the thermal discrete element method (DEM) has been used to model forces and heat transfer between individual pebbles in assemblies. Pebble interaction with slow-moving, interstitial helium purge gas is accomplished by means of two coupling approaches. First, the fluid is considered with a volume-averaged computational fluid dynamic (CFD) method. Volume-averaged models of helium are computationally efficient and provide an overall view of helium influence on heat transfer in solid breeder pebble beds. Second, the lattice-Boltzmann method (LBM) is employed to gain insight into complete fluid flow patterns and conjugate heat transfer. The lattice-Boltzmann method is well-suited to modeling complex porous structures (such as packed beds) due to its inherent parallelizability and simple application of solid-fluid interface boundary conditions on structured grids. 

Several open-source codes have been used as platforms for launching the numerical experiments. The codes provided basic numerical frameworks for, \textit{e.g.} time integration, particle tracking, mesh decomposition, and streaming/colliding operators. However to apply the numeric codes on the unique environment of fusion pebble beds, the following contributions to the code were necessarily developed: stochastic numerical implementation of `apparent' elastic moduli distributions; model of surface roughness effects for heat transfer contact conductance; a mass- and energy-conserving pebble fragmentation algorithm; implementation of the Jeffreson correction to the inter-phase exchange coefficient for moderate-to-high Biot number conditions. The models were validated against experiments measuring effective thermal conductivity of pebble beds with a stagnant interstitial gas. The DEM and CFD-DEM models first agreed well with experiments of lithium ceramic pebble beds in both vacuum and stagnant helium; simulation results of beds with helium match very well with experimental data. The predictive capability of the models were then demonstrated with validation against a broad range of non-fusion-type packed bed experimental data as well as the commonly-applied SZB correlation. The predictive models developed in this thesis were used to address several of the most pressing thermomechanical issues for solid breeder ceramic pebble beds: fragmentation of \lit~or \lis~pebbles and gap formation between pebble beds and structural materials. 
%The predictive models developed in this thesis offer the most complete view of the underlying causes of thermomechanical expressions of packed beds. 

%They have been applied in parametric studies of fusion-relevant representative volume pebble beds to reveal new phenomena that have not yet been considered from standard experimental and modeling efforts. Three main application studies were performed to consider: (i) effective thermal conductivity responses to solid conductivity reductions due to irradiation damage; (ii) changes to temperature distributions in ITER-relevant solid breeders as a function of many variables, including crushing, fragmentation, and resettling; and (iii) complete helium flow fields, tortuosity, conjugate heat transfer, and transverse dispersive conductivity changes in packed beds with crushed pebble fragments. The microscale nature of models developed for this thesis made possible the uncovering of several critical and consequential phenomena.



In ITER-like representative volumes, mass re-distribution in pebble beds due to fragmentation was shown to induce subtle changes to local packing fractions yet have the ability to result in macroscopically consequential changes to temperature distributions with volumetric heating. Pebble fragmentation had the largest impact when fragments were significantly smaller than the original pebble (fragments 1/125 the volume of the original) and a comparably large number of pebbles were crushed, 5\%. In this case, maximum pebble bed temperatures increased approximately 20\% compared to the well-packed bed with a nuclear heating rate of \SI{8}{\mega\watt\per\cubic\meter}. Yet when pebbles broke into larger fragments and the amount of broken pebbles was less extensive, the thermomechanical response of pebble beds was significantly more tame; less than 5\% increases in maximum bed temperatures were seen in all pebble beds considered when only 1\% of the pebbles fragmented. %However, recent material development as resulted in ceramic pebbles with stronger average crush load values. Furthermore, recent investigations into the relationship between contact forces and macroscopic stresses indicate that pebble fragmentation to the extent considered in the parametric study of this thesis are unlikely to occur. Therefore, in the case of small extent of pebble fragmentation, the effects on thermomechanics will not be large. Future pebble modeling should shift focus towards other phenomena such as contact creep inside the pebble bed which will dominate in pebbles which can withstand large forces before crushing. 



From the same parametric range of ITER representative pebble bed configurations, the models predict that horizontal-style configurations of breeder zones, such as in the EU HCPB design for ITER, produced \textit{no gap} between the upper layer of pebbles and coolant surface even up to 5\% of all pebbles fragmenting during operation. In fact, the configuration’s orientation relative to gravity resulted in slight broadening of temperature profiles, and even slightly lower peak-to-average temperatures than vertical-style configurations, as packing structures evolved due to fragmentation. Interstitial helium was seen to accommodate much of the loss of contact conductance on overall thermal transport in packed beds, in spite of the possibility of gap formation. The role of flowing helium purge gas has been considered for the first time by the models of fusion pebble beds developed for this thesis, and its impact on thermal transport was given some careful attention in an LBM-based model. 

From LBM results, the laminar nature of low-Reynolds flow in packed beds implies conduction is the dominant mode of heat transfer through the packed bed. The maximum difference between temperature profiles predicted with CFD-DEM and LBM-DEM models was only 6\%, the difference arising entirely from the pure conduction model (volume-averaged CFD-DEM) versus the consideration of fluctuation terms (LBM) adding to a transverse thermal dispersion on effective conductivity. For a design which will have such low-Re flow, CFD-DEM simulations were run in considerably less time than the full models of LBM-DEM (hours compared to days) and the error from neglecting thermal dispersion may be acceptable. Nevertheless, the small differences between LBM and CFD-DEM results arose due to increases in transverse dispersive conductivity of the fluid which itself suggest larger Reynolds or higher packing fraction pebble beds may increase transverse dispersion and thereby the ability to handle higher heat deposition due to increased effective conductivity. The LBM model of fluid and solid interaction should be used for future studies of this topic.
%The contribution of transverse dispersive conductivity, a heat transfer mechanism not accounted for in CFD-DEM, was shown to be not significant for the range of Reynolds/Peclet numbers of interest for purge gas in solid breeders and the failure percentages in the packed bed configurations considered; the dispersive component of conductivity, even in beds with fragments and higher tortuosity, was only 0.5\% of the total effective thermal conductivity.

}