\subsection{Radiative Transfer with Neighboring Particles}

The temperatures expected in the solid breeder are high enough that we can not a priori neglect radiation. The radiation exchange between contacting neighbors in a packed bed becomes extremely complex due to the local and semi-local nature of radiation. A standard approach to treat radiation exchange between surfaces is to consider the view factor between them. In a dense, randomly packed bed of spheres the computation of view factors between pebbles can be done via a method such as that proposed Feng and Han\cite{Feng2012}. Ideally, we could show this mode of heat transport is negligible compared to the others already discussed.

In ceramic breeder designs, the tritium breeding volume is rarely more than \si{2 cm} wide with pebbles that are, generally, \si{1 mm} in diameter. The maximum expected temperature in the breeding zone is about \si{1000 K}, roughly at the centerline of the \si{2 cm} width. The walls of the coolant must be held below the operable steel temperature of roughly \si{700 K}. This works out to a \si{300 K} differences spanning 10 pebble diameters. From this we can make a first-order approximation of \si{30 K} difference between neighboring pebbles. At the elevated temperatures, an estimate for the radiation exchange between two pebbles (allowing them to act as black bodies for this approximation) is

\begin{equation}
	\dot{Q}_\text{radiation} = \sigma A \left(T_\text{max}^4 - (T_\text{max}-30)^4\right) \approx 0.022\si{W}
\end{equation}
 
 which is the highest amount of radiation exchange we might expect between pebbles. Even though we will neglect this mode of heat transfer for now, after reviewing some of the packed bed heat transfer results we may find that this quantity of energy transfer is not negligible and future versions of the model would have to account for it.