\section{Nusselt number for spheres in packed beds}\label{sec:particle-convection}

Historyically, in the treatment of packed beds for heat transfer, engineers developed relationships between overall heat transfer coefficients to the log-mean temperature of the bed. The Nusselt number correlation was applicable to the bed overall rather than any discrete particle inside of the bed. The correlations will be useful for validating our models of helium flow through packed beds of lithium ceramics.


Recently, however, experimental and numerical research has focused on the heat transfer at the scale of a particle as a component of dilute or dense packed beds. These correlations will be useful for applying to single discrete elements in our DEM framework.

\begin{equation}
	Q_\text{convection} = -hA_r(T_r-T_f)
\end{equation}

where $T_r$ is the temperature of the solid with surface area, $A_r$, and $T_f$ is the local bulk temperature of the passing fluid.

%~~~~~~~~~~~~~~~~~~~~~~~~~~~~~~~~~~~~~~~~~~~~~~~~~~~~~~~~~~~~~~~~~~~~~~~~
\subsection{Packed bed correlations: Nellis and Klein}

A heat transfer coefficient for a packed bed of spheres, as given by Nellis and Klein, is determined strictly on geometric details of the packed bed.  Necessary values are the porosity, or void fraction, $\epsilon$, and the cross-sectional surface area of the bed, $A_t$.

The average heat transfer coefficient is correlated in terms of the Colburn $j_h$ factor.

\begin{equation}
	j_h=\frac{\bar{h}}{G C_f} \Pr^{2/3}
\end{equation}

The mass flux, $G$, is evaluated in terms of the specific surface area and mass flow rate:

\begin{equation}
	G=\frac{\dot{m}}{\epsilon A_t}
\end{equation}

where $C_f$ is the specific heat capacity of the fluid and $\Pr$ the Prandtl number of the fluid. For a packed bed of spheres, Nellis \& Klein used the modified Reynolds number suggested by Kays and London (1984), defined as,

\begin{equation}
	\Re_G=\frac{4 G r_{char}}{\mu_f}
\end{equation}

where $\mu_f$ is the viscosity of the fluid.  The characteristic radius $r_{char}$ is given as:

\begin{equation}
	r_{char}=\frac{\epsilon d_p}{4(1-\epsilon)}
\end{equation}

where $d_p$ is the average diameter of the packed bed particle. The relationship between Colburn $j_h$ factor and mass-flux Reynolds number was provided in Nellis and Klein. The interpolation of their data yields a functional relationship of,

\begin{equation}
	jh=0.191 \Re_G^{-.278}
\end{equation}

Therefore, the heat transfer coefficient is found as,

\begin{equation}
	h=(0.191 \Re_G^{-.278}) (\frac{\dot{m}}{\epsilon A_t}) C_f Pr^{-2/3}
\end{equation}

which is applicable for spherical objects in densely packed beds.
%~~~~~~~~~~~~~~~~~~~~~~~~~~~~~~~~~~~~~~~~~~~~~~~~~~~~~~~~~~~~~~~~~~~~~~~~





%~~~~~~~~~~~~~~~~~~~~~~~~~~~~~~~~~~~~~~~~~~~~~~~~~~~~~~~~~~~~~~~~~~~~~~~~
\subsection{Packed beds correlation: Whitaker}

Definition of Reynolds number

\begin{equation}
	\Re = \frac{D_p G}{\mu_b(1-\epsilon)}
\end{equation}

Definition of Nusselt number

\begin{equation}
	\Nu = \frac{hD_p}{k_f}\frac{\epsilon}{1-\epsilon}
\end{equation}

Definition of h

\begin{equation}
	h_{\ln} = \frac{\dot{Q}}{a_v}V\Delta T_{\ln}
\end{equation}

where $a_v = (A_p/V_p)(1-\epsilon)$ is the surface area per unit volume. And $\delta T_{\ln}$ is the log-mean temperature difference.

Reference temp, $T^*$

\begin{equation}
	T^* = \frac{1}{2}(T_{f1} + T_{f2})
\end{equation}

Range of Reynolds number
$22-8 \times 10^3$

Range of Prandtl number
0.7

Range of ($\mu_b/\mu_0$)
1

Correlation,

\begin{equation}
	\Nu = \left( 0.5 \Re^{1/2} + 0.2\Re^{2/3} \right)\Pr^{1/3}
\end{equation}
%~~~~~~~~~~~~~~~~~~~~~~~~~~~~~~~~~~~~~~~~~~~~~~~~~~~~~~~~~~~~~~~~~~~~~~~~




%~~~~~~~~~~~~~~~~~~~~~~~~~~~~~~~~~~~~~~~~~~~~~~~~~~~~~~~~~~~~~~~~~~~~~~~~
\subsection{Single pebble correlations}

For a sphere in an infinite, quiescent fluid where conduction is the only mode of heat transfer, the Nusselt number is identically 2. In all the single particle correlations, that value is the the limit as $\Re \rightarrow 0$. 

Ranz \& Marshall in 1952 

Zhou, et al 2009

\begin{equation}
	\Nu_i = 2.0 + 1.2\Re_i^{1/2}Pr^{1/3}
\end{equation}