\subsection{Granular Heat Transfer}\label{sec:dem-heat-transfer}

In way analogous to that of handling the momentum of every particle in DEM with Newton's laws of motion, the Lagrangian tracking of energy of each particle is obtained via the first law of thermodynamics. We treat each particle is a single distinct object and thus do not consider any internal temperature gradients (a point which we alluded to in \cref{sec:ht-jeffreson-correction}). The temperature of particle $i$ is governed by

\begin{equation}\label{eq:thermoFirstLaw}
	m_iC_i\ddt{T_i} = Q_{s,i} + Q_{i}
\end{equation}
where $m$ and $C$ are the mass and the specific heat of the solid, respectively. Heat generation inside the particle is input with $Q_{s}$ and the total heat transferred to/from particle $i$ via conduction to all, $Z$, neighboring particles, is
\begin{equation}
	Q_i = \sum_{j=1}^Z Q_{ij}
\end{equation}

The conductive heat transfer to neighboring particles comes from inter-particle conductance formulas (derived in \cref{sec:ht-pebble-conduction}). The heat conducted between two particles is,
\begin{equation}
	Q_{ij} = H_c(T_i - T_j)
\end{equation} 
where the conductance is
\begin{equation}\label{eq:dem-conductance}
	H_c= 2k^*\left[\frac{3F_{n,ij}R^*}{4E^*}\right]^{1/3}
\end{equation}

These equations are valid for perfectly smooth spheres that obey Hertzian contact laws. The validity of this heat conductance model will be tested.

\subsubsection{Thermal Expansion}
The stresses predicted to act upon the solid breeder volume during operation of the fusion reactor arise from the differential rate of thermal expansion from the highly heated ceramic volume and the relatively cool structural container. Moreover, thermal settling motion is observed in pebble beds with cyclic heating.\cite{Tanigawa:2010cr, Vargas2007, Chen2009, Divoux2008} Both of those phenomena origin from effects of thermal expansion of individual particles in the ensemble. Therefore, we introduce a thermal expansion formula that updates the diameter of each particle as,

\begin{equation}
	d_{p,i} = d_{p_0,i}\left[1+\beta_i\left(T_i - T_0\right)\right]
\end{equation}
where $\beta_i$ is the thermal expansion coefficient (in units of \si{1/K}), $T_i$ is the temperature of the pebble at the current step, and $d_{0,i}$ is the initial diameter of the pebble at temperature $T_0$. The update of pebble diameter based on thermal expansion could be computed at every time step as it is not computationally expensive. All the same, I have left flexibility in the code to allow the computation at an arbitrary interval of time steps (typically every $\frac{1}{10000\Delta t}$)

