\section{Heat transfer between contacting spheres}

A complete picture of heat transfer of contacting spheres would include internal conduction in the solid material, conduction between contacts, convection to an interstitial gas (as well as partitioning energy carried downstream and energy redeposited locally), and radiation between neighboring solids (on both local and semi-local scales). The first natural step is to focus simply on internal and inter-particle conduction.

[Go back through Batchelor and O'Brien~\cite{Batchelor1977} paper]

\begin{align}
\frac{ k_s }{ k_f } \frac{a}{R^*} = \lambda
\end{align}

Similar to the lumped capacitance assumptions, if $\lambda \gg 1$, the solid is approximately is isothermal. The second group on the left-hand side of this condition we remember from the assumptions of Hertz theory, where we require $\frac{a}{R^*} \ll 1^*$. Therefore to satisfy the condition of $\lambda \gg 1$, we require very large conductivity ratios of solid to fluid, $\frac{k_s}{k_f} \gg 1$. Alternatively this is satisfied by definition if the solids exist in vacuum.

Assuming that we satisfy the condition of isothermal solids, we address the conduction between solids in their small regions of contact.

[more details]

Handling the heat transfer between contacting particles has been investigated extensively by researchers in a number of fields\cite{Zhou2009,Zhang2011,Wu2011,Vargas2001,Li2000,Chaudhuri2006}. The amount of energy per time that can be transported per difference in temperature between pebble $i$ and $j$ as a conductance $h_{ij}$. Defined as
\begin{align}\label{conductance}
\frac{h_{ij}}{k^*}= 2\left[\frac{3F_nR^*}{4E^*}\right]^{1/3}
\end{align}
$k^*= 2k_ik_j/(k_i+k_j)$ is the effective solid conductivity of the two particles, and $F_n$ is the magnitude of the normal force between particles $i$ and $j$ as calculated by Eq.~\ref{eq:hertzForce}. Therefore, if we consider particles at temperatures $T_i$ and $T_j$ in contact, they will transfer heat at a rate of
\begin{align}
Q_{ij} = h_{ij}(T_i - T_j)
\end{align} 
The temperature of particle $i$, for example, is found from the first law of thermodynamics
\begin{align}\label{thermoFirstLaw}
\rho_iV_iC_i\frac{\mathrm{d}T_i}{\mathrm{d}t} = Q_{s,i} + Q_{ij}
\end{align}
where $\rho$, $V$, and $C$ are the density, volume, and the specific heat of the solid, respectively. The energy equation also allows for source heating of the solid with term $Q_{s}$.