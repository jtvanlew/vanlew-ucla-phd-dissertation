\section{Blanket Concept: Solid Breeder}\label{sec:blanket-design}

Material and Form

A design was proposed by Abdou et al\cite{Abdou1974d} in 1975 wherein the plasma would be surrounded by a so-called blanket of nonmobile, solid lithium. Pure lithium has a melting temperature of only about 180~\celsius so must be combined with a refractory material to keep it in solid form at high temperatures (melting temperatures >1000~\celsius). To date, most parties researching solid breeder blankets are focusing on lithium orthosilicate (\lis) or lithium metatitanite (\lit) as candidate ceramics, though other candidate ceramics do still exist.

As nuclear energy is deposited into the solid breeder, large thermal gradients in the solid lithiated ceramics will induce thermal stresses across large characteristic lengths. Avoiding thermal stress has led to most solid breeder designs implementing packed beds of small, spherical (or near-spherical) pebbles. Moreover, tritium diffusion and release considerations for solid lithium ceramic support the choice of short characteristic lengths of individual pebbles. From an engineering design standpoint, the choice of packed bed has other desirable characteristics. For instance, the ensemble of small spherical pebbles can be filled into many complex shapes with relatively uniform porosity for well-distributed flow of the purge gas. 

Pebble beds of these materials have been chosen by many participants of the International Thermonuclear Experimental Reactor (ITER) for tritium generation studies.\cite{Lulewicz2002, Mandal2012a, Tsuchiya1998, Cho2012} The advantages of the pebble bed design include ease of uniformly assembling the solid into complex geometries, ease of tritium extraction from the porous bed via an interstitial purge of helium, and with the small size of pebbles being more resilient to thermal stresses than a solid brick of lithiated ceramic.\cite{Casadio2004} Nothing comes without a price, naturally, and there are many issues with pebble bed tritium breeders that must be understood and overcome.

TBM unit design

In typical solid breeder modules, coolant fluid runs through the containing structure surrounding the pebble bed. A low-pressure, low-speed purge gas is pumped through the pebble bed to extract the tritium generated and transport it out of the blanket for processing. As nuclear energy is deposited into the poorly-conductive ceramic breeder material, the temperature climbs well above the containing structure. Heat is removed from the pebble bed predominately through inter-particle conduction (with a small contribution from convection of the purge gas) and contact conductance of many pebbles pressed against the containing surface. 

The packed bed will be contained in a structure of ferritic or austenitic steel. The energy of the packed bed is carried away by coolant channels in the structure that have flowing in them high pressure helium gas. Because the structural material is held cooler than the breeding zone, it will confine the thermal expansion of the lithium ceramic and lead to mechanical stresses at the points of contact of the individual pebbles in the packed bed. Engineering design issues surrounding this thermal stress is of great concern to researchers and will be the focus of much of this report.

Once in operation, the ceramic pebble beds will have a specified operating temperature window that is dictated by tritium release characteristics. The low end of the temperature window is governed by a minimum temperature for acceptable release rates of tritium from the ceramic to the purge gas; the value is generally set around 300~\celsius. The upper limit of the temperature window is chosen to avoid sintering of the lithiated ceramic. Sintering of the ceramics, as grains in individual pebbles meld, is predicted to reduce the rate of tritium release. The upper end of the temperature window is generally set around 900~\celsius.

The size of breeder regions is limited by the temperature window combined with the poor effective conductivity of packed beds of ceramic pebbles. The conductivity is a weak function of external pressure but can generally be approximated as about \si{1 W/{mK}}. Because the effective conductivity and packed bed-wall interface conductance is predominately a contact conduction, disruptions to the packing structure will have considerable impact on the heat transfer of the packed bed.

The fusion reaction deposits a great deal of surface radiation on the first wall of the breeding blanket and the blanket will be absorbing energy deposited from neutron interactions and $\gamma$ rays. The blanket must be capable of converting and then recovering the energy at high tempreatures for efficient power production in the fusion power plant. The nuclear heat generated in the pebble bed solid breeder will heat the ceramic pebbles to maximum temperatures of approximately 900~\celsius. The heat of the pebbles is transported through them via conduction through inter-particle contacts, conduction through the purge gas into neighboring particles, and ultimately through contact with the containing structure. The box structure surrounding the solid breeder will have high pressure (\si{8~MPa} in many current designs) helium flowing through channels that remove the heat from the blanket, with the helium reaching temperatures of 500~\celsius, and into a power generation cycle.