\section{Importance of pebble bed integrity and motivation}\label{sec:intro-bed-integrity}


% From ITER 2013
Control of the manufacturing processes of the ceramic pebbles permits manufactureres to custom vary characteristics, such as the pebble's:
\begin{itemize}
\item tritium retention and release properties.
\item Lithium density
\item Opened- and closed-porosity
\item Nominal diameter
\item and, indirectly, crush strength. 
\end{itemize}
However the characteristics of the pebble are often coupled. For instance, for the sake of tritium management the open porosity of the pebble is often increased. But this comes at the expense of a decreased crush strength of the pebble. Because of the relatively weak crush strength distributions among batches of pebbles as well as the value of stresses predicted in the pebble bed, it is inevitable that during operation in the fusion environment individual pebbles will `fail' in the ensemble. Designers of lithium ceramic tritium breeding blankets must mitigate pebble failure but also anticipate the breadth and magnitude of effects that some unavoidable failure will have on macroscopic properties.

% [EDIT: THIS PARAGRAPH IS NOT NECESSARY? I DON'T NEED TO MAKE THE CASE FOR USING DEM. I JUST NEED TO EXPLAIN THE MODEL]The volume of a pebble in a tritium breeder is on the scale of 10$^{-9}$~m$^3$ while the typical container volume can be on the order of 10$^{-2}$~m$^3$\cite{Cho2008}.  Thus a single breeder volume will house upwards of $N = 10^7$ pebbles. Statistically then, the behavior of any single pebble seems insignificant and instead the entire ensemble of pebbles may be treated as a continuous media. Continuum theory for the is the basis of finite element method models that have been able to predict thermomechanical behavior with reasonable accuracy\cite{DiMaio20081287,Zaccari20081282,Gan:2009vn}. However, after the pebble beds are placed into the fusion environment they will be required to operate for long duty times without maintenance. Thus, as time progresses the accumulation of individual failed pebbles will eventually have consequences for the macroscopic thermomechanics.  and no continuum theory exists to account for this. Instead, we turn to the discrete element method to provide a solutino.
