\section{Pebble Bed Integrity, Thermophysics, and the Role of Modeling}\label{sec:intro-bed-integrity}

The pebble bed will experience a constrained thermal expansion as the hot ceramic pebbles press against the relatively cooler container. The restricted thermal expansion of the pebble bed causes an external pressure on the pebble bed. The external pressure may lead to a number of phenomena that disrupt the initial packing -- and by extension the initial predictions of thermal and mechanical properties -- of the pebble bed. For one, in experiments of even well-packed ensembles of pebbles, the beds show an apparent plastic strain of rearranged packing that increases with maximum historical stress on the bed.[cite Chunbo and Reimanns experimental papers] Without careful engineering and packing of the virgin pebble beds, plastic strain in the pebble bed will directly lead to the formation of gaps between the pebble bed and containing structure. Depending on the configuration of the solid breeder design, the gap could cause a tremendous loss of heat transport from the pebble bed to the coolant. Furthermore, any gaps formed in the blanket could lead to more neutron leakage and decreased tritium breeding ratios and detriment to the blanket's shielding function.

Additionally, assuming that the plastic strain is removed from the pebble bed, the thermally-induced pressure on the pebble bed will be balanced by the individual pebbles pressing into each other at small points of contact. The small area over which the contact forces are applied leads to stresses which may crush individual pebbles in the ensemble. With the potential accumulation of many cracked/crushed individual pebbles, the overall packing structure is again altered. Depending on the extent of crushing, the response of the pebble bed may be as benevolent as a negligible decrease in effective thermal conductivity or malevolent as a loss of physical contact and heat transport from the pebble bed to the coolant. 

Finally there are long-term effects expected in the materials experiencing prolonged exposure to cycling irradiation, heat, and stress. Thermal ratcheting, swelling, sintering, or thermally-induced creep can lead to evolutions in thermophysical properties even in the absence of cracked pebbles. As the thermophysical properties evolve, global or local bed temperatures change and ultimately the tritium release characteristics of the bed deviate from any prediction one may have had from the initial packing of the ceramic pebble bed. 

In our group we are most focused on maintaining tritium breeding characteristics of the pebble bed at desirable levels and thus maintaining temperatures in the breeding region. Alleviating any of the issues that may plague the ceramic breeder all boil down to requiring temperature control via an understanding and of the morphological changes of the ceramic packed beds and their interaction with the interstitial purge gas and structural container. [say how temperature control is possible with better models]In this work we introduce enhancements and new elements to build upon the understanding from ceramic breeder models of past research efforts. 







% Control of the manufacturing processes of the ceramic pebbles permits manufactureres to custom vary characteristics, such as the pebble's:
% \begin{itemize}
% \item tritium retention and release properties.
% \item Lithium density
% \item Opened- and closed-porosity
% \item Nominal diameter
% \item and, indirectly, crush strength. 
% \end{itemize}
% However the characteristics of the pebble are often coupled. For instance, for the sake of tritium management the open porosity of the pebble is often increased. But this comes at the expense of a decreased crush strength of the pebble. Because of the relatively weak crush strength distributions among batches of pebbles as well as the value of stresses predicted in the pebble bed, it is inevitable that during operation in the fusion environment individual pebbles will `fail' in the ensemble. Designers of lithium ceramic tritium breeding blankets must mitigate pebble failure but also anticipate the breadth and magnitude of effects that some unavoidable failure will have on macroscopic properties.



% [EDIT: THIS PARAGRAPH IS NOT NECESSARY? I DON'T NEED TO MAKE THE CASE FOR USING DEM. I JUST NEED TO EXPLAIN THE MODEL]The volume of a pebble in a tritium breeder is on the scale of 10$^{-9}$~m$^3$ while the typical container volume can be on the order of 10$^{-2}$~m$^3$\cite{Cho2008}.  Thus a single breeder volume will house upwards of $N = 10^7$ pebbles. Statistically then, the behavior of any single pebble seems insignificant and instead the entire ensemble of pebbles may be treated as a continuous media. Continuum theory for the is the basis of finite element method models that have been able to predict thermo-mechanical behavior with reasonable accuracy\cite{DiMaio20081287,Zaccari20081282,Gan:2009vn}. However, after the pebble beds are placed into the fusion environment they will be required to operate for long duty times without maintenance. Thus, as time progresses the accumulation of individual failed pebbles will eventually have consequences for the macroscopic thermo-mechanics.  and no continuum theory exists to account for this. Instead, we turn to the discrete element method to provide a solutino.
