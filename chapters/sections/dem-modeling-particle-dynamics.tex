\subsection{Particle Dynamics}\label{sec:particle-dynamics}

The particles in our system are allowed translational and rotational degrees of freedom. In a packed bed, we can restrict our attention to local forces between particles; neglecting, say, non-contact forces such as van der Waals or electrostatic forces. In the first construct of momentum and temperature consideration, I will treat the particles as if in a vacuum. However a derivation of fluid interaction forces will be given in \cref{sec:modeling-cfd-dem}.



%~~~~~~~~~~~~~~~~~~~~~~~~~~~~~~~~~~~~~~~~~~~~~~~~~~~~~~~~~~~~~~~~~~~~
\subsubsection{Particle Kinematics}

Assuming we know the contact forces acting upon particle $i$, Newton's equations of motion are sufficient to describe the kinematics of the particle. For the translation and rotational degrees of freedom, the equations are:,
\begin{subequations}
\label{eq:newtons-second}
\begin{align}
	m_i  \ddt{\vec{r}_i}   & = m_i\vec{g} + \vec{f}_i \label{eq:newton-translational} \\
	I_i\dt{\vec{\omega}_i} & = \vec{T}_i \label{eq:newton-rotational}
\end{align}
\end{subequations}
where $m_i$ is the mass of this particle, $\vec{r}_i$ its location in space, $\vec{g}$ is gravity, $I_i$ is the particle's moment of inertia, and $\vec{\omega}_i$ its angular velocity.

The net contact force, $\vec{f}_i$, represents the sum of the normal and tangential forces from the total number of contacts, $Z$, acting on this particle.
\begin{equation}
 	\vec{f}_i = \sum_{j=1}^{Z} \vec{f}_{n,ij} + \vec{f}_{t,ij}
 \end{equation} 
and the net torque, $\vec{T}_i$, is similarly,
\begin{equation}
	\vec{T}_i = -\frac{1}{2}\sum_{j=1}^{Z} \vec{r}_{ij} \times \vec{f}_{t,ij}
\end{equation}

When Cundall and Strack first proposed the discrete element method, they used a linear spring-dashpot structure which saw the normal and tangential forces written as,
\begin{subequations}
\label{eq:dem-forces}
\begin{align}
	\vec{f}_{n,ij} &= k_{n,ij} \delta_{n,ij}\vec{n}_{ij} - \gamma_{n,ij} \vec{u}_{n,ij} 	\label{eq:normal-force} \\
	\vec{f}_{t,ij} &= k_{t,ij} \delta_{t,ij}\vec{t}_{ij} - \gamma_{t,ij} \vec{u}_{t,ij} 	\label{eq:tangential-force}
\end{align}
\end{subequations}
where, in the first model of Cundall and Strack, the stiffness coefficients $k$ were constants and the local damping coefficients $\gamma$ were proportional to them, $\gamma \propto k$, to allow dissipation of energy and the system to reach an equilibrium. The relative normal and tangential velocities, respectively, are decomposed from the particle velocities,

\begin{subequations}
\label{eq:dem-velocities}
\begin{align}
	\vec{u}_{n,ij} &= (-(\vec{u}_i-\vec{u}_j)\cdot\vec{n}_{ij})\vec{n}_{ij} \\
	\vec{u}_{t,ij} &= (-(\vec{u}_i-\vec{u}_j)\cdot\vec{t}_{ij})\vec{t}_{ij}
\end{align}
\end{subequations}
with the unit vector $\vec{n}_{ij}$ pointing from particle $j$ to $i$

Similarly to the approach of Hertz (see \cref{sec:hertz-theory}), the surfaces of the two particles are allowed to virtually pass through each other (no deformation) resulting in normal and tangential overlaps of,

\begin{subequations}
\label{eq:dem-overlaps}
\begin{align}
	\delta_{n,ij} &= (R_i + R_j) - (\vec{r}_i -\vec{r}_j)\cdot \vec{n}_{ij} \\
	\delta_{t,ij} &= \int_{t_{c,0}}^{t} \vec{u}_{t,ij}\,\mathrm{d}\tau 
\end{align}
\end{subequations}
where the fictive tangential overlap, $\delta_{t,ij}$, is truncated to so the tangential and normal forces obey Coulomb's Law, $\vec{f}_{t,ij} \le \mu_i \vec{f}_{n,ij}$ with $\mu$ as the coefficient of friction of the particle.

The result is a relatively simple approach of calculating the interaction forces between particles with Eq.~\ref{eq:dem-forces} based on the kinematics of velocity and position of the interacting particles from Eq.~\ref{eq:dem-velocities} and Eq.~\ref{eq:dem-overlaps}, respectively. As the DEM evolved and drew the attention of more researchers, more complex formulas governing the spring-dashpot coefficients of Eq.~\ref{eq:dem-forces} emerged. But the core approach remained the same and the models all fall into the same family of so-called `soft particle' models of DEM. A well-composed summary of the different DEM force models is given by Zhu\etal\cite{Zhu2007}.

The method used in this work fits into the computational skeleton of Cundall and Strack's method but with non-linear spring-dashpot coefficients defined by simplified Hertz-Mindlin-Deresiewicz model. In this model, the normal-direction stiffness coefficient of Eq.~\ref{eq:normal-force} is based on the Hertzian contact law (derived explicitly in \cref{sec:hertz-theory}). The tangential-direction stiffness coefficient follows from Brilliantov.\cite{Brilliantov1996, Zhu2007, Langston1995} Together, the spring coefficients are,
\begin{subequations}
\begin{align}
	k_{n,ij} &= \frac{4}{3}E_{ij}^*\sqrt{R_{ij}^*\delta_{n,ij}} \\
	k_{t,ij} &= 8 G_{ij}^*\sqrt{R_{ij}^*\delta_{t,ij}}
\end{align}
\end{subequations}
where $E_{ij}^*$ is the pair Young's modulus, $G_{ij}^*$ is the pair bulk modulus, and $R_{ij}^*$ is the relative radius. The terms are defined as,
\begin{subequations}
\begin{align}
	\frac{1}{E^*} &= \frac{1-\nu_1^2}{E_1} + \frac{1-\nu_2^2}{E_2} \\
	\frac{1}{R^*} &= \frac{1}{R_1} + \frac{1}{R_2} \\
	\frac{1}{G^*_{ij}} &= \frac{2(2+\nu_i)}{E_i} + \frac{2(2+\nu_j)}{E_j}
\end{align}
\end{subequations}

The damping coefficients, $\gamma$, arise to account for the energy dissipated from the collision of two particles\cite{DiRenzo2004, Tsuji1992, Tsuji1993}. Whether the damping coefficient is local or global and the exact form of the coefficient is more important for loosely confined granular systems and dictates the way the system approaches an equilibrium state\cite{Makse2004}. For the case of our tightly packed pebble beds, it suffices to use the efficient form of\cite{Dippel1996, Makse2004, Brilliantov1996, Zhang2005, Zhu2007},
\begin{subequations}
\begin{align}
	\gamma_n &= \sqrt{5}\beta_\text{diss}\sqrt{m^*k_{n,ij}} \\
	\gamma_t &= \sqrt{\frac{10}{3}}\beta_\text{damp}\sqrt{k_{t,ij} m^*}
\end{align}
\end{subequations}
with $\beta_\text{damp}$ as the damping ratio, and the pair mass, $\frac{1}{m^*} = \frac{1}{m_i} + \frac{1}{m_j}$. For a stable system with $\beta_\text{damp} < 1$, the damping ratio is related to the coefficient of restitution, $e$, as
\begin{equation}
	\beta_\text{diss} = -\frac{\ln{e}}{\sqrt{\ln^2{e}+\pi^2}}
\end{equation}

Systems are therefore well-defined after specifying the few material properties of $E$, $\nu$, $\rho$, and $R_p$ and the interaction properties of $\mu$ and $e$.

Having expressed the contact mechanics of the discrete element method, we now must integrate the kinematic equations of the particles to resolve their evolutions. The most common means of marching in time with DEM is the velocity-Verlet algorithm\cite{Kruggel-Emden2008}. In this algorithm, Eqs.~\ref{eq:newtons-second} are integrated with half-steps in velocity, full steps in position, and then finally the full step in velocity. In practice, the two half-steps in velocity are often compressed into a single, full step. The computational time integration steps are given in explicit detail in \cref{sec:dem-stability}. Owing to the explicit nature of the velocity-Verlet algorithm, stability is a constant concern with DEM simulations. Stable, critical time steps and means of circumventing unreasonably small time steps will also be addressed in \cref{sec:dem-stability}.

Lastly, in this work, I occasionally required a fully quiesced bed to act as a starting point or demarcate a mechanically steady-state bed. To determine when this occurs, the total kinetic energy of the entire ensemble is monitored and a packed bed is considered to have completely settled once the kinetic energy of the system is less than $10^{-8}$. A similar process was independently determined in a similar matter in the work of Ref.~\cite{Silbert2002}. 