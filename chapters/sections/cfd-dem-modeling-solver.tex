\section{Numerical Implementation of CFD-DEM}\label{sec:cfd-dem-solver}

The infrastructure for solving the DEM equations continues to be handled by LIGGGHTS. Details of the software are described in \cref{sec:dem-solver}. The DEM solver is a highly parallel C++ code based on the Molecular Dynamics (MD) code LAMMPS.\cite{Plimpton1995}

Taking advantage of a separate, stand-alone CFD solver that is maintained by a large community, the CFD simulations are conducted by the pressure-based solver using the PISO algorithm realized within the open-source framework of OpenFOAM\textsuperscript{\textregistered}.\cite{Issa1986,OpenCFDLtd2014} The coupling routines, maintained by DCS Computing GmbH, are collected in a library providing a modular framework for CFD-DEM coupling with the C++ codes LIGGGHTS and OpenFOAM\textsuperscript{\textregistered}.\cite{Kloss2012,Goniva2012}

% The helium purge gas generally flows at very small Reynolds numbers. Particle Reynolds numbers on the order of unity, $\Re\sim1$, are expected for many tritium breeding volumes. 

The routine of coupling CFD-DEM consists of several steps:
\begin{enumerate}
\item the DEM solver calculates the particles positions, velocities, and temperatures with timestep dictated by stability of DEM
\item the particles positions and velocities are passed to the CFD solver using the Message Passing Interface (MPI)
\item for each particle, the cell in the CFD mesh that contains the particle is located
\item for each cell, the particle volume fraction is determined from the divided technique described in \cref{sec:lag-eul-mapping}. The ensemble-average velocity of the particles is determined
\item on the basis of $\epsilon$ and $\Re_p$, the fluid forces and heat transfer rate acting on each particle are calculated from the inter-phase exchange coefficients of Eqs.~\ref{eq:interphase-momentum} and~\ref{eq:interphase-energy}
\item the momentum and energy source/sink terms are assembled from particle-based forces by ensemble averaging over all particles in a CFD cell via Eqs.~\ref{eq:cfd-sources}
\item the inter-phase exchange coefficients of Eqs.~\ref{eq:interphase-momentum} and~\ref{eq:interphase-energy} are sent to the DEM solver
\item the CFD solver calculates the fluid velocity and temperature from the source/sink terms determined in step 6.
\item the routine is repeated from step 1
\end{enumerate}
