
\section{Conclusions}
\label{concs}
The current study aimed at properly simulating a pebble bed with a specified fraction of the pebbles failing during operation; then determining the repercussions of the failures as they affect the macroscopic property of effective thermal conductivity. We used the assumption of homogeneous, random locations of pebble failure to induce a failure routine without requiring external loads on the bed to permit beds that could be directly compared. After heating to a steady-state, an effective thermal conductivity was calculated for the pebble bed. The results show that small amounts of pebble failure correspond to large decreases in the conductive transport of energy through the pebble bed. The increase was due primarily to a drop in the inter-particle forces which lead to a large increase in temperature differences between neighboring pebbles. We note again, however, that this value has been calculated in the absence of interstitial gas so the results apply only to the reduction in energy transferred via inter-particle conduction.


The assumption of homogeneous distribution of pebble failure was found to be inappropriate after a pebble bed reached steady state nuclear heating. The scheme assumes no localization of average forces in the bed but we found an average force profile that had a maximum at the center and minimum at the walls. The next step of modeling will eliminate the error of such an assumption as we must combine failure prediction to failure outcome modeling. 


