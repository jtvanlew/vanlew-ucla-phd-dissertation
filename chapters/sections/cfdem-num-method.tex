\section{Numerical Methodology}
% From TOFE 2014
Models based on the discrete element method (DEM) are currently the only tools available that can extract information on individual pebble interactions. The DEM formulation provides information such as inter-particle forces and individual particle temperatures, which are necessary for predicting and simulating morphological changes in the bed (e.g. pebble cracking, sintering, etc.) However DEM alone is not able to capture the effects, neither on momentum nor energy, of an interstitial fluid. Therefore we present two fluid modeling techniques to supplement the DEM computations. We will first discuss the fully dynamic coupling of the DEM model with a volume-averaged thermofluid model of helium. Then we will introduce the integration of our DEM packing structure into lattice-Boltzmann simulations of the entire bed-fluid system.

\subsection{DEM}
The discrete element framework introduced in \S~\ref{sec:particle-dynamics} is augmented with a drag force term to capture interaction with surrounding fluid velocity fields. To accomplish this, we simply include a drag force to the Newtonian balance of forces given in Eq.~\ref{eq:newtons-first}. The momentum balance now reads:

\begin{equation}\label{eq:cfdem-dem-momentum}
	m_i  \ddt{\vec{r}_i} = m_i\vec{g} + \vec{f}_i + \beta_i V_i \Delta u_{if}
\end{equation}

where $\Delta u_{if} = u_f - u_i$, is the relative velocity between the fluid and pebble, $i$, and the inter-phase momentum exchange coefficient, $\beta_i$, acts upon the pebble volume (not to be confused with the damping coefficient introduced in \S~\ref{sec:particle-dynamics}). Similarly, the energy equation now includes 

\begin{equation}\label{eq:cfdem-dem-energy}
	m_iC_i \ddt{T_i} = Q_{n,i} + \sum_{j=1}^Z Q_{ij} + \beta_{E,I} A_i \Delta T_{if}
\end{equation}

where $\Delta T_{if} = T_f - T_i$, is the relative temperature between teh fluid and pebble, $i$, and the inter-phase energy exchange coefficient, $\beta_{E,i}$, acts upon the pebble surface area, $A_i$.

The trajectory of pebble $i$ is updated based on the force terms on the right hand side of Eq.~\ref{eq:cfdem-dem}: gravity, contact forces between particles (or particle-wall), and a drag force. Similarly, the temperature of the particle updates with the terms from Eq.~\ref{eq:cfdem-dem-energy}: nuclear heating rate, inter-particle conduction, and now a heat transfer with surrounding fluid.

Drag forces from fluid flows through packed beds are found from volume-averaged, empirical correlations of either numerical or experimental studies. Considering a small region of a packed bed surrounding our particle of interest, i, the nondimensional drag force is found only as a function of the local packing fraction of that region. In the zero Reynolds number limit, the nondimensional drag force reduces to a Stokes flow correlation that is only a function of the local packing fraction value, $\phi$. For the value of particle Reynolds numbers seen by the helium purge gas, this is the dominant term. However, for a complete discussion of the nondimensional drag terms see Refs. 5, 6. The correlation used in this study comes from the results of numerical studies of packed beds by Koch and Hill~\cite{Koch2001, Gruber2012, Benyahia2006}. To arrive at their relationships, they did many lattice-Boltzmann simulations of porous flow.

\begin{equation}
	\beta_{i} = \frac{18\nu_f\rho_f}{d_{i}^2}(1-\phi) F
\end{equation}

where 

\begin{equation}
	F = \epsilon (F_0 + \frac{1}{2}F_3 \Re_{p,i})
\end{equation}

Stokes flow

\begin{align}
F_0 = 
	\begin{cases}
    		\frac{1+3\sqrt{\phi/2} + (135/64)\phi\ln(\phi) + 17.14\phi}{1 + 0.681\phi - 8.48\phi^2 + 8.16 \phi^3}	& \text{if } \phi < 0.4\\
    		10\frac{\phi}{(1-\phi)^3}              																& \phi > 0.4
	\end{cases}
\end{align}

and high Reynolds contribution

\begin{equation}
	F_3 = 0.0673 + 0.212\phi + \frac{0.0232}{(1-\phi)^5}
\end{equation}



The packing fraction and  void fraction in any fluid cell is calculated by summing through all the volumes of $k$ particles located in that cell (or the complement thereof)

\begin{equation}
	\phi = \sum_{i=1}^k \frac{V_{p,i}}{\Delta V_f}
\end{equation}

% \begin{equation}
% \epsilon = 1 - \sum_{i=1}^k \frac{V_{p,i}}{\Delta V_f}
% \end{equation}
Other forces, such as Magnus forces, are inconsequential on predominantly stationary packed beds and are not considered.



The inter-phase energy transfer coefficient is of the same form as a traditional heat transfer coefficient and is calculated from the Nusselt number for the helium flow (with conductivity $k_f$) through a packed bed.

\begin{equation}
	\beta_{E,i} = \frac{\Nu_i k_f}{d_i}
\end{equation}

Li and Mason\cite{Li2000} summarize correlations for Nusselt number as a function of Reynolds number for packed beds with the following equations
\[
\Nu= 
\begin{cases}
    2 + 0.6(1-\phi)^n\Re_p^{1/2}\Pr^{1/3}											& \Re_p < 200 \\
    2 + 0.5(1-\phi)^n\Re_p^{1/2}\Pr^{1/3} + 0.2(1-\phi)^n\Re_p^{4/5}\Pr^{1/3}   & 200 < \Re_p \le 1500 \\
    2 + 0.000045(1-\phi)^n\Re_p^{9/5}												& \Re_p > 1500
\end{cases}
\]
where $n=3.5$ was found to fit best for small particles in dilute flows. [we should find a new value for high packing fraction] 

Thus we have a formulation whereby a known fluid flow field and temperature throughout the domain, we can calculate the influence of that fluid on every particle’s position and temperature. Next we will cover how we can calculate the flow field based on a volume-averaged influence of particles on the fluid.




\subsection{Volume-averaged CFD Helium}
The technique of coupling CFD to DEM was first proposed by Tsuji, et al9. In this formulation of the helium flow, a fluid cell is much larger than the individual particles (in application, this meant approx. 5~6 particles per cell) and as such, the particles themselves are not resolved in the fluid space but are simply introduced via volume-averaged terms. Therefore momentum and energy of a fluid flow through a solid phase is governed by volume-averaged Navier-Stokes and energy equations10. These equations are applied to a discretized volume of fluid space. For fluid cell, k, these are5:



\begin{align}
\pder[\epsilon_k \rho_f]{t} + \nabla\cdot(\epsilon_k u_f \rho_f) &= 0\\
\pder[\epsilon_k u_f]{t} + \nabla\cdot(\epsilon_k u_f u_f) &= -\frac{\epsilon_k}{\rho_f}\nabla P_f + \nabla\cdot\left(\nu_f\epsilon_k\nabla u_f\right) - \frac{S_k}{\rho_f}\\
\pder[\epsilon_k T_f]{t} + \nabla\cdot(\epsilon_k u_f T_f) &= \nabla\left(\epsilon_k\epsilon\nabla T_f\right)-\frac{E_k}{\rho_fC_f}
\end{align}

where the fluid void fraction is the complement of the solid packing fraction, $\epsilon = 1 - \phi$. The momentum and energy exchanges with the solid phase are represented in the source terms. They are volume-weighted sums of the drag forces and energy exchanges, respectively, for all particles in the discretized fluid cell:

\begin{align}
S_k &= \frac{1}{V_k}\sum_{\forall i \in k} \beta_i V_i \Delta u_{if}\\
E_k &= \frac{1}{V_k}\sum_{\forall i \in k} \beta_{E,i} A_i \Delta T_{if}
\end{align}

The inter-phase momentum and energy exchange coefficients act as the communicators between the particle information from the DEM solver and the fluid fields from CFD. Thus the motion and energy of the fluid field are intimately coupled with the particle positions and energy, but computational time is preserved by only considering volume-averaged values in the fluid domain. The cross-communication between fluid and solid is accomplished with a coupling routine that is explained in detail in Refs. 11, 12.


% In the discrete element method, we showed the forces acting on a particle with Eq.~\ref{NewtonsFirst}. It is given again here,
% \begin{align*}
%  F_i = m_i g + \sum_{j}^Z&\left(k_n\delta_{n_{ij}} - \gamma_n v_{n_{ij}}\right) + \left(k_t\delta_{t_{ij}} - \gamma_t v_{t_{ij}}\right)
% \end{align*}

% The influence of the fluid phase is introduced through a new drag force term, $f_{d,i}$:
% \begin{align}
%  F_i = m_i g + f_{d,i} + \sum_{j}^Z&\left(k_n\delta_{n_{ij}} - \gamma_n v_{n_{ij}}\right) \left(k_t\delta_{t_{ij}} - \gamma_t v_{t_{ij}}\right)
% \end{align}




\subsection{Modeling Setup and Procedure}
The pebble bed has dimensions in the x-y directions of 20d×15d, respectively. There are structural walls, providing cooling, at the x-limits and periodic walls in the y-limits. 10 000 pebbles were loaded into the system which went to a height of approximately 24d after the bed was vibration packed. The pebble bed had a roof loaded at the upper limit of the z-direction that was lowered by force-control up to 6 MPa. This bed is referred to as the ‘well-packed’ bed. This was meant to simulate a fresh, densely-packed bed that is under compressive load during fusion operation. As such, this would be when pebbles would be likely to crack during operation. Therefore, based on the well-packed bed, a second bed was generated by simulating crushed pebbles; crudely the extensive crushing is simulated by simply removing 10\% of the pebbles at random from the ensemble and then allowing the bed to resettle, from the now-imbalanced gravity and inter-particle forces, to a new stable packing structure. This bed is then referred to as the ‘resettled’ bed for the rest of the analysis. The intent is to deduce changes in thermomechanical properties from an ideally packed bed to one where significant cracking has altered the ideal morphology of the bed.