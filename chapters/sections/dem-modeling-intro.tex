\section{Background}
\label{sec:dem-intro}

The observable, macroscopic behavior of particulate, or granular, systems is a complex function of the multitude of particle-scale interactions. Historically, empirical relationships have been used to describe these systems as if a continuous media. But with the advent of the discrete element method by Cundall and Strack\cite{Cundall1979} and the acceleration of computing power, it became practical to investigate these particulate systems at the particle scale. With the discrete element method, we track all the particles in the system in a Lagrangian manner. In the ensemble, the kinematics of each particle is tracked and updated based on balances (or imbalances) of forces or energy acting upon the particle.

Experiments on packed beds are generally limited to measurements of statistically averaged, macroscopic responses. Unlike continuous materials, packed beds as yet can not be described by any State. For instance, with an ideal gas if we know two properties such as temperature or pressure, the State of the gas is known and its behavior predicted. Two packed beds with the same temperature, packing fraction, average particle diameter, and stress state may react wildly different. As researchers we create empirical fits to data on the particular packed bed under investigation but then might have to dubiously apply the relationships to beds of different packing states. 

DEM is emerging as a reliable method to remove speculation about the internal state of the packed bed as the simulations provide valuable information on the dynamics of particle interactions and how they relate to the macroscopic responses that are measured experimentally.

In this chapter we will lay out the formulas governing interaction of particles in the DEM framework, the methods of computation, and the code used for implementation. In \S\ref{sec:dem-stability}, we will use the derivation of the Hertz contact law described in \S\ref{sec:hertz-theory} to argue for a technique of accelerating the computational time without loss of physics via proper scaling of physical parameters. Lastly, in \S\ref{sec:dem-studies-effective-conductivity}, we use our DEM tools to investigate the thermo physics of representative packed beds for solid breeders in fusion reactors.
