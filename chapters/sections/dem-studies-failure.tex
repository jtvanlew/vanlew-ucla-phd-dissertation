
\section{Pebble failure modeling}
\label{sec:failure-discussion}
Research on pebble failure has been taken up by others in the fusion community (see \cref{sec:failure-discussion}) to predict the onset of pebble crushing as a function of an external pressure and the resulting changes to mechanical properties such as the stress-strain of the pebble bed. 


%In modeling pebble failure, there are two main tasks. The first is to develop a model for predicting a pebble failure event; { i.e.} what load (mechanical or thermal) will cause a pebble to crack, shatter, fracture, etc. The second is to develop a model which simulates the failure of that pebble; { i.e.} a scheme to treat a cracked, shattered, or crushed pebble in the assembly. 

The discrete element method has been used for studies in a variety of fields for studying inter-particle forces and the homogeneously distributed force networks that arise in packed beds (for example, see Ref.~\cite{Makse2000}). The discrete element method was also used in the fusion community to attempt to model failure initiation and propagation\cite{Annabattula2012a, Zhao2012, Zhao2013}. They too observed that a relatively few number of high-force networks, distributed troughought the bed supported the external mechanical loads. The even distribution of the force networks was used to defend the development of a probability-based predictor for failure. We make use of the probability argument of Zhao, {et al.} for the current study\cite{Zhao2013}. Their basic premise is that probability distributions of strength curves for pebble crushing have been observed (see, for example crush loads of Ref.~\cite{Tsuchiya1998}). Then in DEM models, a probability distribution of inter-particle forces are also observed. Overlaying the two probabilities resulted in seemingly random locations of pebbles satisfying the failure criteria -- not strictly along the high-force chains running through packed beds.

We apply the theory of Zhao, { et al.} in the following manner. If pebbles fail at random locations, we may de-couple the task of predicting pebble failure ({ i.e.} finding the mechanical or thermal load that causes a pebble to fail) from the task of modeling the ramifications of pebble failure. In our model, we begin with a starting point of a packed bed and then simply flag pebbles at random for `failing'. For our first model of failure, after a pebble has been flagged it is removed from the system entirely. The removal disrupts the meta-static state of the ensemble and the remaining pebbles re-settle. In reality, the ceramic pebbles generally break into just a few large pieces that remain in the system. Under development is a method for recreating that behavior in the DEM domain, it will be reported in future studies.

%Experiments on crushing single, brittle pebbles reveal that there are a number of failure modes\cite{Wu2004}. At one end, the pebble may simply crack and continue to hold a load for some time. At the other extreme, a pebble may crush virtually into a dust. We concern ourselves with the latter for this study. When a pebble in our simulation has been flagged for failure, we remove the pebble completely from the ensemble and then allow the remaining pebbles to rearrange to compensate for the lack of equilibrium on their contact forces. 


%\section{Simulation methods}
%\label{back} 