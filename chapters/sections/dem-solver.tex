\section{DEM solver}\label{sec:dem-solver}

Time-discretization of the integration of Eq.~\ref{eq:newtons-first} is handled by the core Large-scale Atomic/Molecular Massively Parallel Simulator (LAMMPS) code released by Sandia National Laboratories\cite{Plimpton1995, Parks2008}. The code calculates velocity and position via the semi-explicit velocity-Verlet integration. The algorithm is stable with a global error of approximately $O(\Delta t^2)$ for displacement; details can be found in Ref.~\cite{Grubmuller1991}.

In the process of the study, to demonstrate the ability of the dynamic integration to capture resettling (and any possibly asymmetries), some beds were generated wherein the failure of pebbles was slightly localized near one or both $x$-walls. The profile of the pebbles near the top of the stack, after resettling, are shown in Fig.~\ref{fig:settlingStudy}. 

In our work, we occasionally required a fully quiesced bed. To determine when this occurred, the total kinetic energy of the entire ensemble was monitored and a packed bed was considered to have completely settled once the kinetic energy of the system was less than $10^{-9}$; similar to the process described in Ref.~\cite{Silbert2002a}. 


The granular heat transfer equations (Eqs.~\ref{conductance}-\ref{thermoFirstLaw}) are layered onto the LAMMPS code via a package of code named LIGGGHTS (LAMMPS Improved for General Granular and Granular Heat Transfer Simulations \cite{kloss2012a}). Parallelization of the code is straightforward with LAMMPS and we run the code on 128 nodes of UCLA's Hoffman2 cluster for typical run times of 18 to 24 hours per routine ({e.g.} filling, packing, heating, etc.).