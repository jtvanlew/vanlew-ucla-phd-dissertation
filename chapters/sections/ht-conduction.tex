\subsection{Inter-particle heat conduction}\label{sec:ht-pebble-conduction}

When two particles come into contact, energy is transmitted through their region of contact. For this discussion, we assume the particles are spherical, elastic, in vacuum, and we neglect radiation transfer between them. The resistance to heat flowing between the two objects is quantified through a contact conductance, $H_c$. The amount of energy passing between the two particles (labeled $i$ and $j$) is then

\begin{equation}\label{eq:pebble-conduction-heat-transfer}
	Q_{ij} = H_{c}(T_i - T_j)
\end{equation}

where each contacting pair has a specific value of $H_c$. We note that the heat conductance, unlike standard heat transfer coefficients, has units of \si{W/K}.

Batchelor and O'Brien\cite{Batchelor1977} developed a formulation of similar form and then made the brilliant observation that the temperature fields in the near-region of contacting spheres are analogous to the velocity potential of the an (incompressible, irrotational) fluid passing from from one reservoir to another through a circular hole in a planar wall separating the two reservoirs. With the analogy, they could make use of the fluid flow solution to write the total flux across the circle of contact as Eq.~\ref{eq:pebble-conduction-heat-transfer} with the heat conductance as

\begin{equation}\label{eq:batchelor-pebble-conductance}
	H_c = 2k_ra
\end{equation}

where $k_r$ is the conductivity of the contacting solids and $a$ is the radius of contact. In \cref{sec:hertz-theory}, with Hertz theory we found the contact radius in terms of the contact pressure. Here, we give the radius in terms of the compression force acting on the bodies,

\begin{equation}
	a =  \left(\frac{3}{4}\frac{R^*}{E^*}\right)^{1/3}F_n^{1/3}	
\end{equation}

as before, $\frac{1}{E^*} = \frac{1-\nu_1^2}{E_1} + \frac{1-\nu_2^2}{E_2}$ and $\frac{1}{R^*} = \frac{1}{R_1} + \frac{1}{R_2}$. 

In Eq.~\ref{eq:batchelor-pebble-conductance}, Batchelor and O'Brien had assumed the two contacting spheres to be of equal conductivity, $k_r$. Cheng, et al.\cite{Cheng19994199} proposed a slightly modified conductance which allows for contacting materials of different thermal conductivity. They have,

\begin{equation}\label{eq:cheng-modification-batchelor}
	H_c = 2k^*a = 2k^* \left(\frac{3}{4}\frac{R^*}{E^*}\right)^{1/3}F_n^{1/3}
\end{equation}

where $\frac{1}{k^*} = \frac{1}{k_i} + \frac{1}{k_j}$. As well as being a more general, flexible formulation, the models analyzed by Cheng, et al.\cite{Cheng19994199} are in good agreement with experiments. In the DEM numerical structure, we use the form given by Eq.~\ref{eq:cheng-modification-batchelor}.

Batchelor and O'Brien developed Eq.~\ref{eq:batchelor-pebble-conductance} with the assumption of two contacting particles in vacuum but, once developed, showed\cite{Batchelor1977} that this form is still valid when immersed in a fluid providing that the thermal conductivity ratio of solid and fluid is well above unity. The condition is expressed as,

\begin{equation}\label{eq:conductance-validity-fluid}
	\frac{ k_r }{ k_f } \frac{a}{R^*} \gg 1
\end{equation}

The term $\frac{a}{R^*}$, from \cref{sec:hertz-theory}, is necessarily much less than 1 for Hertz theory to be applicable. Thus for fluid in vacuum, the condition is identically satisfied but we must consider inaccuracies if we introduce an interstitial fluid with low conductivity ratios. 

For lithium ceramics in helium, the ratio is approximately $\frac{k_r}{k_f} \approx 10$ which is not necessarily large enough to satisfy the requirement of Eq.~\ref{eq:conductance-validity-fluid}.

As we step back from the contact of a single pair of particles and consider a particle in an ensemble with many contacts, we must again consider the validity of applying Eq.~\ref{eq:cheng-modification-batchelor} at each contact. [give a better description here of why this equation is valid]Vargas and McCarthy\cite{Vargas2002a}, proposed introducing a conduction Biot number to relate resistance to heat transfer internal to the particle with the resistance between particles. We use the following form

\begin{equation}
	\Bi_c = \frac{H_c}{k^* d_p} = 2\frac{a}{d_p}
\end{equation}

Then if $\Bi_c \ll 1$, the individual energy transferred between each point of contact can be decoupled. The Biot number criteria is already satisfied for Hertz theory to be valid; having assumed that $\frac{a}{d_p} \ll 1$. Therefore the total heat transferred out of a single particle with $Z$ contacts is summed from the individual contacts as 

\begin{equation}
	Q_i = \sum_j^Z Q_{ij}
\end{equation}

The form of contact conductance used in our study, built upon the solution of Batchelor and O'Brien\cite{Batchelor1977}, has been implemented by others in a variety of studies\cite{Vargas2001, Chaudhuri2006, Zhou2009,Cheng19994199}. However, in many other fields, the researchers are interested in such things as the parallel conduction through a stagnant interstitial gas\cite{Bu2013} or the temporary conduction during impact of fluidized beds\cite{Zhu2008,Zhang2011,Wu2011,Li2000}. In such cases, the formula for conductance can be quite different.

Nevertheless, for the packed beds of ceramic spheres we intend to model, the heat conductance of Eq.~\ref{eq:cheng-modification-batchelor} is an appropriate and valid form. When we incorporate the influence of an interstitial gas, it will be done in such a way as to leave the DEM heat transfer intact and only add an energy source term to stand in for the interaction with the fluid. The details will be discussed in \cref{sec:cfdem-heat-transfer}, but for now we conclude with a solid conduction theory that will be implemented in the discrete element method computations.