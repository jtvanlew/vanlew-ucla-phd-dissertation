\section{Inter-particle conduction}\label{sec:particle-conduction}
[Go back through Batchelor and O'Brien~\cite{Batchelor1977} paper]

\begin{equation}
	\frac{ k_s }{ k_f } \frac{a}{R^*} = \lambda
\end{equation}

Similar to the lumped capacitance assumptions, if $\lambda \gg 1$, the solid is approximately is isothermal. The second group on the left-hand side of this condition we remember from the assumptions of Hertz theory, where we require $\frac{a}{R^*} \ll 1^*$. Therefore to satisfy the condition of $\lambda \gg 1$, we require very large conductivity ratios of solid to fluid, $\frac{k_s}{k_f} \gg 1$. Alternatively this is satisfied by definition if the solids exist in vacuum.

Assuming that we satisfy the condition of isothermal solids, we address the conduction between solids in their small regions of contact.

[more details]










\subsection{Particle-particle conduction}

Handling the heat transfer between contacting particles has been investigated extensively by researchers in a number of fields\cite{Zhou2009,Zhang2011,Wu2011,Vargas2001,Li2000,Chaudhuri2006}. The amount of energy per time that can be transported per difference in temperature between pebble $i$ and $j$ as a conductance $h_{ij}$. Defined as

\begin{equation}\label{eq:pebble-conductance}
	\frac{h_{ij}}{k^*}= 2\left[\frac{3F_nR^*}{4E^*}\right]^{1/3}
\end{equation}

$k^*= 2k_ik_j/(k_i+k_j)$ is the effective solid conductivity of the two particles, and $F_n$ is the magnitude of the normal force between particles $i$ and $j$ as calculated by Eq.~\ref{eq:hertzForce}. Therefore, if we consider particles at temperatures $T_i$ and $T_j$ in contact, they will transfer heat at a rate of

\begin{equation}
	Q_{ij} = h_{ij}(T_i - T_j)
\end{equation} 

