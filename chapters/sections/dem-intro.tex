\section{Background}
\label{sec:dem-intro}



In the framework of the discrete element method, we track particle motion in a Lagrangian sense. In the ensemble of particles, each particle's position, velocity, and acceleration are tracked and updated based on balances (or imbalances) of forces acting upon the particle. The discrete element method for granular material 

\subsection{Numerical Implementation Overview}

The primary computational tools used in this study is LAMMPS (Large-scale Atomic/Molecular Massively Parallel Simulator)\cite{Plimpton1995}; a classical molecular dynamics code. The package of code, maintained by Sandia National Labs (http://lammps.sandia.gov), has many features making it particularly attractive for our use on the simulation of pebble beds. LAMMPS is open-source and written in highly-portable C++ allowing customization of any feature used in modeling. LAMMPS runs with distributed-memory message-passing parallelism (MPI) and provides simple control (manual or automatic) of the spatial-decomposition of simulation domain for parallelizing. The code can be built as a library so that LAMMPS can be coupled to other code or wrapped with Python as an umbrella script. Perhaps most importantly, LAMMPS provides an efficient method for detecting and calculating pair-wise interaction forces; the largest consumer of run-time in the DEM algorithm\cite{Plimpton1995}.

LAMMPS provides modeling of granular particle types; the use of LAMMPS for studying granular material since at least 2001 when Silbert, et al\cite{Silbert2001} studied granular flow on inclined planes. However, usefulness of LAMMPS for studying granular systems was greatly enhanced by LIGGGHTS (LAMMPS Improved for General Granular and Granular Heat Transfer Simulations), a suite of modules included on top of LAMMPS. LIGGGHTS has many academic and industrial contributors from around the world, with the code maintained as open-source by DCS Computing, GmbH.

Some notable features LIGGGHTS has added to LAMMPS include: Hertz/Hooke pair styles with shear history, mesh import for handling wall geometry, moving meshes, stress analysis of imported meshes, a macroscopic cohesion model, a heat transfer model, and improved dynamic load balancing of particles on processors\cite{Kloss2011}. Both LIGGGHTS and LAMMPS are distributed under the open-source codes under terms of the Gnu General Public License.

We will review some of the important physical modeling from LAMMPS/LIGGGHTS as they relate to features we wish to investigate for packed beds of pebbles in fusion reactors.