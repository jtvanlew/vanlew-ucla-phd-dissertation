\section{Conclusions}
\label{sec:dem-conclusions}
The results shown in Figs.~\ref{fig:contact-forces-scatter} and~\ref{fig:coord-scatter} demonstrate that the heat transfer through a pebble bed is simultaneously a function of both the coordination number and inter-particle contact forces. The average values of both of these parameters reduced as pebbles in the bed were crushed. Interestingly, when a pebble bed has lower overall inter-particle contact forces such as what we see when pebbles are crushed, we would predict fewer pebbles are likely to break. This result implies that pebble breakage is self-dampening; as pebbles begin to break the ensemble quickly relaxes and avoids future pebble failure. So while in this study we induced failure up to $\eta = 15\%$ without a concern for predicting if such a large amount would break, such large values may not occur in real beds during operation of a fusion reactor. 

The first study of this dissertation established the groundwork of the DEM modeling to be carried out in the other studies. We simulated a pebble bed with a specified fraction of the pebbles crushing during operation; then determining the repercussions of the missing pebbles as they affect the macroscopic property of effective thermal conductivity. We used the assumption of homogeneous, random locations of pebble failure to induce a failure routine without requiring external loads on the bed to actually induce the pebble crushing. After heating to a steady-state, an effective thermal conductivity was calculated for the pebble bed. The results show that large amounts of pebble failure correspond to large decreases in the conductive transport of energy through the pebble bed. The increase was due primarily to a drop in the inter-particle forces which lead to a large increase in temperature differences between neighboring pebbles. 

As the first step in the modeling effort, there were many simplifications that had to be made in this study. We must note here the shortcomings of the assumptions and simplifications of this study before drawing any major conclusions from the results.

First, the `container walls' surrounding the pebble bed in this model are completely rigid and do not react as the swelling pebble bed presses into them while heating. The confined thermal expansion leads to very high contact forces in the pebble bed that may not be realistic. The abnormally high contact forces are most likely to be the source of the abnormally high baseline effective thermal conductivity, $k_0 = 1.03$~W/m-K. In experiments on the effective thermal conductivity of lithium ceramics in vacuum, the beds are often allowed to expand freely while heating (in at least one direction) and in vacuum were measured to be closer to $k_0 = 0.5$ W/m-K [FIND THE REAL VALUES TO PUT HERE!]. We note, however, that this value has been calculated in the absence of interstitial gas so the results apply only to the reduction in energy transferred via inter-particle conduction.

Second, we saw from Fig.\ref{fig:temp-scatters} that the majority of the pebbles in the ensemble have their temperatures close fitting to an average curve but a number of the pebbles had less thermal contact with neighboring particles and consequently had much larger temperatures. This was true even in the baseline case of a tightly packed ($\phi = 64\%$) pebble bed. This phenomena is only possible because the contribution to heat transfer of the interstitial gas was not considered in this model. The flowing helium gas is expected to prevent any runaway temperatures of individual pebbles as it provides another route of energy transfer in the bed. This will be addressed in \cref{sec:cfd-dem-studies}.

Lastly, the pebble crushing did not conserve mass and did not have predictive whatever.