\section{Effective Thermal Conductivity from CFD-DEM}\label{sec:cfd-dem-effective-conductivity}

\subsection{Stagnant interstitial fluid}
can use correlations for stagnant gas in packed bed.



\subsection{Purge Gas}
The well-packed and resettled pebble beds were run to thermal steady-state with nuclear heating and wall cooling in both pure DEM and coupled CFD-DEM simulations for comparison. From steady-state temperature distributions, seen in the pebble scatter plots in Fig. 3, an average profile is calculated and an effective thermal conductivity computed. The values are tabulated in Table I. 
In the case of pure DEM, energy is transported solely along conduction routes in the ensemble. When the packing of the bed is disturbed, this results in a substantial drop in effective conductivity (a drop of 31\%). The details of the conductivity reduction were studied extensively in Ref. 23. Perhaps more important than the reduction in effective conductivity, is the appearance of isolated pebbles. Because heat deposition is volumetrically applied, pebbles with poor conduction routes become much hotter than their neighbors. This is evident in the high temperatures seen in many of the pebbles in the right figure of Fig. 3. Over-heating of isolated pebbles could induce sintering and impact their tritium release even when the average temperatures measured in the bed are well below sintering values.
When CFD-DEM beds are analyzed, there is still a large reduction in effective conductivity (22\% drop), but interesting to note is the lack of isolated pebbles with high temperatures. In the CFD-DEM scatter plot of the right image in Fig. 3, there is evidence of the reduced heat transfer in the same region as the isolated pebbles from the DEM bed, but the temperatures are much closer to the average values of neighboring pebbles. The helium purge gas has effectively smoothed out the temperatures and provided heat transport paths for any pebbles that have loose physical contact with neighbors.
In spite of the 22\% decrease in effective conductivity, the maximum temperature of the pebble bed only increased 6.2\% (from 725 to 751 K) when helium is included in the model. This result is significant for solid breeder designers. They may choose a solid breeder volume such that in the event of extensive pebble cracking, the maximum temperature of the bed would remain within the ideal windows dictate for the lithium ceramics.

\begin {table}[htp] %
\caption{Pebble bed values from the test matrix of the beds analyzed in this study.}
\label {tab:cfdem-keff} \centering %
\begin {tabular}{ rccccc }
\toprule %
			& 	\multicolumn{2}{c}{$k_\text{eff}$}	&   \multicolumn{2}{c}{$T_\text{max}$}	&	$\frac{Q_h}{Q_\text{nuc}}$		\\
			& 	\multicolumn{2}{c}{(W/mK)}			&	\multicolumn{2}{c}{(K)}				&									\\
			& 	DEM 		& 	CFD-DEM				&	DEM 		& 	CFD-DEM 			& 	CFD-DEM							\\\toprule
Well-packed	& 	0.96		& 	1.09				& 	745			& 	725					& 	1.15							\\
Resettled	& 	0.66		& 	0.85				& 	800			& 	751					& 	1.52							\\\bottomrule
\end{tabular}
\end{table}




An accompanying result is the increased amount of energy carried out of the system by the helium purge gas. In Table I, the last column provides the ratio of energy carried out of the system to the nuclear energy deposited into the bed. The amount of energy carried out by the helium increased from 1.15\% to 1.52\% from ‘well-packed’ to ‘resettled’.
evap-x-T-color