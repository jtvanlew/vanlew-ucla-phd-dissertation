\section{Granular heat transfer}

Much like our treatment of the momentum of every particle in DEM, we handle the energy in the Lagrangian specification. The temperature of particle $i$, for example, is found from the first law of thermodynamics

\begin{equation}\label{eq:thermoFirstLaw}
	\rho_iV_iC_i\frac{\mathrm{d}T_i}{\mathrm{d}t} = Q_{s,i} + Q_{i}
\end{equation}

where $\rho$, $V$, and $C$ are the density, volume, and the specific heat of the solid, respectively. 

On the right hand side of Eq.~\ref{eq:thermoFirstLaw}, is the nuclear heating source term, $Q_{s}$, and the total heat transferred to/from particle $i$ via conduction to neighboring particles,

\begin{equation}
	Q_i = \sum_{i}^Z Q_{ij}
\end{equation}

The conductive heat transfer to neighboring particles was discussed in length in \S~\ref{sec:ht-pebble-conduction}. In that section we derived Eq.~\ref{eq:pebble-conductance} as a conductance term between neighboring particles. It is repeated here for reference:

\begin{equation*}
	\frac{h_{ij}}{k^*}= 2\left[\frac{3F_nR^*}{4E^*}\right]^{1/3}
\end{equation*}

where the $^*$ terms are the effective properties of the two particles under consideration. Therefore, if we consider particles at temperatures $T_i$ and $T_j$ in contact, they will transfer heat at a rate of

\begin{equation}
	Q_{ij} = h_{ij}(T_i - T_j)
\end{equation} 

We are assuming that in a single timestep a pebble is transfering heat only with its immediate neighbors. Vargas and McCarthy\cite{Vargas2001} provide arguments for the validity of Eq.~\ref{thermoFirstLaw} given numeric time steps and contact areas. The conclusion is that any time step that satisfies stability of the particle dynamics will automatically satisfy particle heat transfer.%thus we need not continuously check the validity of those assumptions when we have different input parameters to the heat transfer of our simulation.



\subsection{Thermal expansion}
Owing to the impact thermal expansion has on pebble bed structures undergoing thermal cycling\cite{Tanigawa:2010cr, Vargas2007, Chen2009}, we also included a simple thermal expansion model.  The diameter of the pebbles was updated after a specific number of timesteps according to the following

\begin{equation}
	d_i = d_{0,i}\left[1+\alpha\left(T_i - T_\text{ref}\right)\right]
\end{equation}

where $\alpha$ is the thermal expansion coefficient, $T_i$ is the temperature of the pebble at the current step, and $d_{0,i}$ is the diameter of the pebble at temperature $T_\text{ref}$.






\subsection{Pebble Bed Heat Transfer: Test Case}
In our pebble bed test case, we establish heat transfer that is essentially one-dimensional in the $x$-direction. The pebble bed has very little variation of forces and temperatures in the $y$-direction due to the periodic boundary condition at the edges of the domain. Gravity effects are minor in the overall heat transfer and induce only a slight $z$-dependency  to the results. With the one-dimensional assumption, we step back into a continuum mechanics formulation to aid us in finding an effective thermal conductivity of our numeric pebble bed. 

A steady state for a material with constant temperature boundary conditions ($T(\pm 10d) = T_s$) and nuclear heating has the following heat equation

\begin{equation}\label{eq:continuum-heateqn}
	0 = \frac{\mathrm{d}^2T}{\mathrm{d}x^2} + \frac{q'''}{k_\text{eff}}
\end{equation}

In this continuum mechanics formulation, we are assuming that the nuclear source, $q'''$ term is applied evenly over the entire volume. In our DEM formulation, our source term applies to a single pebble. To find the effective thermal conductivity of our pebble bed, we must reconcile this discrepency. This is accomplished  with the exchange of

\begin{equation}
	q''' = \frac{Q_\text{tot}}{V_\text{tot}} = \frac{Q_sN}{300Hd^2}
\end{equation}

where $H$ is the average height of the top layer of pebbles. We apply symmetry about the centerline and impose our boundary conditions to solve the differential equation. If we take the temperature of the midplane as $T(0) = T_0$, we back-out an effective thermal conductivity (ETC) as

\begin{equation}\label{eq:etc}
	k_\text{eff} = \frac{Q_sN}{6H(T_0-T_s)}
\end{equation}


We will use this formulation to analyze and compare our test-case pebble beds.



