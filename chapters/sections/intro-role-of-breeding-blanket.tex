\section{The Role of Breeding Blankets}
Absorb energy and convert to high quality heat for power production.

* where approximately 80\% of the fusion energy is in the kinetic energy of the ejected neutron. The blanket surrounding the plasma must convert

Breed tritium

A commonly used classification of the efficacy of a breeding blanket is via the tritium breeding ratio (TBR), defined as the number of tritium atoms produced in the blanket per fusion neutron. In reality, only 60-80\% of the fusion neutrons react with the lithium due to neutron leakage and parasitic interactions. Futhermore, when we take into account tritium retention in structural material or losses due to inefficiency in collecting tritium, then self-sufficiency of the fuel cycle is clearly not possible unless we produce more than one tritium per fusion neutron. 

For solid breeders, beryllium is introduced into the blanket as a neutron multiplier. The incident neutron breaks Be up into two $\alpha$ particles and an additional two neutrons. Thus it is possible, with careful neutronics analysis and engineering of tritium breeding volumes and neutron multiplying regions, to attain a TBR which makes the power plant not only self-sufficient in terms of fuel, but also able to seed tritium for a future power plant. Assuming that the fuel cycle of tritium is handled properly (perhaps the biggest assumption we will make in this work), the last remaining function of the blanket is to supply energy for the electricity generation of the power plant. 

As each individual tritium breeding region is small, in a typical solid breeder blanket design there are several alternating layers of breeding zone, cooling plate, and neutron multiplier. 

The two most prominently analyzed neutron multipliers for a fusion reactor are beryllium and lead. Beryllium has a very high nuclide density while also being very light, with a high melting temperature, and high thermal conductivity. For solid breeding blankets, beryllium has been pegged as the element of choice. The beryllium, based on its own design requirements, is generally also chosen to exist in a pebble bed form in the breeding blanket device.





Protect magnets and other components