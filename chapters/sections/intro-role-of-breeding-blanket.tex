\section{The Role of Breeding Blankets}
As was shown in \cref{sec:fusion-basics}, the blanket surrounding the plasma must perform the critical role of breeding tritium from lithium. The measure of a blanket design's effectiveness at breeding tritium is commonly done with the metric of the tritium breeding ratio (TBR). The TBR is defined as the number of tritium atoms produced in the blanket per fusion neutron. When the fusion neutron is ejected it may stream through gaps in the blanket that are necessary for instrementation, plasma heating, etc. The neutron may also collide with supporting structure or other elements in the breeding material. Due to the neutron leakage and parasitic interactions, only 60-80\% of the fusion neutrons may react with lithium. Futhermore, once tritium is actually generated, it is retained in structural material or lost to the fuel cycle simply from inefficiencies in handling. It becomes quickly obvious self-sufficiency of the fuel cycle is clearly not possible unless we produce more than one tritium per fusion neutron [maybe not accurate -- check Abdou's notes]. 

One scheme for increasing the TBR of breeder blankets is to introduce beryllium as a neutron multiplier. Beryllium has a very high nuclide density while also being very light, with a high melting temperature, and has a high thermal conductivity. The incident neutron smashes beryllium into two $\alpha$ particles and an additional two neutrons. Therefore, in order to breed tritium, the blanket will first generate more neutrons for every neutron spit out by the fusion reaction.

In addition to breeding tritium, the blanket will be responsible for power extraction in the fusion reactor. The blanket must function to capture the kinetic energy of neutrons (80\% of the fusion energy is carried by the neutron), secondary $\gamma$ rays, and the plasma radiation on the plasma-facing first wall (an integral component to the blanket). The blanket must also be able to convert the fusion's energy into high quality heat that can be extracted into the power cycle connected to the fusion reactor. 

Finally, the last function of the breeding blanket is to provide radiation shielding of the vacuum vessel and super-conducting magnets that are containing the plasma. 

[From Alice:]there are two lines of blanket concepts using different forms of breeder: liquid or solid. In this thesis, the research is on the solid breeder blankets, specifically the blanket uses lithium ceramic breeder pebbles, as the form of solid breeder, for tritium fuel production. 