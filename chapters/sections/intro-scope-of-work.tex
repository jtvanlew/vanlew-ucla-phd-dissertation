\section{Scope of the work}\label{sec:intro-scope-of-work}
The objective of this dissertation is to develop numerical models of ceramic pebble beds, based on first principles and experimental observations, to simulate the hysteritic evolution of pebble bed morphology and predict the subsequent changes to heat transport characteristics after thermally-induced damage to pebbles. The numerical tools are constructed in the following progression: 1. Transient DEM code of inter-particle interactions is employed to simulate packed bed restructuring in the wake of crushed pebbles from the ensemble and the effective thermal conductivity following the restructurin, 2. Transient, volume-averaged equations of Navier-Stokes and energy of the helium purge gas are coupled to the DEM model of pebbles to simulate conjugate heat transfer and the interstitial fluid influence on thermo-physical properties after crushing events, 3) Complete simulations of the tortuous path of helium purge gas with lattice-Boltzmann models to expose flattened temperature profiles due to laminar mixing in the pebble bed, 4) Single pebble experiments to both deduce proper material properties to employ in the DEM framework as well as develop crush-prediction models, 5) To apply the modeling tools to a thermal evaluation of coolant designs of pebble beds in solid breeder blankets in ITER. 

A thorough understanding of the evolution of pebble bed morphology and the impact on thermo-physical properties is critical for solid breeder designers. The understanding allows for temperature control of breeder pebble beds over the entire lifetime of the blanket which is crucial to the function of the solid breeder for tritium and energy generation. Thus we aim to provide designers of packed beds with tools to understand how packing states may evolve from time-dependent phenomena (e.g. sintering, creep, pebble cracking, etc.). These phenomena may, for instance: decrease the effective thermal conductivity which will raise bed temperatures beyond initial predictions, produce isolated pebbles which will sinter and potentially decrease tritium release rates, or even form gaps between pebble beds and containing structures leading to divergence from properties of the initial packing of the bed.

The objective of this work fits into the broader mission of our research group in the UCLA Fusion Science and Technology Center to develop and apply complete numerical models of ceramic pebble bed solid breeder modules. Any complete numerical model for a pebble bed would require the interaction of many sub-models or sub-functions operating at disparate scales. To demonstrate, a possible top-level algorithm could proceed in the following way: To begin, one must have knowledge of the interaction of the pebble bed with the containing structure as they exist in a fusion environment. The interactions are generally analyzed via the finite element method to find internal stresses and temperature fields of the entirety of the pebble bed and surrounding container. After the internal fields are mapped into the bed, one would use the discrete element method (DEM) to interpret the macroscopic stress fields into the inter-particle forces. With the inter-particle forces and total absorbed thermal energy calculated, a prediction of the initiation and evolution of morphological changes (i.e. crushed pebbles, sintering, creep, etc.) to each computational volume. Following this, DEM would calculate new effective properties as a result of the morphological changes to the pebble bed region. Finally, the updated bed properties would feed back into the FEM formulation to update calculations in the macroscopic stress fields. While a suite of integrated numerical tools that follows this example algorithm is the ultimate goal of our group, the work of this dissertation is focused entirely on the development of pebble-scale simulations that are predominately in the realm of the discrete element method.

In the following subs-sections, we briefly outline the studies fitting into the scope of this dissertation. 


\subsection*{Discrete element method study on the evolution of thermomechanics of a pebble bed experiencing pebble damage}
In the first study of \cref{sec:dem-studies}, we analyze the effective thermal conductivity of a pebble bed assuming different fractions of pebbles in the ensemble are completely crushed. The focus of this study is to determine the extent of change in aggregate ensemble properties due to individual pebble crushing, relate the changes in effective conductivity to quantifiable pebble-scale properties (e.g. contact force, coordination number, etc.), as well as help designers anticipate acceptable limits of pebble loss from a thermal management point of view. For the DEM tools used in this study, the only mode of heat transfer is conduction between the solid particles. 


\subsection*{Coupling DEM Models of Ceramic Breeder Pebble Beds to Thermofluid Models of Helium Purge Gas Using Volume-averaged CFD}
In a fusion breeder, the helium purge gas winding through the interstitial gaps of the pebbles has a substantial contribution to overall heat transfer.\cite{Reimann:2002mi,Abou-Sena2005} The model of \cref{sec:dem-studies} is improved to include the flowing interstitial gas. In \cref{sec:cfd-dem-studies}, we continue to employ our DEM tools to provide particle-scale information such as contact force, but couple the pebbles to a volume-averaged computational fluid dynamics (CFD) code. The coupled CFD-DEM model is again used to simulate the heat transfer in packed beds of ceramic spheres that experience pebble crushing -- but now investigate the impact of a flowing interstitial helium purge gas when pebbles are crushed.


\subsection*{Lattice-Boltzmann method integrating DEM packing structures to study laminar mixing}
The models to account for helium purge gas emplpoyed in the studies of \cref{sec:cfd-dem-studies,sec:applied-studies} assume effective drag or heat transfer coefficients for pebbles in a computational volume and then include the pebble influence through effective source/sink terms in the momentum and energy equations. The volume-averaged approach allows for simpler meshing of the fluid volume while still retaining much of the physical realism of the system. Complete models of the conjugate heat transfer of both the fluid moving through the tortuous interstitial gaps pebble beds pressing each other with small contact areas are intractable with current computational hardware and finite-element modeling techniques. To overcome deficiencies in computational power, in \cref{sec:modeling-lbm}, we apply the lattice-Boltzmann algorithm to our pebble bed with helium. The lattice-Boltzmann method (LBM) is a non-traditional fluid simulation technique that allows us to resolve pebble-scale interactions with bed-scale conjugate heat transfer with flowing gas on realistic simulation time scales. The LBM approach is applied to the same pebble beds analyzed in \cref{sec:cfd-dem-studies} to provide comparison between the two modeling techniques. Furthermore the LBM model, accounting for the complex helium purge gas pathways, provides more insight to the influence of helium on the heat transfer in the heat transfer of packed beds.


\subsection*{Modeling tools to study coolant designs of ITER solid breeder module volumes}
In the study of \cref{sec:applied-studies}, we apply our coupled CFD-DEM computational tools to the analysis of ITER-relevant solid breeder geometries. In this study we consider the combined effects of pebble crushing, packing restructuring due to both gravity and the unbalanced force network in the pebble bed, and convection from helium purge gas on temperature profiles in solid breeders for different breeding configurations. In typical solid breeder modules, coolant fluid runs through the containing structure surrounding the pebble bed. Heat is removed from the pebble bed predominately through inter-particle conduction and contact conductance of many pebbles pressed against the containing surface. As such, heat transfer out of the pebble bed relies on maintaining good pebble-pebble and pebble-wall contact. However, physical contact is interrupted to different degrees when a pebble bed responds to various amounts of individual crushed pebbles. Furthermore, the restructuring of the pebble bed after a pebble crushing event is, in part, dependent on gravity forces acting upon each pebble in the ensemble. We investigate two representative pebble bed configurations where heat is removed from the bed via inter-particle conduction, convection of purge gas, and contact between the pebble bed and its container. In the first, the coolant containing structural walls (heat transfer walls) are oriented parallel to the gravity vector. In the second configuration, the heat transfer walls are perpendicular to the direction of gravity. To simulate a crushed pebble, we replace the pebble with many smaller, non-cohesive elements while maintaining mass-conservation between the original solid pebble and crushed fragments. The fragments are then free to resettle into interstitial gaps and the rest of the bed resettles as determined by forces from gravity, contact of neighboring particles, and even the small influence of the moving purge gas. The thermo-fluid interaction with the helium purge gas will be included with volume-averaged Navier-Stokes and energy equations. The representative solid breeder volumes will be compared with respect to their temperature peaks and profiles and how those temperatures vary as a function of the percentage of crushed pebbles in the ensemble. The results can be used to optimize solid breeder pebble bed designs through the choice of breeding zone orientation relative to the gravity vector.

\subsection*{Experiments on individual ceramic pebbles}
In \cref{sec:exp-reduction-factor}, Experiments to crush individual ceramic pebbles have lead to a new predictions on when a pebble will be damaged in a loaded ensemble. \cref{sec:theoryStrainEnergy} -- experiments on pebble crushing, using strain energy to convert lab data to ensemble forces.

\subsection*{Implementation of experimentally-based enhancements to standard DEM and CFD-DEM modeling tools}
The discrete element method, as currently employed by members of the fusion community, begins with the assumption that each pebble is a perfectly elastic material that obeys Hertz’s theory for normal interaction. This assumption impacts the magnitude of inter-particle forces predicted by the models. We scrutinize the Hertzian assumption with single-pebble crush experiments with carefully recorded force-displacement responses and compare them to the non-linear forces predicted by a Hertzian pebble with bulk properties reported in literature. We found each pebble generally has a non-linear force response but with varying levels of stiffness that qualitatively matched the curves from Hertz theory. Assuming Hertzian interaction, we then backed-out an elastic modulus for each pebble. We define a stiffness reduction factor, $k$, as the ratio of the pebble's elastic modulus to the sintered bulk value from literature. After determining the $k$ value for every pebble in our batch, we discovered a probability distribution for different batches. The distribution is attributed to the varying micro-structure of each pebble. We incorporate the results into our DEM algorithms, distributing $k$ values at random to pebbles satisfying the probability curves of experiments. DEM simulations of pebble beds in oedometric compression are carried out to determine macroscopic responses of stress-strain, contact force distributions at maximum stress, and a prediction of pebbles crushing at that point. In all cases studied in \cref{sec:dem-studies-youngs-modulus}, the pebble beds with modified Young's modulus had smaller overall contact forces and fewer predicted crushed pebbles. 

We employ the theory based on experimental results of individual pebble crushing to create modules to attach to DEM tools for predicting pebble crushing in ensembles in \cref{sec:failure-discussion}. On a small ensemble of pebbles, we simulate a uniaxial compression test with the pebble crushing module activated in the DEM code. Our group has also experimented on pebble beds to measure the percent of damaged pebbles under certain loads and load cycles. The results of our uniaxial compression test in DEM can be scaled up to compare to the experimental results on pebble beds.

In \cref{sec:ht-jeffreson-correction}, we introduce a correction to the heat transfer coefficient used in the CFD-DEM framework to account for low conductivity and high heat generation inside of particles. In the discrete element method (DEM), an innate assumption in the computational framework is of isothermal DEM particles. When DEM models are coupled to volume-averaged Navier-Stokes models of purge gas flow, the accuracy of the lumped capacitance method is quantified with the Biot number. For moderately sized Biot numbers ($\Bi$ > 0.5), the lumped capacitance method is inaccurate for both steady-state and transient temperatures for a particle with heat generation. We introduce a correction to the heat transfer coefficient which allows the lumped capacitance assumption of the DEM to be accurate in both transient and steady-state regions. The correction is first compared with an analytic solution of a sphere in a heat transfer flow, then the correction is implemented in codes of coupled computational fluid dynamics with the discrete element method (CFD-DEM). We begin with a test case of a single pebble with heat generation being cooled by a passing gas. The result is compared to both the idealized analytic solution as well as a complete model of the conjugate heat transfer as calculated by the lattice-Boltzmann method (LBM). Once shown to be effective for a single sphere, the correction is use in a study of a sample packed bed and again compared to the more exact result of an LBM computation. The results show the simple correction to be effective at capturing the thermal physics of moderate-to-large Biot numbers while still using the simplified equations of the lumped capacitance method.

In \cref{sec:dem-stability} we show that proper modification of material properties, we can dramatically increase the timestep and thus decrease the total time of simulation. In our thermal DEM simulations, the time to reach steady state can often take hundreds of real-time seconds owing to the poor thermal transport in the ceramic pebble bed. 