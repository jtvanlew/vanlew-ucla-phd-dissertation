\section{Scope of the work}\label{sec:intro-scope-of-work}


To develop a complete numerical model for a pebble bed requires completing many interactive sub-models. To demonstrate, we give here the path of a possible analysis scheme of these models. To begin, one must have knowledge of the interaction of the pebble bed with the containing structure as they exist in a fusion environment. The interactions are generally analyzed via the finite element method to find internal stresses and temperature fields of the entirety of the pebble bed. After the internal fields are mapped, one would use the discrete element method (DEM) to interpret the macro-scopic stress fields into the inter-particle forces. With the inter-particle forces and total absorbed thermal energy calculated, a prediction of the initiation of pebble failure would predict the number of pebbles (if any) that would be crushed in that computational volume. When a pebble is crushed, it loses contact with its neighbors and subsequently breaks any thermal or mechanical transport that the pebble was providing. Fragmentation of a failed pebble would also be handled by the DEM with another model. Following this, DEM would determine how the pebble bed re-settles and effective properties evolve in the presence of failed pebbles. 
Finally, the updated bed properties would feed back into the FEM formulation to predict how overall stress fields and material interactions are altered in light of the failure. The fusion community is far from an integrated simulation that can follow such a path, but it is the principle goal of the overall efforts at UCLA.

Research on pebble failure up to now has focused on predicting when pebbles may fail in a bed as a function of an external load (typically, stress from walls). In this study, we analyze the evolution of pebble bed properties assuming some fraction of pebbles in the ensemble have failed. The focus of this study is to determine the extent of change in aggregate ensemble properties due to individual pebble failure, as well as help designers anticipate acceptable limits of pebble failure from a thermal management point of view. We make use of DEM to simulate individual pebbles in a packed bed. From this scale of simulation, we can study single pebbles undergoing failure while the bed as a whole is subject to mechanical and thermal boundaries.

For the DEM tools used in this study, the only mode of heat transfer is conduction through the solids. In a fusion breeder however, the helium purge gas winding through the interstitial gaps of the pebbles will have a large contribution to overall heat transfer\cite{Reimann:2002mi,Abou-Sena2005}. To overcome the current limit on DEM heat transfer, we are also working with computational fluid dynamics coupling to the discrete element method to account for the helium energy transport. The next step is to combine our analysis tools with a failure initiation predictor as well as a new method of simulating a pebble after failure. Those modeling enhancements will be reported in the future. As these models become more comprehensive in their scope, the fusion community will be better prepared to determine the survivability and performance of a solid breeder design in the fusion environment.


% From SOFT 2014
The discrete element method (DEM) is used by many ceramic breeder researchers to model the interaction of individual pebbles in an ensemble in an effort to obtain a more detailed understanding of pebble beds than is possible with experimental measurements of effective properties. For example see Refs.~\cite{An20071393, Lu2000, Zhao2010, Gan2010a, Annabattula2012a, VanLew2014}. A major assumption in the DEM formulation is that each pebble acts perfectly elastically and adheres to Hertz theory for contacting spheres. With Hertz theory, one finds contact forces as a simple function of: the virtual overlap between two objects, the Young's modulus of the contacting material (and Poisson ratio), and radii of the two. In past studies, the Young's modulus of the ceramic materials  used in DEM simulations was taken from historical data, for instance lithium metatitanate from Ref.~\cite{Gierszewski1998}.

Based on observations of experimental data from single pebble crush data, in this study we propose a new method of obtaining the Young's modulus for a batch of ceramic pebbles as the historical values from literature are not always appropriate.


% From TOFE 2014
We aim to provide designers of packed beds with tools to understand how packing states may evolve from time-dependent phenomena (e.g. sintering, creep, pebble cracking, etc.). These phenomena may, for instance: decrease the effective thermal conductivity which will raise bed temperatures beyond initial predictions, produce isolated pebbles which will sinter and potentially decrease tritium release rates, or even the form gaps between pebble beds and containing structures leading to divergence from initial packing properties. 

Modeling research on ceramic pebble beds should have as its main objective a thorough understanding of the evolution of pebble bed morphology and the impact on thermophysical properties; allowing for temperature control of breeder pebble beds over the entire lifetime of the blanket. To accomplish that goal, this current study is aimed at developing a methodology for coupling established discrete element models of individual pebbles in the ensemble with thermo-fluid simulations of the interstitial helium purge gas. Specifically, we will address the impact of helium on the thermal transport in a bed experiencing evolving morphology due to cracked pebbles.

Global models of pebble beds and helium flow with pebble-scale detail are intractable with current computational hardware and modeling techniques. To overcome deficiencies in computational power, we introduce two new modeling approaches that allow us to resolve pebble-scale interactions with bed-scale conjugate heat transfer with flowing gas.