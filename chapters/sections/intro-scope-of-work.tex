\section{Scope of the work}\label{sec:intro-scope-of-work}
[Here we need an objective (or objectives) first. Perhaps you mention the objective in the middle of the section. It should be clearly written in the beginning of the section-- The objective of this research is to create modeling tools to be able to simulate the pebble bed morphology evolution, and to address consequent heat transfer and temperature, and to provide ----. 

scope of work- 1. to develop a transient DEM code for simulating pebbled bed morphology evolution, 2. to develop DEM coupled CFD code for heat transfer and temperature analysis, 3. to assess thermal mixing using LBM, 4. to apply the developed modeling tools for blanket pebble bed design thermal evaluations, 5. further implication---.]

To develop a complete numerical model for a pebble bed requires completing many interacting sub-models. To demonstrate, we give here the path of a possible analysis scheme of these models. To begin, one must have knowledge of the interaction of the pebble bed with the containing structure as they exist in a fusion environment. The interactions are generally analyzed via the finite element method to find internal stresses and temperature fields of the entirety of the pebble bed and surrounding container. After the internal fields are mapped into the bed, one would use the discrete element method (DEM) to interpret the macro-scopic stress fields into the inter-particle forces. With the inter-particle forces and total absorbed thermal energy calculated, a prediction of the initiation and evolution of morphological changes (i.e. crushed pebbles, sintering, creep, etc.) to each computational volume. Following this, DEM would calculate new effective properties as a result of the morphological changes to the pebble bed region. Finally, the updated bed properties would feed back into the FEM formulation to update calculations in the macroscopic stress fields. A suite of integrated numerical tools that follows this example algorithm is the goal of the Fusion Science adn Technology Center at UCLA, but we are far from that at this time. The work of this dissertation is focused entirely on the development of pebble-scale simulations that are predominately in the realm of the discrete element method.

We aim to provide designers of packed beds with tools to understand how packing states may evolve from time-dependent phenomena (e.g. sintering, creep, pebble cracking, etc.). These phenomena may, for instance: decrease the effective thermal conductivity which will raise bed temperatures beyond initial predictions, produce isolated pebbles which will sinter and potentially decrease tritium release rates, or even form gaps between pebble beds and containing structures leading to divergence from properties of the initial packing of the bed.

Modeling research on ceramic pebble beds should have as its main objective a thorough understanding of the evolution of pebble bed morphology and the impact on thermophysical properties; allowing for temperature control of breeder pebble beds over the entire lifetime of the blanket. To accomplish that goal, this current study is aimed at developing a methodology for coupling established discrete element models of individual pebbles in the ensemble with thermo-fluid simulations of the interstitial helium purge gas. Specifically, we will address the impact of helium on the thermal transport in a bed experiencing evolving morphology due to cracked pebbles.

Specifically, much of the work in this dissertation concerns the evolution of thermophysical properties of a pebble bed in the presence of crushed individual pebbles. When a pebble in an ensemble is crushed or cracked it loses contact with its neighboring pebbles and subsequently breaks any thermal or mechanical transport that the pebble was providing -- ultimately this is manifest in measurable changes to thermophysical properties. We attempt to quantify the evolving thermophysical changes with increasingly sophisticated DEM models of ceramic pebble beds.


%DEM
\subsection{DEM}
In the first study of \cref{sec:dem-studies}, we analyze the effective thermal conductivity of a pebble bed assuming different fractions of pebbles in the ensemble are completely crushed. The focus of this study is to determine the extent of change in aggregate ensemble properties due to individual pebble crushing, relate the changes in effective conductivity to quantifiable pebble-scale properties (e.g. contact force, coordination number, etc.), as well as help designers anticipate acceptable limits of pebble loss from a thermal management point of view. For the DEM tools used in this study, the only mode of heat transfer is conduction between the solid particles. 



%CFD-DEM
\subsection{CFD-DEM}
In a fusion breeder, the helium purge gas winding through the interstitial gaps of the pebbles has a substantial contribution to overall heat transfer.\cite{Reimann:2002mi,Abou-Sena2005} The model of \cref{sec:dem-studies} is improved to include the flowing interstitial gas. In \cref{sec:cfd-dem-studies}, we continue to employ our DEM tools to provide particle-scale information such as contact force, but couple the pebbles to a volume-averaged computational fluid dynamics (CFD) code. The coupled CFD-DEM model is again used to simulate the heat transfer in packed beds of ceramic spheres that experience pebble crushing -- but now investigate the impact of a flowing interstitial helium purge gas when pebbles are crushed.



%APPLICATION -- ORIENTATION OF CONTAINER
\subsection{Applied CFD-DEM}
In the study of \cref{sec:applied-studies}, we apply our coupled CFD-DEM computational tools to the analysis of ITER-relevant solid breeder geometries. In this study we consider the combined effects of pebble crushing, packing restructuring due to both gravity and the unbalanced force network in the pebble bed, and convection from helium purge gas on temperature profiles in solid breeders for different breeding configurations. In typical solid breeder modules, coolant fluid runs through the containing structure surrounding the pebble bed. Heat is removed from the pebble bed predominately through inter-particle conduction and contact conductance of many pebbles pressed against the containing surface. As such, heat transfer out of the pebble bed relies on maintaining good pebble-pebble and pebble-wall contact. However, physical contact is interrupted to different degrees when a pebble bed responds to various amounts of individual crushed pebbles. Furthermore, the restructuring of the pebble bed after a pebble crushing event is, in part, dependent on gravity forces acting upon each pebble in the ensemble. We investigate two representative pebble bed configurations where heat is removed from the bed via inter-particle conduction, convection of purge gas, and contact between the pebble bed and its container. In the first, the coolant containing structural walls (heat transfer walls) are oriented parallel to the gravity vector. In the second configuration, the heat transfer walls are perpendicular to the direction of gravity. To simulate a crushed pebble, we replace the pebble with many smaller, non-cohesive elements while maintaining mass-conservation between the original solid pebble and crushed fragments. The fragments are then free to resettle into interstitial gaps and the rest of the bed resettles as determined by forces from gravity, contact of neighboring particles, and even the small influence of the moving purge gas. The thermofluid interaction with the helium purge gas will be included with volume-averaged Navier-Stokes and energy equations. The representative solid breeder volumes will be compared with respect to their temperature peaks and profiles and how those temperatures vary as a function of the percentage of crushed pebbles in the ensemble. The results can be used to optimize solid breeder pebble bed designs through the choice of breeding zone orientation relative to the gravity vector.


%LBM
\subsection{LBM}
The models used to account for helium purge gas in the studies of \cref{sec:cfd-dem-studies,sec:applied-studies} assume effective drag or heat transfer coefficients for pebbles in a computational volume and then include the pebble influence through effective source/sink terms in the momentum and energy equations. The volume-averaged approach allows for simpler meshing of the fluid volume while still retaining much of the physical realism of the system. Complete models of the conjugate heat transfer of both the fluid moving through the tortuous interstitial gaps pebble beds pressing each other with small contact areas are intractable with current computational hardware and finite-element modeling techniques. To overcome deficiencies in computational power, in \cref{sec:modeling-lbm}, we apply the lattice-Boltzmann algorithm to our pebble bed with helium. The lattice-Boltzmann method (LBM) is a non-traditional fluid simulation technique that allows us to resolve pebble-scale interactions with bed-scale conjugate heat transfer with flowing gas on realistic simulation time scales. The LBM approach is applied to the same pebble beds analyzed in \cref{sec:cfd-dem-studies} to provide comparison between the two modeling techniques. Furthmore the LBM model, accounting for the complex helium purge gas pathways, provides more insight to the influence of helium on the heat transfer in the heat transfer of packed beds.


%Background studies
\subsection{Other kinds of studies and shit}
\cref{sec:studies-experiments}
\cref{sec:studies-numerics}








