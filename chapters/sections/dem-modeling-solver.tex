\subsection{Numerical Implementation of DEM}\label{sec:dem-solver}


The primary computational tools used in this study is LAMMPS (Large-scale Atomic/Molecular Massively Parallel Simulator)\cite{Plimpton1995}; a classical molecular dynamics code. The package of code, maintained by Sandia National Labs (http://lammps.sandia.gov), has many features making it particularly attractive for our use on the simulation of pebble beds. LAMMPS is open-source and written in highly-portable C++ allowing customization of any feature used in modeling. LAMMPS runs with distributed-memory message-passing parallelism (MPI) and provides simple control (manual or automatic) of the spatial-decomposition of simulation domains for parallelizing. Perhaps most importantly, LAMMPS provides an efficient method for detecting and calculating pair-wise interaction forces; the largest consumer of run-time in the DEM algorithm\cite{Plimpton1995}. We build the code as a library so that LAMMPS can be coupled to other numerical tools; we use the scripting language of Python (Python 2.7) as an umbrella code to call LAMMPS routines with the full availability of Python libraries. 

LAMMPS by default provides a rudimentary method of modeling of granular particles (the term `granular' here differentiates the `discrete element' of molecules or atoms from larger-scale granular particles of powders or pebbles); LAMMPS has been used for studying granular material since at least 2001 when Silbert\etal\cite{Silbert2001} studied granular flow on inclined planes. However, the usefulness of LAMMPS for studying granular systems was greatly enhanced by LIGGGHTS (LAMMPS Improved for General Granular and Granular Heat Transfer Simulations), a suite of modules included on top of LAMMPS. LIGGGHTS has many academic and industrial contributors from around the world, with the code maintained as open-source by DCS Computing, GmbH.

Briefly, some notable features the LIGGGHTS code brings to the LAMMPS environment include: Hertz/Hooke pair styles with shear history, mesh import for handling wall geometry, moving meshes, stress analysis of imported meshes, a macroscopic cohesion model, a heat transfer model, and improved dynamic load balancing of particles on processors\cite{Kloss2011,Kloss2012}. Both LIGGGHTS and LAMMPS are distributed under the open-source codes under terms of the Gnu General Public License.

We will review some of the important physical models used in LAMMPS/LIGGGHTS as they relate to the important features we wish to investigate for packed beds of pebbles in fusion reactors.


