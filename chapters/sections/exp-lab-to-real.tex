\section{Linking interactions with strain energy}\label{theoryStrainEnergy}
Hertz theory is applicable to any two contacting elastic objects. In practice, we cannot probe the contacts of small particles and rely on experiments where we press pebbles between flat platens. Here we will develop a theory for connecting the results of the experiments with the interaction of two spherical objects.

To relate the situation in the lab to two particles, we first integrate the Hertzian force along the overlap to find the strain energy, $W_\epsilon$, of that contact. 

\begin{equation}
	W_\epsilon = \int_0^{\delta_c}\!F_n(\delta')\,\mathrm{d}\delta'
\end{equation}

where the upper limit of the integration is the critical overlap $\delta_c$ (the meaning of this value will be explained in detail later). With the force defined from Eq.~\ref{eq:hertzForce}, this is straightforward to integrate.

\begin{align}
	W_\epsilon& = \int_0^{\delta_c}\!  \frac{4}{3}E^*\sqrt{R^*}\,\delta'^{3/2} \,\mathrm{d}\delta' \\
	%W_\epsilon & = \frac{4}{3}E^*\sqrt{R^*} \left[\frac{2}{5}\,{\delta_c}^{5/2}\right] \\
	W_\epsilon & = \frac{8}{15}E^*\sqrt{R^*}\, {\delta_c}^{5/2}
\end{align}

We will call the strain energy of the pebble compressed between platens as the lab strain energy, $W_{\epsilon,L}$. The strain energy of two particles in contact will be $W_{\epsilon,B}$. The assumption we make is that, if each interaction is integrated to the proper critical overlap, the strain energies will be equal at that point.

\begin{equation}
	W_{\epsilon,L} = W_{\epsilon,B} = \frac{8}{15}E_B^*\sqrt{R_B^*}\, {\delta_{c,B}}^{5/2}
\end{equation}

We solve for the interacting particle overlap as a function of the lab strain energy as

\begin{equation}
	\delta_{c,B} = \left[\frac{15W_{\epsilon,L}}{8E_B^*\sqrt{R_B^*}}\right]^{2/5}
\end{equation}

This overlap can be reinserted to Eq.~\ref{eq:hertzForce} to find the critical force of the interacting particles as a function of the critical strain energy of the lab. Doing this, we find:

\begin{equation}\label{eq:peb_hertz}
	F_{c,B} = C{E_B^*}^{2/5}{R_B^*}^{1/5}W_{\epsilon,L}^{3/5}
\end{equation}

where $C = \frac{4}{3}\left(\frac{15}{8}\right)^{3/5}$.

In this analysis we have referred to a `lab' and `particle' for the two situations. In fact, the result is more general and can be used to relate any two scenarios. The only requirement is that both conditions adhere to the assumptions of Hertz theory. The ramifications of this relationship will be explored in more detail in \S\ref{analysisExp}.





% FROM SOFT PAPER
\subsection{Pebble crushing predictions}
Along with proper material properties, we present a relationship to translate between the experimental data of crush force to a value that can be applied to DEM simulations. We relate crush force experimental data to predictions of crushing pebbles in an ensemble with:

\begin{equation}\label{eq:crush-predict}
  F_{c,B} = \frac{4}{3} \left(\frac{15}{8}\right)^{2/5}\left(R_B^* \right)^{1/5}\left( W_{\epsilon,L} \right)^{3/5}
\end{equation}

where $W_{\epsilon,L}$ is measured strain energy at the point of crushing from the experiment. This value follows a probability distribution and therefore imparts a distribution shape to the $F_{c,B}$ prediction. At the peak load of 6 MPa we use the above prediction to determine how many pebbles would be cracking at this state of external pressure