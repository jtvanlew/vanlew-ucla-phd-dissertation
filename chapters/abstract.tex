Predictive models of temperature distributions in packed beds of ceramics, beyond initial packing states, are critical for operation of solid breeders in fusion reactors. The importance of high-fidelity predictive capabilities is due to relatively narrow operational temperature windows imposed on solid breeders for optimum tritium release performance. Heat transfer in pebble beds is a strong function of their packing state as a consequence of coupling between mechanical forces in pebble beds and heat conductance among individual pebbles. Packed beds are often treated as fictitious continua, allowing them to be characterized with simpler phenomenological models of effective material properties. However, structural changes to initial packing states are expected in solid breeders due to high temperatures and forces in ceramic pebble regions. Continuum approaches are currently unable to adapt the models to packing changes and subsequent thermal reactions. Consequently, modeling approaches based on continuum assumptions are insufficient for ceramic breeder pebble beds of solid breeders. In the current study, a multi-scale approach is adopted as an alternative; the approach will allow direct modeling of packing structures and temperature distributions thereof.

Several phenomena are identified as capable of causing alterations to packing structures in solid breeders and thereby heat transfer properties, including: (i) inter-particle sintering, or necking; (ii) creep relaxation; (iii) crushing/cracking of individual pebbles in ensembles. These phenomena, and their specific changes to temperature distributions in pebble beds, will each require individualized attention and modeling efforts. In this study, I focus on heat transfer phenomena and develop predictive capabilities for temperature distributions in pebble beds with packing structures solely altered by pebble crushing and fragmentation.

From the microscopic point-of-view, the thermal discrete element method (DEM) is used to track motions of, and heat transfer between, individual pebbles in packed bed assemblies. DEM models provide an opportunity to study transient changes to packing structures and simulate pebble fragmentation and heat transfer. The slow-moving, interstitial helium purge gas is considered by coupling DEM models with macroscopic considerations of volume-averaged computational fluid dynamic (CFD) methods. Volume-averaged models of helium are computationally efficient and provide an global view of helium influence on heat transfer in solid breeder pebble beds. However, to gain insight into the complete fluid flow patterns and heat transfer in packed beds with changing packing states, the lattice-Boltzmann method (LBM) is also employed. The lattice-Boltzmann method is well-suited to fluid modeling of complex porous structures due to its inherent parallelizability and simple application of solid-fluid interface boundary conditions. 

Coupled CFD-DEM model are applied to ITER-relevant solid breeder volumes to parametrically consider fragmentation and orientation on temperatures in their pebble beds. We discovered that changes in packing structure and subsequent changes to pebble bed temperature distributions are sensitive to, in decreasing order, initial packing states (characterized by initial packing fractions), pebble fragmentation size, breeder unit orientation, and extent of pebble fragmentation (characterized by percent of crushed pebbles). At present, it is not possible to determine what percent of crushed pebbles should be expected in pebble beds, but the results of this study provides guidance on limits. For example, in this study when 1\% of pebbles were crushed, maximum temperatures in pebble beds increased by less than 5\%. However, when the number of crushed pebbles increased to 5\%, between 10 and 20\% increases in maximum bed temperatures were observed (among all parametrically studied pebble beds). To maintain within operational temperature windows, these results guide the tolerability of crushed pebbles in given pebble bed regions and orientations. We also discovered that, to the extent of pebble crushing considered in this study, no gap was formed between walls and pebble assemblies. Finally, we determined that an important contribution to temperature distributions were affected by small crush fragments traveling through interstitial gaps in pebble beds and creating small regions of increased nuclear heating due to redistribution of mass; temperature responses of beds with traveling fragments were also dependent upon breeder orientation.

The LBM model will be applied to small scale representative volumes of packed beds wherein the complete, tortuous flow of helium and the conjugate heat transfer in packed beds is studied. With LBM models...