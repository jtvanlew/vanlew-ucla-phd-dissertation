Ceramic pebble beds exploited in solid breeder designs are expected to operate reliably under the harsh environment inside a fusion reactor. The behavior of the many interacting individual pebbles composing the pebble bed dictate the macroscopic properties and effective transport characteristics of the tritium breeding volume. The discrete pebble is observed to generally fragment under relatively low contact forces; a characteristic which is expected to worsen under an irradiation environment. Breakage of particles in the ensemble will, among other potential repercussions, manifest in changes to heat transfer characteristics of the pebble bed. Furthermore, large stresses are anticipated in the pebble bed region in response to the confined thermal expansion of the pebbles heating from deposition of nuclear energy. Current modeling efforts of solid breeder pebble beds break down in the face of pebble damage and thus the fusion community is in need of modeling tools capable of predicting and modeling pebble damage and subsequent changes to thermophysical properties of the bed.

This dissertation describes the work behind the development and implementation of numeric tools to describe the pebble bed from the scale of the individual pebble in order to predict damage of distinct particles in the ensemble, model the fragmentation process of the particle, and monitor changes to the thermophysical characteristics of the pebble bed as a whole. The thermal model includes the pebble bed interaction with the flowing interstitial purge gas present in solid breeders. Two solid-fluid coupling methodologies are implemented in the study of conjugate heat transfer with the purge gas and pebble bed.

Experimental measurements of single pebbles are performed and used to validate and correct common assumptions of similar discrete particle modeling techniques, as applied to the ceramic pebbles. The same experimental measurements are analyzed and used as a translation tool between force measurements of a pebble between anvils in a test stand and forces between pebbles in an ensemble. Moreover, the crush force measurements of the experiments provide a predictive opportunity for crushing in the numeric pebble bed. In the development of the fluid-solid modeling efforts, a correction factor for Nusselt number correlations was introduced to allow the discrete particle models of heat transfer to remain valid even under circumstances of increased Biot numbers.

Finally, the computational models of the pebble bed and helium purge gas are used to analyze the response in effective thermal conductivity of ensembles with evolving descriptions of packing structures and determine possible limits of tolerability of damaged pebbles in solid breeders. Numerically, arbitrary amounts of pebbles are induced to crush and temperature distributions in several ITER-relevant geometries are studied. The results are available to ceramic pebble manufacturers as a criteria for necessary crush strength of their product. 