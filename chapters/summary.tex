\chapter{Summary \& Future Work}\label{sec:summary}
Ceramic pebble beds as tritium breeding volumes in fusion reactors must endure high power densities while maintaining both continued transport of high quality heat into coolants for power production as well as ceramic temperatures within relatively-norrow prescribed operating windows. The ceramic pebble beds, as non-cohesive granular material, exist with meta-stable packing structures that will evolve from the external and internal forces acting upon them during long-term operation as tritium breeders. As a consequence, predictive models of solid breeder heat transfer characteristics must contend with transient packing structures and the changing modes of heat transfer they present. To provide such predictive modeling, microscale numerical models were developed to allow investigation of thermal transport in pebble beds operating in environmental conditions relevant to planned fusion reactors. Specifically, this involved development and validation of: transient, three-dimensional, thermal DEM models of packed beds; fragmentation models to simulate pebble crushing in ensembles; fully-coupled, transient, volume-averaged CFD-DEM models of fluid-solid heat transfer; and one-way-coupled, transient DEM-LBM models of conjugate heat transfer in packed beds. %The new models grant us the ability to understand heat transfer in packed beds with altered packing structures, \textit{viz.}, crushed and fragmented pebbles. 

In the course of validating DEM-formulations of heat transfer in packed bed ensembles, new phenomonlogical descriptions of contact conductance and material properties were constructed. Experimental measurements of elastic response for individual pebbles showed a wide scatter that violates Hertzian predictions for contact force. However, it was shown that the force-travel response of individual pebbles follows the proper relationship predicted by Hertzian contact, namely $F_n \propto s^{3/2}$. Therefore we introduced a statistical spread with an \textit{apparent} Young's modulus fit to experimental measurements. The statistical spread was reproduced numerically in DEM simulations when applying Young's modulii to pebbles. In numeric simulations of uniaxial compression of DEM packed beds, volumes with the distributed apparent elastic moduli were seen to have nearly 40\% more strain for the same stress state. Additionally, a roughness model was introduced into DEM formulations of contact conductance. Comparing DEM results of effective thermal conductivity, models with roughness parameters (estimated from similar ceramic materials) reduced error by more than 120\%. 

A new phenomenological model of pebble crushing, which allows fragmentation, was developed for DEM to simulate crushed pebbles and concomitant effects on pebble bed thermomechanics. The modeling approach inserts fragments of arbitrary size into the system in an energy- and mass/volume-conserving fashion. A parametric study on fragment sizes revealed that smaller pebble fragments were seen to be capable of traveling relatively long distances before re-settling; approximately 10\% of pebbles traveled more than the original pebble diameter. Redistribution of mass was seen in increased local packing fractions of beds. Nuclear heating of the beds will also be affected by mass redistribution and was studied during applications of the models for ITER-relevant pebble beds. A discussion on radiation heat transfer among pebbles in an ensemble was able to show that at elevated pebble temperatures, \textit{i.e.} $>\SI{600}{\celsius}$, and small contact forces, \textit{i.e.} $< \SI{10}{\newton}$, radiation heat transfer can be as high as 50\% of that due to contact conductance. The combination of loosely re-settled small fragments and large view-factors between the fragments and large pebbles indicate that radiation exchange in DEM must be incorporated with future models, but was nevertheless neglected in the present study.

%Coupled models of volume-averaged CFD and DEM have existed in porous media research since the approach was introduced by Tsujita\etal~in 1992.\cite{Tsuji1992} 
The work of this thesis represents the first application of CFD-DEM modeling approach to solid breeder research for which nuclear heat generation in solids, tight-packed structures, and slow-moving purge gas fluid phase are dominate physical phenomena. Due to the unique operating conditions of solid breeders, advancements to the CFD-DEM code were also required to complete the research objectives of this work. Using an exact analytic solution for a single sphere with heat generation in a quiescent fluid, it was shown that lumped capacitance, an assumption inherent to DEM, remains valid for both transient and steady-state solutions only after a so-called Jeffreson correction is used in the calculation of heat transfer coefficient. Without the Jeffreson correction, nuclear heating of low conductivity pebbles caused error in both transient and steady-state calculations of temperatures in pebbles, even if a low Biot number indicates the validitiy of lumped capacitance assumption. The algebraic form of Jeffreson correction was implemented in the coupling between CFD-DEM with negligible computational overhead. Corrected versions of heat transfer coefficients compensated for low-Biot number errors due to low solid conductivity with large volumetric heat sources. Once the correction factor was implemented for calculations of inter-phase heat transfer exchange coefficients, temperature predictions of CFD-DEM models were validated against experimental measurements of effective conductivity with stagnant helium with excellent agreement. Numeric predictions of effective thermal conductivity at $\SI{600}{\celsius}$ reported values of \SI{1.05}{\watt\per\meter\per\kelvin} and \SI{1.20}{\watt\per\meter\per\kelvin} for beds of 62\% and 64\% packing fractions, respectively. Experimental measurements at similar temperatures ranged between \SI{1.18}{\watt\per\meter\per\kelvin} and \SI{1.3}{\watt\per\meter\per\kelvin}. Furthermore, the pressure drop measured in CFD-DEM packed bed models were shown to match the Kozeny-Carman correlation over the range of all Reynolds numbers relevant to solid breeder designs.

%Finally, advantages of using the lattice-Boltzmann method for studying packed or porous flow have been recognized since at least 1992 when Benzi\etal~studied its application for flows in complex geometries.\cite{Benzi1992} 
This thesis is also the first application of the lattice-Boltzmann method for modeling the conjugate heat transfer of flowing helium and volumetric heating of pebble beds in solid breeders. As a first of its kind study, much attention was paid to the method's ability to faithfully reproduce hydrodynamics in a computationally tractable manner. It was shown that proper selection of lattice spacing and resolution of digital mapping between DEM and LBM lattices are important for LBM simulations of fluid flow and heat transfer. Recommendations from literature suggest 4 nodes per throat are necessary for capturing hydrodynamics of channel flow. It was shown that when a \SI{1}{\milli\meter} pebble is represented by only 10 nodes, nearly 90\% of pore sizes were less than 4 nodes/pixels wide. For more reasonable resolutions of 20 and 40 pixels per \si{\milli\meter}, the violation of 4 node criteria was compared to the error in packed beds due to violating Knudsen criteria for continuum assumptions. The criteria for a gas to act as a continuum is that $\Kn < 10^{-2}$, for slightly greater $\Kn$, modified slip-conditions will exist at solid-fluid interfaces, and at large values of Knudsen number, fluids enter rarefied regimes. It was shown that the mean-free-path of helium from \SIrange{400}{900}{\celsius} results in packed beds violating Knudsen criteria in 22\% and 41\% of the void space, respectively. Therefore, without a quantifiable method to determine if matching continuum hydrodynamics is critical when the fluid itself is not matching continuum conditions, and for computational accessibility, resolutions of 20 and 40 pixels per \si{\milli\meter} were determined to be currently acceptable. It was then shown that beds with a resolution of 20 provided numerically stable solutions and error in velocity profiles for packed beds with a resolution of 20 were acceptable compared to res = 40 beds and that computational times for the former beds were nearly 20 times faster. Ultimately, lattice-Boltzmann simulations applied in this thesis maintained with lattices of 20 nodes per pebble diameter (\textit{i.e.} 20 pixels per \si{\milli\meter}).

Once the models and modeling approaches were established, they were applied to studying representative volumes of pebble beds as they would exist in current designs of tritium breeding modules. First, the thermomechanical impact of irradiation damage on ceramic materials was studied. Effective thermal conductivity values were numerically measured for pebble beds with irradiation-induced reductions in solid thermal conductivity. It was shown that effective thermal conductivity reduced linearly at a rate of $\keff [\si{\watt\per\meter\per\kelvin}] = 0.545 \eta + 0.458$ with $\eta = k_{irr}/k_{unirr}$. Helium purge gas was seen to help maintain thermal transport in packed beds as solid conductivity dropped. The trend was compared with SBZ correlations for effective thermal conductivity, a correlation which is itself a function of solid conductivity, and the numerical results agreed well for all values greater than $\kappa > 1$. The fit to SBZ is another validation of the numeric formulations of heat transfer in CFD-DEM, yet the models developed here can be expanded into predictive spaces well beyond the limited applicability of SBZ correlations.

CFD-DEM models were applied to study representative tritium-breeding ceramic pebble bed volumes with altered packing structures as a result of crushed pebbles. The studies were performed with parametric variations of: bed orientation with respect to gravity ($\chi$ or $\zeta$ configurations), pebble crushing amount ($\eta$), initial packing fraction ($\phi$), and crushed fragmentation size ($r^*$). One one general trend was observed that reiterates past conclusions from solid breeder research. Namely, more persistent behavior is witnessed in pebble beds with higher initial packing fractions. Total mean temperatures of $\phi_1 = 0.62$ beds increased between \numrange{16}{19}\% at $\eta = 5\%$. Yet beds initially packed to $\phi_2 = 0.64$ increased by only \numrange{10}{13}\% at the same value of crushed pebble amount. We therefore conclude that manual densification, from either long-term vibration packing or load-induced pre-compaction, should be performed on ceramic pebble bed volumes to gain some temperature control during operation in a fusion reactor. To achieve packing fractions of $64\%$ in the relatively small sizes of this study, for example, a load of over \SI{1}{\mega\pascal} was necessary. In the assembly of tritium breeding modules, it must be kept in mind that similar pre-compaction may need to be performed.

One purpose of the study was to identify the effects of breeder orientation on heat transfer after pebble fragmentation. In configurations where heat transfer out of the pebble bed is in parallel directions with gravity ($\zeta$-configuration), concerns have always focused on development of gap formation between pebble bed and upper walls. Over all the crush percentages studied here, up to 5\% total crushed pebbles, no gaps were detected at the upper wall. In fact, this configuration of breeder bed demonstrated the lowest maximum temperatures for beds with many crushed pebbles. We showed that freedom of fragments to travel between zones in these beds prevented a build-up of loose fragments (and thereby avoided build-up of heating) in the hottest regions. By comparison, beds with heat removal directions perpendicular to gravity ($\chi$-configuration) as $\eta$ went above 3\%, maximum temperature rise jumped well-above the $\zeta$-configurations. We showed that for these beds it was the inability for fragments to move between zones which left many small fragments to settle in the hottest region, further contributing to heating.

From the results of this study, pebble crushing and bed resettling effects on temperature are shown to be complicated, synergistic responses and are particular to breeder design and ceramic material employed. We found indications of certain operational spaces for which different designs responded less severely to pebble crushing. For instance, from the point of view of temperature response in pebble beds, if one were to employ a material known to have a limited crush strength, one might accept that many pebbles could break (at least up to 5\%, as studied here) over the life of the breeder and choose to employ the $\zeta$-style which avoided large increases in temperature after long operation of the breeder. Alternatively, if one had a ceramic material with a larger crush strength, the $\chi$-design would be preferable as it generally retained lower overall and maximum bed temperatures when fewer pebbles in the ensemble were crushed.


Lastly, using lattice-Boltzmann method simulations of helium flow through pebble beds, we looked into pore-scale influences of crushing on void fraction distribution and consequently velocity fields of pebble beds. We found that when 5\% of pebbles are fragmented, tortuosity of the pebble bed increased by nearly 7\% while the effective dispersive thermal conductivity increased over 23\%. In spite of the increase in dispersive conductivity, the impact on temperature profile spreading remained negligibly small due to the low Peclet number flow, as expected in tritium breeding blankets. LBM models confirmed that packed beds with low-Peclet flows were, in fact, well characterized by the considerably less computationally draining volume-averaging techniques of CFD-DEM, as measured by comparison of temperature profiles between LBM-DEM and CFD-DEM. Therefore, CFD-DEM coupling is the preferred method for packed beds with low Reynolds flow.

The framework of solving heat transfer in packed beds, considering interstitial flow and packing structure rearrangement, established by the numerical models in this thesis provide rich grounds for future research. The modeling approach of this thesis was to leverage widely-supported open-source codes as foundations. One advantage is that the codes adopted in this thesis have had their mathematical formulations of physics, time-integration techniques, and boundary condition implementations validated and verified. Moreover, the open-source codes have all their library source codes available for modification and expansion, permitting new contact laws and material property formulations to be implemented in a straight-forward manner. Next we discuss future research opportunities revealed during the course of this thesis.

\section{Recommendation for Future Work}

The models of packed beds with fragmentation developed in this thesis were meant to study changes in heat transfer due to alterations in the structure. The work was inspired by observations made on fragmented \lis~and cracked \lit~pebbles coming out of post-irradiation experiments -- as well as the observed low crush force of \lis~in single pebble crush experiments. Paradoxically, the models of this thesis were developed because experimental observation of packing structures with fragmented pebbles is extremely difficult, and yet the numerical developments can not be fully validated without proper experimental data. Future experiments must be carried out to yield data which can directly validate DEM results of fragmentation.

Models of crushing and heat transfer employed on top of LIGGGHTS can readily be implemented alongside modules describing other thermomechanical phenomena. Beyond pebble crushing, there are several other causes that may lead to alterations of packing structures inside solid breeder volumes. For example, in pebble bed experiments of \lit~at UCLA, we have observed both contact sintering and necking amongst pebbles when exposed to temperatures as low as \SI{700}{\celsius} with prolonged external stress as well as crush forces nearly 10 times larger than \lis~pebbles. For these pebbles, crushing and bed restructuring appear far less important than cohesion at contacts. Contact forces, temperatures, and duration of contact is available from the DEM code and can be used for implementation of contact creep and contact sintering phenomonological or theoretical models. These effects could be studied in tandem with crush predictions for \lit~or independently.

As has been discussed, the current implementation of DEM has no mathematical description of radiative heat transfer in packed beds. It was shown that radiation could be as important a heat transfer driver as contact conductance for certain pebbles in an ensemble. In fact, considerations of radiation may even overshadow the importance of packed bed restructuring in terms of overall heat transfer in packed beds. Inclusion of radiation in DEM is only limited by numerical schemes as ray-tracing algorithms for determining view factors has been shown in literature to be currently attainable.

DEM and CFD-DEM models should also be applied to studying methods for enhancing heat transfer in pebble beds from mixed breeder/multiplier concepts such as the Japanese-proposed titanium beryllides (\textit{e.g.} \ce{Be12Ti}). Mixed volumes of both neutron multiplier and tritium breeder may have high tritium breeding ratios with acceptable tritium inventory of the beryllides. Moreover, the dense packing available from potentially large differences of pebble radii in conjunction with high solid conductivity of beryllides can have advantageous impacts on heat transfer in pebble beds. DEM can reveal changes to contact forces, contact conductance, and segregation of material after many thermal cycles. 

Finally, large packing fractions are generally avoided in solid breeder blankets in order to avoid their concomitant increase in pressure drop. However, from the investigation performed with heat transfer in lattice-Boltzmann simulations, it is worth revisiting the topic from the point of view of energy management and the ability for packed beds to sustain larger power densities. We observed that transverse thermal dispersion was negligible at low Peclet numbers and with mono-sized pebble packings in the LBM-based study. Increasing flow velocity will increase Peclet number and pressure drop, though at different rates. On the one hand, pressure drop is linearly proportional to $u_f$ until about $\Re = 100$. As an example, the Korean TBM design has a particle $\Re \approx 1$. This could allow a hundred-fold increase in velocity with only a linearly similar increase in pressure drop. On the other hand, Kuwahara \etal~fit a correlation for $k_\text{disp}$ to be proportional to $\Pe^{1.4}$, and thus $u_f^{1.4}$ and $\epsilon^{-1.4}$. As a consequence, dispersive conductivity increases at a much more rapid rate than pressure drop and heat transfer enhancements due to dispersive conductivity may overwhelm the negative effects of increased pressure drop. Therefore, it appears worthwhile to study packed beds with high packing fraction and increased Reynolds number to ascertain the benefits of higher dispersion on overall effective thermal conductivity.
