\chapter{Summary}\label{sec:summary}
The objectives of the research was to develop predictive capabilities for temperature distributions in tritium breeding pebble beds with altered packing structures. The scope of the thesis includes development of DEM, coupled CFD-DEM, and LBM-DEM models; and application of the models under several fusion-relevant conditions to provide understanding of packing alterations and their ramifications on temperature distributions.

\section{Development of Predictive Capabilities}
\subsection{Discrete Element Method Models of Solid Breeder Heat Transfer}
The discrete element method is important for blanket researchers and designers because it allows interrogation of solid-solid interactions in pebble beds and nature of heat transfer on scales at which packing structures operate. In this way, the models allow exploration of evolving packing structures that will occur during operation of a ceramic pebble bed. The DEM modeling of heat transfer was validated against experimental measurements of effective thermal conductivity of pebble beds in vacuum, when contact conductance dominates. I showed that smooth-sphere approximation of pebbles was seen to over-predict the ability of pebbles to transport energy between contacts. Reductions in contact conductance due to roughness were then shown to allow DEM models to approach experimental measurements.

A new phenomenological model of fragmentation is introduced into DEM for simulating crushed pebbles. The modeling approach allows fragments of arbitrary size to be inserted into the system in an energy- and mass/volume-conserving fashion. A parametric study on fragment sizes revealed that smaller pebble fragments were seen to be capable of traveling relatively long distances before re-settling. Redistribution of mass was seen in increased local packing fractions of beds. Nuclear heating of the beds will also be affected by mass redistribution and was studied during applications of the models for ITER-relevant pebble beds.

From a mechanical perspective on packed bed modeling with DEM, experimental measurements on crushed pebbles showed a wide spread in force-displacement responses. A novel approach to capture the phenomena numerically was with a modified Young's modulus which replicated the reduction in stiffness seen in experiments. The model with modified Young's moduli predict more compliant pebble beds and smaller peak contact forces in beds, which ultimately led to fewer predicted crushed pebbles from DEM models.


\subsection{CFD-DEM}
The interstitial helium purge gas is introduced into the heat transfer modeling of solid breeders by means of a continuum approach with volume-averaged integrations of Navier-Stokes and energy conservation equations. Coupling between the volume-averaged computational fluid dynamics (CFD) model and DEM is dynamic and two-way. Experimental correlations for Nusselt number and drag coefficient are used to calculate exchange coefficients in every cell of the CFD mesh. The exchange coefficients are applied to heat transfer and momentum equations for DEM particles located inside the CFD cell and the summation of those interactions, on a volume-average basis, are included as source terms to the fluid conservation equations. The volume-averaged approach to solving conservation equations for helium allows for efficient computational calculation of two-phase flow in packed beds.

Using an exact analytic solution for a single sphere with heat generation in a quiescent fluid, I showed how the lumped capacitance solution can remain valid for both transient and steady-state solutions after a so-called Jeffreson correction was used in the calculation of heat transfer coefficient from Nusselt correlations. The algebraic form of Jeffreson correction was implemented in the coupling between CFD-DEM with negligible computational overhead. Corrected versions of heat transfer coefficients compensated for low-Biot number errors due to low solid conductivity with large volumetric heat sources.

Heat transfer predictions of CFD-DEM models were validated against numerical measurements of effective conductivity with stagnant helium. Furthermore, the pressure drop measured in CFD-DEM packed bed models were shown to match the Kozeny-Carman correlation over the range of all Reynolds numbers relevant to solid breeder designs.

\subsection{DEM-LBM}
6) Method for mapping DEM to LBM
7) Resolution \& effects on hydrodynamics

% SUMMARY OF CH 5
Toward development of hydrodynamically accurate lattice-Boltzmann method (LBM) nodal construction, several numerical studies were performed. In



We also showed that proper selection of lattice properties can lead to stable solutions that are also faithful to the macroscopic fluid mechanics being modeled. Consistency in packing structure representation is maintained between DEM and LBM through modification of pebble radii when mapped on LBM voxels; accomplished through measurements of digital packing fraction in LBM lattices. We then considered the effects of grid sizing and resolution on stability of packed bed simulations. For densely packed beds, we found a resolution of 20 pixels/nodes per pebble diameter was sufficient for results that were: stable, capable of reproducing correct hydrodynamics, and computationally tractable.

\section{Application of models}
Once preliminary validation had been performed with the DEM, CFD-DEM, and LBM-DEM models developed, the models were applied to various studies of fusion-relevant cases of pebble bed volumes that extend beyond currently known configurations.

In the first application of the CFD-DEM models developed for this research, a study was performed on the effects of irradiation damage to solid conductivity on overall effective thermal conductivity of pebble beds. The study was carried out by numerically reducing solid conductivity and measuring the resultant effective conductivity in a bed with and without stagnant helium. From the results, it was seen that effective conductivity of pebble beds with irradiation damage to the solid phase conductivity followed trends predicted by effective thermal conductivity correlations such as the SBZ model.

I then used the CFD-DEM modeling tools to conduct several multi-scale studies of granular heat transfer in representative tritium-breeding ceramic pebble bed volumes. The studies focused on measuring the temperature response in tritium breeding volumes with altered packing structures as a result of crushed pebbles. The studies were performed with parametric variations of: bed orientation with respect to gravity, pebble crushing amount, initial packing fraction, and crushed fragmentation size.

The results from the models developed in this thesis revealed that pebble crushing and bed resettling effects on temperature are complicated, synergistic functions of breeder design and ceramic material employed. From the predicted temperature distributions, several recommendations to breeder designers were put forth. For instance, if one were using a material known to have a limited crush strength, one might accept that many pebbles could break (at least up to 5\%, as studied here) over the life of the breeder and choose to employ the European ITER-style TBM because it saw smaller increases in temperature after long operation of the breeder. Alternatively, if one had a ceramic material with a larger crush strength, the vertical design of TBM, common to many other Party designs, would be preferable as it generally retained lower overall and maximum bed temperatures when fewer pebbles in the ensemble were crushed.

Finally, lattice-Boltzmann model developed in this thesis was applied to study effects of helium's tortuous flow through tightly packed solid breeder volumes. With the model, I was capable of considering pore-scale influences of crushing on void fraction distribution and consequent velocity fields of pebble beds. I found that crush fragments increased tortuosity of pebble beds, in spite of the constant overall void fraction of the bed overall. The increased tortuosity resulted in considerably higher transverse thermal dispersion for pebble beds with crushed fragments; however, transverse thermal conductivities for crushed and non-crushed beds were negligible compared to molecular diffusion of heat in transverse directions as a result of low Peclet numbers anticipated for solid breeder purge gas. I also showed that temperature distributions in low-Peclet packed beds were well characterized by the considerably less computationally draining volume-averaging techniques of CFD-DEM, as measured by comparison of temperature profiles between LBM-DEM and CFD-DEM.

\section{Recommendation for Future Work}
There are many causes that can lead to alterations of packing structures inside solid breeder volumes. Yet in this thesis, I considered only the heat transfer response to packing alterations due to crushed pebbles and fragmentation.


Moreover, large packing fractions are generally avoided in solid breeder blankets in order to avoid their concomitant increase in pressure drop. However, from the investigation performed with heat transfer in lattice-Boltzmann simulations, it is worth revisiting the topic from the point of view of energy management and the ability for packed beds to sustain larger power densities. The lattice-Boltzmann technique can, however, help in studying novel concepts for increasing heat transfer in packed beds in order to increase power density, such as the increase in dispersive conductivity due to mixed pebble beds and high packing fractions.
