\chapter{Summary}\label{sec:summary}
The objectives of the research was to develop predictive capabilities for temperature distributions in tritium breeding pebble beds with altered packing structures. The scope of the thesis includes development of DEM, coupled CFD-DEM, and LBM-DEM models; and application of the models under several fusion-relevant conditions to provide understanding of packing alterations and their ramifications on temperature distributions.


\section{Discrete Element Method Models of Solid Breeder Heat Transfer}
The discrete element method is important for blanket researchers and designers because it allows interrogation of solid-solid interactions in pebble beds and nature of heat transfer on scales at which packing structures operate. In this way, the models allow exploration of evolving packing structures that will occur during operation of a ceramic pebble bed. The DEM modeling of heat transfer was validated against experimental measurements of effective thermal conductivity of pebble beds in vacuum, when contact conductance dominates. I showed that smooth-sphere approximation of pebbles was seen to over-predict the ability of pebbles to transport energy between contacts. Reductions in contact conductance due to roughness were then shown to allow DEM models to approach experimental measurements. 

A new phenomenological model of fragmentation is introduced into DEM for simulating crushed pebbles. The modeling approach allows fragments of arbitrary size to be inserted into the system in an energy- and mass/volume-conserving fashion. A parametric study on fragment sizes revealed that smaller pebble fragments were seen to be capable of traveling relatively long distances before re-settling. Redistribution of mass was seen in increased local packing fractions of beds. Nuclear heating of the beds will also be affected by mass redistribution and was studied during applications of the models for ITER-relevant pebble beds.

From a mechanical perspective on packed bed modeling with DEM, experimental measurements on crushed pebbles showed a wide spread in force-displacement responses. A novel approach to capture the phenomena numerically was with a modified Young's modulus which replicated the reduction in stiffness seen in experiments. The model with modified Young's moduli predict more compliant pebble beds and smaller peak contact forces in beds, which ultimately led to fewer predicted crushed pebbles from DEM models.


\section{CFD-DEM}
The interstitial helium purge gas is introduced into the heat transfer modeling of solid breeders by means of a continuum approach with volume-averaged integrations of Navier-Stokes and energy conservation equations. Coupling between the volume-averaged computational fluid dynamics (CFD) model and DEM is dynamic and two-way. Experimental correlations for Nusselt number and drag coefficient are used to calculate exchange coefficients in every cell of the CFD mesh. The exchange coefficients are applied to heat transfer and momentum equations for DEM particles located inside the CFD cell and the summation of those interactions, on a volume-average basis, are included as source terms to the fluid conservation equations. The volume-averaged approach to solving conservation equations for helium allows for efficient computational calculation of two-phase flow in packed beds. 

Using an exact analytic solution for a single sphere with heat generation in a quiescent fluid, I showed how the lumped capacitance solution can remain valid for both transient and steady-state solutions after a so-called Jeffreson correction was used in the calculation of heat transfer coefficient from Nusselt correlations. The algebraic form of Jeffreson correction was implemented in the coupling between CFD-DEM with negligible computational overhead. Corrected versions of heat transfer coefficients compensated for low-Biot number errors due to low solid conductivity with large volumetric heat sources.

Heat transfer predictions of CFD-DEM models were validated against numerical measurements of effective conductivity with stagnant helium. Furthermore, the pressure drop measured in CFD-DEM packed bed models were shown to match the Kozeny-Carman correlation over the range of all Reynolds numbers relevant to solid breeder designs. 

\section{DEM-LBM}
6) Method for mapping DEM to LBM
7) Resolution \& effects on hydrodynamics

% SUMMARY OF CH 5
The lattice-Boltzmann method of modeling fluid flow was introduced and its merits for application to porous flow discussed. We have shown a method for mapping data of packing structures, generated in DEM, onto nodes of lattices solving momentum/mass and energy conservation with collision-streaming operations of the lattice-Boltzmann method. The LBM approach was chosen in place of finite element or finite volume methodology because of, firstly, the extreme ease with which boundary conditions can be applied inside the highly complex packing structure of ceramic pebble beds; enforcing no-slip conditions on complex geometry is trivially realized with bounce-back rules on distribution functions. Furthermore, discretization of fluid domains in lattice-Boltzmann frameworks requires no special meshing in the highly-skewed regions near contacting pebbles, such as is necessary with standard CFD/FEM solvers. The multi-relaxation-time lattices for momentum and energy offer complete modeling of complex geometry and conjugate heat transfer with far less computational overhead compared to FEM models.

We also showed that proper selection of lattice properties can lead to stable solutions that are also faithful to the macroscopic fluid mechanics being modeled. Consistency in packing structure representation is maintained between DEM and LBM through modification of pebble radii when mapped on LBM voxels; accomplished through measurements of digital packing fraction in LBM lattices. We then considered the effects of grid sizing and resolution on stability of packed bed simulations. For densely packed beds, we found a resolution of 20 pixels/nodes per pebble diameter was sufficient for results that were: stable, capable of reproducing correct hydrodynamics, and computationally tractable.

\section{Application of models}
8) Irradiation study

As an example of the applicability of these results, suppose we wish to find the amount of irradiated damage is allowable in a ceramic breeder region. In this scenario, a blanket designer chooses to allow a maximum operating temperature with 10\% margin away from \SI{950}{\celsius}. The planned bed mid-line temperature would therefore be \SI{855}{\celsius}. To find the amount of damage allowable within the 10\% margin, we then know $\Theta = \frac{950-573}{855-573} = 1.34$. Thus, solving for $\eta$ in \Cref{eq:theta-eta-midline} yields $\eta = 0.52$; a solid conductivity of $k_{s,irr} = \SI{1.25}{\watt\per\meter\per\kelvin}$. The last step would require knowledge of a relationship between solid conductivity and dpa. With such information, allowable dpa could be ascertained and the 10\% design margin on temperature evaluated.

Neutrons from fusion plasma are highly energetic and it is important to consider how these neutrons will affect heat transport internal to lithium ceramics during their operation in a fusion reactor. In this study, we varied solid thermal conductivity parametrically as a proxy to represent material damaged by neutron irradiation. When stagnant helium gas is included in the calculation, the results fit well within limits of correlations and theoretical limits above $\kappa \approx 1$. We also arrived at correlations relating effective thermal conductivity and maximum bed temperatures as a function of irradiated solid conductivity. At present we have no data indicating the precise quantity of expected dpa in lithium ceramics, nor a correlation between dpa and reduced conductivity. There is, however, irradiation data from a high-dose fission experiment, HICU, in which it appears conductivity values did not decrease dramatically at the temperatures and dpa experienced. Next steps should be to study precisely the relationship between dpa and conductivity in the lithiated ceramic materials considered for use in ITER and future DEMO reactors.

9) Orientation study

We conducted several multi-scale simulations of granular heat transfer using coupled CFD-DEM simulations of representative tritium-breeding ceramic pebble bed volumes with parametric variations of: bed orientation with respect to gravity, pebble crushing amount, initial packing fraction, and crushed fragmentation size.

There was one general trend observed that reiterates past conclusions from solid breeder research. Namely, more persistent behavior is witnessed in pebble beds with higher initial packing fractions. In this study, the most dominant parameter observed to affect temperatures in pebble beds was the initial packing fraction: beds with higher initial packing fraction had smaller increases in bed temperatures due to pebble crushing. We therefore conclude that manual densification, from either long-term vibration packing or load-induced pre-compaction, must be done to ceramic pebble bed volumes to gain some temperature control during operation in a fusion reactor. To achieve packing fractions of $64\%$ in the relatively small sizes of this study, for example, a load of over \SI{1}{\mega\pascal} was necessary. In the assembly of tritium breeding modules, it must be kept in mind that similar pre-compactions may need to be performed.

As stated earlier, a concern with the $\zeta$ design is the possibility of gap formation between pebble bed and upper walls after bed resettling, particularly after pebble fragmentation. In this study, we found pebble beds initially packed to $\phi = 62\%$ experienced the highest increases in both total average bed temperature as well as maximum temperature rise. The comparably looser packing allowed a quick reduction in bed stresses and consequently a reduction in heat conduction to the upper wall. Nevertheless, no gaps were detected even at 5\% of pebbles crushed. However, temperatures in pebble beds packed to 64\% showed a resistance to fragmentation; overall average temperatures were comparable to $\chi$-configurations, and in fact these EU-style beds had the lowest maximum temperatures for beds with many crushed pebbles. We showed that freedom of fragments to travel between zones in these beds prevented a build-up of loose fragments (and thereby avoided build-up of heating) in the hottest regions.

As for the $\chi$-configurations, we found that when there were not many broken pebbles ($\eta \le 3$\%), these beds generally had lower temperatures in comparison to similar $\zeta$-config beds. But as $\eta$ went above 3\% for many of the beds, the averaged bed temperature and, importantly, the maximum temperature rise actually jumped above the $\zeta$-configurations. We showed that for these beds it was the inability for fragments to move between zones which left many small fragments to settle in the hottest region, further contributing to heating.

From the results we have shown, it is obvious that pebble crushing and bed resettling effects on temperature are complicated, non-linear functions of breeder design and ceramic material employed. We found indications of certain operational spaces for which different designs excelled. For instance, if one were using a material known to have a limited crush strength, one might accept that many pebbles could break (at least up to 5\%, as studied here) over the life of the breeder and choose to employ the $\zeta$-style which avoided large increases in temperature after long operation of the breeder. Alternatively, if one had a ceramic material with a larger crush strength, the $\chi$-design would be preferable as it generally retained lower overall and maximum bed temperatures when fewer pebbles in the ensemble were crushed.

This study was performed on some generic geometries and has provided some generalized conclusions. But in light of the pebble beds' complex responses, as breeder designs continue to evolve into their final form before deployment in ITER, CFD-DEM models should continuously be employed to study the specific thermomechanical responses to pebble crushing and bed resettling unique to each design.
10) LBM study

Much effort was made to accurately model the tortuous flow of helium in packed beds. Using the lattice-Boltzmann method, we looked into pore-scale influences of crushing on void fraction distribution and consequently velocity fields of pebble beds. We found that when 5\% of pebbles are fragmented, tortuosity of the pebble bed increased by nearly 7\% while the dispersive conductivity increased over 23\%. In spite of the increase in dispersive conductivity, the impact on temperature profile spreading remained negligibly small due to the low Peclet number flow, as expected in tritium breeding blankets. The result, however, does inspire a path of research toward allowing higher power density in solid breeders. The pebble bed with crush fragments studied here can be considered as a low-porosity, polydisperse pebble bed. Polydispersity in pebble beds can lead to very high packing fractions.

Because the Darcian velocity increases with more tightly packed beds (smaller void fraction), there appears to be some control over dispersive conductivity by means of polydisperse packings. Using material properties for helium and \SI{1}{\milli\meter} pebbles (neglecting for a moment the influence of polydispersity), we can calculate dispersive conductivity as a function of void fraction. In \Cref{fig:lbm-dis-peclet}, the example calculation is shown as a function of packing fraction. Experimentally, packing fractions in polydisperse beds can easily reach $\phi = 0.80$.\cite{Reimann:2000tw} From \Cref{fig:lbm-dis-peclet}, dispersive conductivity (again, calculated assuming single sized pebbles) increases by an order of magnitude for packing fractions around $\phi =0.88$.

Large packing fractions are generally avoided in solid breeder blankets in order to avoid their concomitant increase in pressure drop. However, from the investigation performed with heat transfer in lattice-Boltzmann simulations, it is worth revisiting the topic from the point of view of energy management and the ability for packed beds to sustain larger power densities.

The lattice-Boltzmann method allowed us to study the tortuous path of helium through packed beds of spheres with and without crush fragments. Unlike traditional CFD methods for packed beds, no simplifications of contact regions was required and a direct mapping, which maintained consistency of packing fractions, was performed when digitizing DEM packing structures onto LBM lattice nodes. LBM simulations were run on a machine with quad-core, 3 GHz Intel Xeon E6-1607. Computation times for the lattices with a resolution of 20 voxels per diameter, up to 400 seconds of real time simulation were generally around 30 hours. On the same machine, CFD-DEM simulations generally concluded in approximately 4 hours. Scaling up to a full-size pebble bed would require significant computational expenses. Yet, owing to the small Peclet number of pebble beds under consideration, there does not appear to be any evidence that considerations of the entire helium flow field is necessary when attempting to model thermomechanical interactions of solid breeders. The lattice-Boltzmann technique can, however, help in studying novel concepts for increasing heat transfer in packed beds in order to increase power density, such as the increase in dispersive conductivity due to mixed pebble beds and high packing fractions.

\section{Recommendation for Future Work}
NOTHING ITS BULLSHITS