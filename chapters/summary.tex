\chapter{Summary}\label{sec:summary}

Numerical modeling development and application were conducted to understand thermal responses of pebble beds under conditions of altered packing, \textit{viz.}, crushed and fragmented pebbles and ultimately demonstrated predictive capabilities of our models under these conditions. The scope of the thesis includes development of DEM, coupled CFD-DEM, and LBM-DEM models; and application of the models under several fusion-relevant conditions to provide understanding of packing alterations and their ramifications on temperature distributions.

%\section{Development of Predictive Capabilities and Modeling}

%The discrete element method was first applied to studying solid breeder pebble beds with two-dimensional models by Zi Lu at UCLA in 2000.\cite{Lu2000} The modeling approach is recognized as providing valuable information of packing states in pebble beds and, as such, has gainted traction in the fusion community and now groups in the US, EU, India, Japan, Korea, and China have all begun DEM modeling efforts. 
This thesis presents the most complete and validated three-dimensional, dynamic thermal DEM modeling for solid breeders, to date. Currently-available DEM models of heat transfer due to contact conductance was validated against experimental measurements of effective thermal conductivity of pebble beds in vacuum, when contact conductance dominates. I showed that smooth-sphere approximation of pebbles, as used in all other DEM models of packed beds, was seen to over-predict the ability of pebbles to transport energy between contacts with an error of 155\%. A roughness model was implemented into the numeric calculation of contact conductance in DEM and the resultant effective thermal conductivity calculations reduced the error to only approximately 50\%. 

A new phenomenological model of fragmentation was developed for DEM to simulate crushed pebbles and concomitant effects on pebble bed thermomechanics. The modeling approach allows fragments of arbitrary size to be inserted into the system in an energy- and mass/volume-conserving fashion. A parametric study on fragment sizes revealed that smaller pebble fragments were seen to be capable of traveling relatively long distances before re-settling; approximately 10\% of pebbles traveled more than the original pebble diameter. Redistribution of mass was seen in increased local packing fractions of beds. Nuclear heating of the beds will also be affected by mass redistribution and was studied during applications of the models for ITER-relevant pebble beds. From a mechanical perspective on packed bed modeling with DEM, experimental measurements on crushed pebbles showed a wide spread in force-displacement responses. A novel approach to capture the phenomena numerically was with a modified Young's modulus which replicated the reduction in stiffness seen in experiments. Variations in elastic modulus is implemented numerically with Gaussian distributions fit to experimental measurements in DEM. DEM models with modified Young's moduli predict more compliant pebble beds and smaller peak contact forces in beds, which ultimately led to fewer predicted crushed pebbles from DEM models.

%Coupled models of volume-averaged CFD and DEM have existed in porous media research since the approach was introduced by Tsujita\etal~in 1992.\cite{Tsuji1992} 
The work of this thesis also represents the first application of CFD-DEM approach to solid breeder research for which nuclear heat generation in solids, tight-pack structures, and slow-moving purge gas fluid phase are dominate physical phenomena. Due to the unique operating conditions of solid breeders, advancements to the CFD-DEM code were also required to complete the research objectives of this work. Using an exact analytic solution for a single sphere with heat generation in a quiescent fluid, I showed how the lumped capacitance method, an assumption inherent to DEM, remains valid for both transient and steady-state solutions after a so-called Jeffreson correction is used in the calculation of heat transfer coefficient from Nusselt correlations. The algebraic form of Jeffreson correction was implemented in the coupling between CFD-DEM with negligible computational overhead. Corrected versions of heat transfer coefficients compensated for low-Biot number errors due to low solid conductivity with large volumetric heat sources. Once the correction factor was implemented for calculations of inter-phase heat transfer exchange coefficients, temperature predictions of CFD-DEM models were validated against experimental measurements of effective conductivity with stagnant helium with excellent agreement. Furthermore, the pressure drop measured in CFD-DEM packed bed models were shown to match the Kozeny-Carman correlation over the range of all Reynolds numbers relevant to solid breeder designs.

%Finally, advantages of using the lattice-Boltzmann method for studying packed or porous flow have been recognized since at least 1992 when Benzi\etal~studied its application for flows in complex geometries.\cite{Benzi1992} 
Finally, in terms of modeling development, in this thesis is also the first application of the lattice-Boltzmann method for modeling the conjugate heat transfer of flowing helium and volumetric heating of pebble beds. We also showed that proper selection of lattice properties is critical for LBM simulations of fluid flow and heat transfer. For densely packed solid breeder beds, we found a resolution of 20 voxels per pebble diameter was sufficient for results that were: stable, capable of reproducing correct hydrodynamics, and, importantly, computationally tractable. Consistency in packing structure is also guaranteed with a numerical scheme for mapping DEM pebble information and LBM with automated  modification of pebble radii; accomplished through measurements of digital packing fraction in LBM lattices. The schemes are written as methods available to for 

Studying the thermomechanical impact of irradiation damage on ceramic materials, temperature profiles were predicted for pebble beds with irradiation-induced reductions in solid thermal conductivity. With interstitial fluid, reduction in effective thermal conductivity was linearly related to reduction in solid conductivity. The model, in the limit of stagnant interstitial fluid was shown to match common SBZ correlations for effective thermal conductivity as a function of solid conductivity. However, the CFD-DEM-based model is also applicable in the cases of altered packing, flowing fluid, and other mixed pebble bed types which are all conditions beyond the realm of SBZ application.

CFD-DEM models were used to conduct multi-scale studies of granular heat transfer in representative tritium-breeding ceramic pebble bed volumes with altered packing structures as a result of crushed pebbles. The studies were performed with parametric variations of: bed orientation with respect to gravity, pebble crushing amount, initial packing fraction, and crushed fragmentation size. The models revealed that pebble crushing and bed resettling effects on temperature are complicated, synergistic functions of breeder design and ceramic material employed. From the predicted temperature distributions, several recommendations to breeder designers were put forth. For instance, if one were using a material known to have a limited crush strength, one might accept that many pebbles could break (at least up to 5\%, as studied here) over the life of the breeder and choose to employ the European ITER-style TBM because it saw smaller increases in temperature after long operation of the breeder. Alternatively, if one had a ceramic material with a larger crush strength, the vertical design of TBM, common to many other Party designs, would be preferable as it generally retained lower overall and maximum bed temperatures when fewer pebbles in the ensemble were crushed.

The tortuous advection of energy from helium moving through the complex interstitial voids in packed beds, over the range of Reynolds numbers currently relevant to solid breeder designs, was negligible compared to molecular diffusion of energy in the direction transverse to flow. Nevertheless, for equivalent packing fractions, pebble beds with crush fragments increased tortuosity of pebble beds due to localized increases in packing fractions, in spite of the constant overall void fraction of the bed overall. The increased tortuosity resulted in considerably higher transverse thermal dispersion for pebble beds with crushed fragments, an increase of 23\%. However, transverse thermal conductivities for crushed and non-crushed beds were both negligible compared to heat conducted in transverse directions. LBM models confirmed that packed beds with low-Peclet flows were, in fact, well characterized by the considerably less computationally draining volume-averaging techniques of CFD-DEM, as measured by comparison of temperature profiles between LBM-DEM and CFD-DEM.

\section{Recommendation for Future Work}

Any predictive tool must be validated. The models developed in this thesis were meant to study packed beds due to alterations in the structure due to crushed pebbles because of observations made on fragmented \lis~and cracked \lit~pebbles coming out of post-irradiation experiments -- as well as the observable low crush force of \lis~in single pebble crush experiments. Paradoxically, the models of this thesis were developed because experimental observation of packing structures with fragmented pebbles is extremely difficult, and yet the numerical developments can not be validated without proper experimental data. Careful attention must be paid to experimental efforts which may yield data which can directly validate DEM results, even on a small scale, such that large-scale simulations which are untenable with experiments may be performed. 

The numeric models and schemes developed in this thesis were done on top of popular open-source DEM code, LIGGGHTS, CFD code, OpenFOAM, and LBM code, Palabos. This was done with the intention that the modules developed here may be used by future researchers on pebble beds. Beyond pebble crushing, however, there are several other causes that may lead to alterations of packing structures inside solid breeder volumes and they can be added on top of the code modules developed here. For example, contact forces and other information as a function of time is available from the DEM code. While \lit~pebbles have shown themselves to be much more resistant to crushing (crush forces are nearly an order of magnitude larger than \lis~pebbles) experiments we have performed on \lit~pebble beds have revealed their tendency to `neck' or sinter at contact areas. Sintering and creep-diffusion may have a major impact on heat transfer in pebble beds and we are capable of modeling them on the pebble scale with DEM. 

DEM and CFD-DEM models should also be applied to studying methods for enhancing heat transfer in pebble beds due to mixed breeder/multiplier concepts such as the Japanese design of titanium beryllides (\ce{Be12Ti}). Mixed volumes of both neutron multiplier and tritium breeder may have high tritium breeding ratios with acceptable tritium inventory of the beryllides. Moreover, the dense packing available from potentially large differences of pebble radii can have advantageous impacts on heat transfer in pebble beds. DEM can reveal changes to contact forces, contact conductance, and segregation of material after many thermal cycles. 

Finally, large packing fractions are generally avoided in solid breeder blankets in order to avoid their concomitant increase in pressure drop. However, from the investigation performed with heat transfer in lattice-Boltzmann simulations, it is worth revisiting the topic from the point of view of energy management and the ability for packed beds to sustain larger power densities. We observed that transverse thermal dispersion was negligible at low Peclet numbers and with mono-sized pebble packings in the LBM-based study. Increasing flow velocity will increase Peclet number and pressure drop, though at different rates. On the one hand, pressure drop is linearly proportional to $u_f$ until about $\Re = 100$, above which it is proportional to $u_f^2$. As an example, the Korean TBM design has a particle $\Re \approx 1$. This could allow a ten-fold increase in velocity with only a linearly similar increase in pressure drop. On the other hand, Kuwahara \etal~fit a correlation for $k_\text{disp}$ to be proportional to $\Pe^2$, and thus $u_f^4$. As a consequence, dispersive conductivity increases at a much more rapid rate than pressure drop and heat transfer enhancements due to dispersive conductivity may overwhelm the negative effects of increased pressure drop.
