\chapter{Remaining Work and Challenges}\label{sec:remaining}

Complete the development of numerical tools
Validate with experimental data \& analytically-based correlations of effective conductivity with stagnant gas (e.g. SBZ model)

Augment pebble damage simulations with the crush-prediction algorithm

Apply the complete set of tools to two different configurations of breeders designed for ITER


Stability of LBM balanced with sufficiently high resolution of pebbles.

\section{Remaining Issues to Address and Overcome}
Discrete Element Method
Conclude if conduction model requires a roughness coefficient to match experimental data

Volume-averaged fluid coupling to DEM
CFD-DEM simulations may not capture laminar mixing of energy as shown by LBM models, but are significantly faster to simulate. Quantify error against the ‘exact’ solution of LBM
Lattice-Boltzmann Stability
Discover nondimensional, lattice parameters in scaled packed bed with stronger stability


%%%%%%%%%%%%%%%%%%%%%%%%%%%%%%%%%%%%%%%%%%%%%%%%%%%%%%%%%%%%%%%%%%%%%%%%%%%%%%%%%%%%%%%%%%%%%%
\section{Applications to Real Solid Breeder Blanket Designs}\label{sec:applied-studies}

In this study we apply coupled computational fluid dynamics and discrete element method (CFD-DEM) modeling tools to study the combined effects of pebble crushing, packing restructuring due to both gravity and the unbalanced force network in the pebble bed, and convection from helium purge gas on subsequent temperature profiles in solid breeders for different breeding configurations. In typical solid breeder modules, coolant fluid runs through the containing structure surrounding the pebble bed. Heat is removed from the pebble bed predominately through inter-particle conduction and contact conductance of many pebbles pressed against the containing surface. As such, heat transfer out of the pebble bed relies on maintaining good pebble-pebble and pebble-wall contact. However, physical contact is interrupted to different degrees when a pebble bed responds to various amounts of individual crushed pebbles. Furthermore, the restructuring of the pebble bed after a pebble crushing event is, in part, dependent on gravity forces acting upon each pebble in the ensemble. We investigate two representative pebble bed configurations where heat is removed from the bed via inter-particle conduction, convection of purge gas, and contact between the pebble bed and its container. In the first, the coolant containing structural walls (heat transfer walls) are oriented parallel to the gravity force. In the second configuration, the heat transfer walls are perpendicular to the direction of gravity. To simulate a crushed pebble, we replace the pebble with many smaller, non-cohesive elements while maintaining mass-conservation between the original solid pebble and crushed fragments. The fragments are then free to resettle into interstitial gaps and the rest of the bed resettles as determined by forces from gravity, contact of neighboring particles, and even the small influence of the moving purge gas. The thermofluid interaction with the helium purge gas will be included with volume-averaged Navier-Stokes and energy equations. The representative solid breeder volumes will be compared with respect to their temperature peaks and profiles and how those temperatures vary as a function of the percentage of crushed pebbles in the ensemble. The results can be used to optimize solid breeder pebble bed designs through the choice of breeding zone orientation relative to the gravity vector.