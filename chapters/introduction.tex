%%%%%%%%%%%%%%%%%%%%%%%%%%%%%%%%%%%%%%%%%%
\chapter{Introduction} \label{sec:introduction}
%%%%%%%%%%%%%%%%%%%%%%%%%%%%%%%%%%%%%%%%%%
From the very beginning of fusion reactor studies, researchers recognized the necessity of generating tritium inside the thermonuclear reactor. In the attempt to force the fusion neutron to collide with a lithium atom and thus produce a tritium atom, a design approach was proposed by Abdou et al\cite{Abdou1974d} in 1975 wherein the plasma would be surrounded by a `blanket' of nonmobile, solid lithium. The lithium would be combined with ceramic materials to maintain the solid phase at elevated temperatures and exist in a packed bed (also referred to as pebble bed) form.

To date, lithiated ceramic pebbles have been chosen by many participants in ITER experiments as a material to be used for tritium generation\cite{Lulewicz2002, Mandal2012a, Tsuchiya1998, Cho2012}. The advantages of the pebble bed design include ease of assembling the solid into complex geometries, ease of tritium extraction from the porous bed via an interstitial purge of helium, and with the small size of pebbles being more resilient to thermal stresses than a solid brick of lithiated ceramic. Naturally, however, the pebble bed form carries with it many of its own disadvantages that must be understood and overcome.

In thermonuclear fusion reactors, the ceramic pebble beds will be contained and cooled by a structural material (currently it is foreseen to use ferritic or austenitic steel). As nuclear energy is deposited into the poorly conductive ceramic breeder the temperature rises well above the containment structure and subsequently attempts to swell from thermal expansion but is confined due to the structure. This simple action is the root of many design issues for the pebble bed. For one, the confined expansion directly leads to cracking of individual pebbles from the high contact stresses. The packing structure response depends on the extent and modes of cracking and the thermophysical properties likewise change. Second, thermal ratcheting or thermally-induced bed creep can lead to evolutions in thermophysical properties even in the absence of cracked pebbles. Finally, as the thermophysical properties evolve, global or local bed temperatures change and ultimately the tritium release characteristics of the bed deviate from any prediction one may have had from the initial packing of the ceramic pebble bed. 

Alleviating any of the issues that may plague the ceramic breeder all boil down to requiring temperature control via an understanding and of the morphological changes of the ceramic packed beds and their interaction with the interstitial purge gas and structural container. In this work we introduce enhancements and new elements to build upon the understanding from ceramic breeder models of past research efforts. 

In this chapter we will introduce more detail on the fundamental features of a fusion reactor blanket design as it relates to a breeder blanket design. Following that will be a list of the research objectives of this study. In Part II (\cref{sec:hertz-theory,sec:modeling-heat-transfer,sec:modeling-pressure-drop}) we survey the state of the art in analysis of ceramic pebble beds, contact mechanics, and modeling thermal and mechanical interactions of packed beds.  In Part III (\cref{sec:modeling-dem,sec:modeling-cfd-dem,sec:modeling-lbm}) we outline the numerical methodology of the models and tools used in this study, namely: the discrete element method (DEM), coupled computational fluid dynamics and the discrete element method (CFD-DEM), and an integrated lattice-Boltzmann method (LBM). We employ our physics knowledge and numerical tools to cover a range of studies in Part IV (\cref{sec:cfd-dem-studies,sec:dem-studies,sec:lbm-studies,sec:analysis-experiment}). Finally, in Part V, we discuss the future of the current work and any limitations or interesting work that was beyond the scope of this dissertation.










\section{Solid breeder background}\label{sec:intro-blanket-description}

The solid breeder blanket is an integral part of the power generation and fuel cycle in a fusion reactor. On the road to understanding the specific functional requirements of the solid breeder blanket, we will review the major features of a fusion reactor power and fuel cycle as they relate to the blanket. Currently, the worldwide choice for fusion reaction in a power plant is the deuterium-tritium (DT) reaction. The choice is based on DT having: a high reaction probability at the lowest ion temperature, high energy yield, availabile fuel (to a degree), and relatively harmless reaction products. The DT reaction is
\begin{align}
	\mathrm{D} + \mathrm{T}&\xrightarrow{}~^4\mathrm{He}+\mathrm{n}+17.58~\text{MeV} \label{eq:dt-reaction}
\end{align}

Of the two isotopes fused, deuterium ($D$, or $^2$H) is a stable isotope and is naturally occuring in an average abundance of 0.015 mole percent in water on Earth. But tritium ($T$, or $^3$H), contrarily, is radioactive with a half-life of only about 12.32 years; naturally decaying as a $\beta^-$ emitter,
\begin{align}\label{eq:t-decay}
	\mathrm{T} \xrightarrow{}~^3\mathrm{He} + \beta^-
\end{align}

Owing to its short half-life, any naturally occurring tritium decays at such a rapid pace it will never accumulate to an appreciable amount on Earth. Because of this, to use the DT reaction in a fusion power plant, tritium will need to be generated artificially (bred) and collected as fuel source. One way of breeding tritium for a fusion reactor is to include a so-called tritium breeding blanket that surrounds the fusion plasma with a phase of lithium. We will keep our focus completely on the solid, non-mobile choice of a lithium blanket but note that lithium as a liquid is also heavily studied as a potential tritium breeding blanket.

The two most abundant isotopes in natural lithium interact with the neutrons as given in Eq.\ref{eq:lithium-t}
\begin{subequations}\label{eq:lithium-t}
\begin{align}
	\mathrm{n} + ~^7\mathrm{Li} &\xrightarrow ~\mathrm{n}+\alpha + \mathrm{T} -2.47~\text{MeV}\label{eq:li7-t}\\
	\mathrm{n} + ~^6\mathrm{Li} &\xrightarrow ~ \alpha + \mathrm{T} +4.78~\text{MeV} \label{eq:li6-t}
\end{align}
\end{subequations}
where we haved used the common short-hand of $\alpha$ in place of the helium nucleus. The cross-sections of the lithium reactions are given in Fig.~\ref{fig:li-xsects}. Note the exothermic lithium-6 reaction and the threshold energy required of the incident neutron to incite the lithium-7 reaction.

\begin{figure}
	\centering
	\includegraphics[width=0.75\textwidth]{chapters/figures/breeding_xsecs} 
	\caption{Cross-sections of various blanket materials. Note the threshold for the $^7$Li and neutron multiplying reactions.}
	\label{fig:li-xsects}
\end{figure}

Serendipitously, the DT reaction itself produces a high energy neutron (see Eq.~\ref{eq:dt-reaction}) that can interact with the two isotope reactions of Eq.~\ref{eq:lithium-t}. Therefore sulf-sufficiency of the fusion fuel cycle can be realized with the fusion neutron interacting with the lithium blanket. A commonly used classification of the efficacy of a breeding blanket is via the tritium breeding ratio (TBR), defined as 
\begin{equation}
	\text{TBR} = \cfrac{\dot{N}^+}{\dot{N}^-}
\end{equation}
where $\dot{N}^+$ is the number of tritium atoms generated per unit time in the blanket and $\dot{N}^-$ are the number of tritium atoms consumed per unit time.

For a DT cycle that creates a single neutron we can simplify this definition to say the tritium breeding ratio is the number of tritium atoms produced in the blanket per fusion neutron. If every single neutron from the fusion reaction were to be captured by lithium, we would have a TBR of $\approx 1$ and the reactor would possibly be self-sufficient in this ideal case. Unfortunately, in reality, only 60-80\% of the fusion neutrons actually react with the lithium due to neutron leakage and parasitic reactions. Futhermore, when we take into account tritium retention in structural material or losses due to inefficiency in collecting tritium, then self-sufficiency of the fuel cycle is clearly not possible unless we produce more than one tritium per fusion neutron. 

For solid breeders, beryllium is introduced into the blanket as a neutron multiplier. The incident neutron breaks Be up into two $\alpha$ particles and an additional two neutrons. Thus it is possible, with careful neutronics analysis and engineering of tritium breeding volumes and neutron multiplying regions, to attain a TBR which makes the power plant not only self-sufficient in terms of fuel, but also able to seed tritium for a future power plant. Assuming that the fuel cycle of tritium is handled properly (perhaps the biggest assumption we will make in this work), the last remaining function of the blanket is to supply energy for the electricity generation of the power plant. 

The fusion reaction deposits a great deal of surface radiation on the first wall of the breeding blanket and the blanket will be absorbing energy deposited from neutron interactions and $\gamma$ rays. The blanket must be capable of converting and then recovering the energy at high tempreatures for efficient power production in the fusion power plant. The nuclear heat generated in the pebble bed solid breeder will heat the ceramic pebbles to maximum temperatures of approximately 900~\celsius. The heat of the pebbles is transported through them via conduction through inter-particle contacts, conduction through the purge gas into neighboring particles, and ultimately through contact with the containing structure. The box structure surrounding the solid breeder will have high pressure (\si{8~MPa} in many current designs) helium flowing through channels that remove the heat from the blanket, with the helium reaching temperatures of 500~\celsius, and into a power generation cycle.

In summary, it is feasible, in principle, for a fusion power plant to produce electricity and be self-sustaining in terms of its limiting fuel. The feasibility depends on the the ability engineer a device that surrounds the fusion reaction, captures the ejected neutron to breed tritium, allows recovery of that tritium to attain self-sufficiency, and delivers the energy deposited in the blanket as high quality heat into a power cycle to create electricity. Blanket designs of solid breeders have evolved significantly since their introduction in the 1970s. Some features of current breeder designs will be discussed next.
%~~~~~~~~~~~~~~~~~~~~~~~~~~~~~~~~~~~~~~~~~~~~~~~~



%~~~~~~~~~~~~~~~~~~~~~~~~~~~~~~~~~~~~~~~~~~~~~~~~
\subsection{Solid breeder material and form considerations}
Pure lithium has a melting temperature of only about 180~\celsius so must be combined with a refractory material to keep it in solid form at high temperatures (melting temperatures >1000~\celsius). To date, most parties researching solid breeder blankets are focusing on lithium orthosilicate (\lis) or lithium metatitanite (\lit) as candidate ceramics, though other candidate ceramics do still exist. Lithium oxide had been considered because of its favorable lithium density, among other attractive features, though the reaction of lithium with elemental oxygen is a concern. Pure lithium reacts with oxygen exothermically in reactions such as

\begin{subequations}
\begin{align}
	2\mathrm{Li} + \frac{1}{2}\mathrm{O} &\rightarrow \mathrm{Li}_2\mathrm{O} - 142.75~\text{kCal/mol}\\
	2\mathrm{Li} + \mathrm{O} &\rightarrow \mathrm{Li}_2\mathrm{O}_2 - 151.9~\text{kCal/mol}
\end{align}
\end{subequations}

Of primary concern in lithium fires is the peak flame temperature. This will determine, to a large extent, whether many radioactive species become air-borne by vaporization. The flame temperature depends on many variables. Some investigations found it to be about 2500~K which would cause some materials to melt but not vaporize. [cite Abdou's class notes?]

As nuclear energy is deposited into the solid breeder, large thermal gradients in the solid lithiated ceramics will induce thermal stresses across large characteristic lengths. Avoiding thermal stress has led to most solid breeder designs implementing packed beds of small, spherical (or near-spherical) pebbles. Moreover, tritium diffusion and release considerations for solid lithium ceramic support the choice of short characteristic lengths of individual pebbles. From an engineering design standpoint, the choice of packed bed has other desirable characteristics. For instance, the ensemble of small spherical pebbles can be filled into many complex shapes with relatively uniform porosity for well-distributed flow of the purge gas. 

The packed bed will be contained in a structure of ferritic or austenitic steel. The energy of the packed bed is carried away by coolant channels in the structure that have flowing in them high pressure helium gas. Because the structural material is held cooler than the breeding zone, it will confine the thermal expansion of the lithium ceramic and lead to mechanical stresses at the points of contact of the individual pebbles in the packed bed. Engineering design issues surrounding this thermal stress is of great concern to researchers and will be the focus of much of this report.

Once in operation, the ceramic pebble beds will have a specified operating temperature window that is dictated by tritium release characteristics. The low end of the temperature window is governed by a minimum temperature for acceptable release rates of tritium from the ceramic to the purge gas; the value is generally set around 500~K. The upper limit of the temperature window is chosen to avoid sintering of the lithiated ceramic. Sintering of the ceramics, as grains in individual pebbles meld, is predicted to reduce the rate of tritium release. The upper end of the temperature window is generally set around 1000~K.

The size of breeder regions is limited by the temperature window combined with the poor effective conductivity of packed beds of ceramic pebbles. The conductivity is a weak function of external pressure but can generally be approximated as about \si{1 W/{mK}}. Because the effective conductivity and packed bed-wall interface conductance is predominately a contact conduction, disruptions to the packing structure will have considerable impact on the heat transfer of the packed bed.

As each individual tritium breeding region is small, in a typical solid breeder blanket design there are several alternating layers of breeding zone, cooling plate, and neutron multiplier. 

The two most prominently analyzed neutron multipliers for a fusion reactor are beryllium and lead. Beryllium has a very high nuclide density while also being very light, with a high melting temperature, and high thermal conductivity. For solid breeding blankets, beryllium has been pegged as the element of choice. The beryllium, based on its own design requirements, is generally also chosen to exist in a pebble bed form in the breeding blanket device.


\section{Importance of pebble bed integrity and motivation}\label{sec:intro-bed-integrity}


% From ITER 2013
Control of the manufacturing processes of the ceramic pebbles permits manufactureres to custom vary characteristics, such as the pebble's:
\begin{itemize}
\item tritium retention and release properties.
\item Lithium density
\item Opened- and closed-porosity
\item Nominal diameter
\item and, indirectly, crush strength. 
\end{itemize}
However the characteristics of the pebble are often coupled. For instance, for the sake of tritium management the open porosity of the pebble is often increased. But this comes at the expense of a decreased crush strength of the pebble. Because of the relatively weak crush strength distributions among batches of pebbles as well as the value of stresses predicted in the pebble bed, it is inevitable that during operation in the fusion environment individual pebbles will `fail' in the ensemble. Designers of lithium ceramic tritium breeding blankets must mitigate pebble failure but also anticipate the breadth and magnitude of effects that some unavoidable failure will have on macroscopic properties.

% [EDIT: THIS PARAGRAPH IS NOT NECESSARY? I DON'T NEED TO MAKE THE CASE FOR USING DEM. I JUST NEED TO EXPLAIN THE MODEL]The volume of a pebble in a tritium breeder is on the scale of 10$^{-9}$~m$^3$ while the typical container volume can be on the order of 10$^{-2}$~m$^3$\cite{Cho2008}.  Thus a single breeder volume will house upwards of $N = 10^7$ pebbles. Statistically then, the behavior of any single pebble seems insignificant and instead the entire ensemble of pebbles may be treated as a continuous media. Continuum theory for the is the basis of finite element method models that have been able to predict thermomechanical behavior with reasonable accuracy\cite{DiMaio20081287,Zaccari20081282,Gan:2009vn}. However, after the pebble beds are placed into the fusion environment they will be required to operate for long duty times without maintenance. Thus, as time progresses the accumulation of individual failed pebbles will eventually have consequences for the macroscopic thermomechanics.  and no continuum theory exists to account for this. Instead, we turn to the discrete element method to provide a solutino.

\section{Scope of the Work}\label{sec:intro-scope-of-work}
The objective of this dissertation is to develop numerical models of ceramic pebble beds, based on first principles and experimental observations, to simulate the hysteritic evolution of pebble bed morphology and predict the subsequent changes to heat transport characteristics after thermally-induced damage to pebbles. The numerical tools are constructed in the following progression: 1. Transient DEM code of inter-particle interactions is employed to simulate packed bed restructuring in the wake of crushed pebbles in the ensemble -- and the effective thermal conductivity following the restructuring, 2. Transient, volume-averaged equations of Navier-Stokes and energy of the helium purge gas are coupled to the DEM model of pebbles to simulate conjugate heat transfer and the interstitial fluid influence on thermophysical properties after crushing events, 3) Complete simulations of the tortuous path of helium purge gas with lattice-Boltzmann models (based on the packing structure determined in DEM simulations) to expose flattened temperature profiles due to laminar mixing in the pebble bed. 

A thorough understanding of the evolution of pebble bed morphology and the impact on thermophysical properties is critical for solid breeder designers. The understanding allows for temperature control of breeder pebble beds over the entire lifetime of the blanket which is crucial to the function of the solid breeder for tritium and energy generation. Thus we aim to provide designers of packed beds with tools to understand how packing states may evolve from time-dependent phenomena (e.g. sintering, creep, pebble cracking, etc.). These phenomena may, for instance: decrease the effective thermal conductivity which will raise bed temperatures beyond initial predictions, produce isolated pebbles which will sinter and potentially decrease tritium release rates, or even form gaps between pebble beds and containing structures leading to divergence from properties of the initial packing of the bed.

The objective of this work fits into the broader mission of our research group in the UCLA Fusion Science and Technology Center to develop and apply complete numerical models of ceramic pebble bed solid breeder modules. Any complete numerical model for a pebble bed would require the interaction of many sub-models or sub-functions operating at disparate scales. To demonstrate, a possible top-level algorithm could proceed in the following way: To begin, one must have knowledge of the interaction of the pebble bed with the containing structure as they exist in a fusion environment. The interactions are generally analyzed via the finite element method to find internal stresses and temperature fields of the entirety of the pebble bed and surrounding container. After the internal fields are mapped into the bed, one would use the discrete element method (DEM) to interpret the macroscopic stress fields into the inter-particle forces. With the inter-particle forces and total absorbed thermal energy calculated, a prediction of the initiation and evolution of morphological changes (i.e. crushed pebbles, sintering, creep, etc.) to each computational volume. Following this, DEM would calculate new effective properties as a result of the morphological changes to the pebble bed region. Finally, the updated bed properties would feed back into the FEM formulation to update calculations in the macroscopic stress fields. While a suite of integrated numerical tools that follows this example algorithm is the ultimate goal of our group, the work of this dissertation is focused entirely on the development of pebble-scale simulations that are predominately in the realm of the discrete element method.

In the following subs-sections, we briefly outline the studies fitting into the scope of this dissertation. 

\subsection*{Discrete Element Method Study on the Evolution of thermo-mechanics of a Pebble Bed Experiencing Pebble Damage}
In the first study of \cref{sec:dem-studies}, we analyze the effective thermal conductivity of a pebble bed assuming different fractions of pebbles in the ensemble are completely crushed. The focus of this study is to 1) determine the extent of change, in aggregate, to ensemble properties due to individual pebble crushing, 2) relate the changes in effective conductivity to quantifiable pebble-scale properties (e.g. contact force, coordination number, etc.), 3) use the results to create guidelines for designers to anticipate acceptable limits of pebble loss from a thermal management point of view. For the DEM tools used in this study, the only mode of heat transfer considered is conduction between the solid particles. 


\subsection*{Coupling DEM Models of Ceramic Breeder Pebble Beds to Thermofluid Models of Helium Purge Gas Using Volume-averaged CFD}
In a fusion breeder, the helium purge gas winding through the interstitial gaps of the pebbles has a substantial contribution to overall heat transfer.\cite{Reimann:2002mi,Abou-Sena2005} The model of \cref{sec:dem-studies} is improved to include the flowing interstitial gas. In \cref{sec:cfd-dem-studies}, we continue to employ our DEM tools to provide particle-scale information such as contact force, but couple the pebbles to a volume-averaged computational fluid dynamics (CFD) code for the conjugate heat transfer simulation. The coupled CFD-DEM model is used to again simulate the heat transfer in packed beds of ceramic spheres that experience pebble crushing -- but now with a focus on highlighting the impact of a flowing interstitial helium purge gas when pebbles are crushed.


\subsection*{Lattice-Boltzmann Method Integrating DEM Packing Structures to Study Laminar Mixing}
The models to account for helium purge gas employed in the studies of \cref{sec:cfd-dem-studies,sec:applied-studies} assume effective drag or heat transfer coefficients for pebbles in a computational volume and then include the pebble influence through effective source/sink terms in the momentum and energy equations. The volume-averaged approach allows for simpler meshing of the fluid volume while still retaining much of the physical realism of the system. Complete models of the conjugate heat transfer of both the fluid moving through the tortuous interstitial gaps pebble beds pressing each other with small contact areas are intractable with current computational hardware and finite-element modeling techniques. To overcome deficiencies in computational power, in \cref{sec:modeling-lbm}, we apply a relatively new technique wherein a lattice-Boltzmann algorithm solves for complete flow fields and conjugate heat transfer of helium winding through a packed bed. The lattice-Boltzmann method (LBM) is a non-traditional fluid simulation technique that allows us to resolve pebble/pore-scale momentum and energy transfer. The LBM approach is applied to the same pebble beds analyzed in \cref{sec:cfd-dem-studies} to provide comparison between the two modeling techniques. Furthermore the LBM model, accounting for the complex helium purge gas pathways, provides more insight to the influence of helium on the heat transfer in the heat transfer of packed beds.





\subsection*{Modeling Tools to Study Coolant Designs of ITER Solid Breeder Module Volumes}
In the study of \cref{sec:applied-studies}, we apply our coupled helium-pebble computational tools to the analysis of ITER-relevant solid breeder geometries. In this study we consider the combined effects of pebble crushing, packing restructuring due to both gravity and the unbalanced force network in the pebble bed, and convection from helium purge gas on temperature profiles in solid breeders for different breeding configurations. Heat transfer out of the pebble bed relies on maintaining good pebble-pebble and pebble-wall contact. However, physical contact is interrupted to different degrees when a pebble bed responds to various amounts of individual crushed pebbles. Furthermore, the restructuring of the pebble bed after a pebble crushing event is, in part, dependent on gravity forces acting upon each pebble in the ensemble. We investigate two representative pebble bed configurations where heat is removed from the bed via inter-particle conduction, convection of purge gas, and contact between the pebble bed and its container. In the first, the coolant containing structural walls (heat transfer walls) are oriented parallel to the gravity vector. In the second configuration, the heat transfer walls are perpendicular to the direction of gravity. To simulate a crushed pebble, we replace the pebble with many smaller, non-cohesive elements while maintaining mass-conservation between the original solid pebble and crushed fragments. The fragments are then free to resettle into interstitial gaps and the rest of the bed resettles as determined by forces from gravity, contact of neighboring particles, and even the small influence of the moving purge gas. The thermo-fluid interaction with the helium purge gas will be included with volume-averaged Navier-Stokes and energy equations. The representative solid breeder volumes will be compared with respect to their temperature peaks and profiles and how those temperatures vary as a function of the percentage of crushed pebbles in the ensemble. The results can be used to optimize solid breeder pebble bed designs through the choice of breeding zone orientation relative to the gravity vector.








