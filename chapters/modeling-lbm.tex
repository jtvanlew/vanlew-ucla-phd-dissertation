\chapter{Development of lattice-Boltzmann Modeling Tools for Ceramic Solid Breeders}\label{sec:modeling-lbm}
The volume-averaged approach of the CFD-DEM coupling is an effective and efficient method for solving transiently coupled helium flow and pebble interaction. However, there are cases when a complete knowledge of the tortuous flow of the interstitial helium is desired. But because the CFD-DEM solver does not resolve the pathways on the particle scale, knowledge of precise helium flow is not possible with that technique. Therefore we have also investigated a combination of DEM and with lattice-Boltzmann solvers. 

The lattice-Boltzmann approach to fluid simulations is a growing field of numerical modeling with a rich historical development. As the LBM approach is relatively unfamiliar, we will go through some of the notable evolutions of the modeling history and the background physics leading to the governing equations to be implemented numerically. Certainly this short study cannot do justice to a proper explanation of the underlying physics. References~\cite{Chen1998a,Viggen2009,Sukop2007,Chopard2002,succi2001lattice} should be read by those curious for excellent and thorough descriptions of the physics, modeling approaches, and applications of LBM theory to fluid dynamics problems.

In the rest of this chapter we introduce the core concepts behind the lattice-Boltzmann method, their application into numerical code, and finally a discussion of the solver used in this research.

%%%%%%%%%%%%%%%%%%%%%%%%%%%%%%%%%%%%%%%%%%
\input{chapters/sections/lbm-modeling-intro.tex}
\section{Numerical Methodology}

The distribution function at a given node is explicitly updated in time in two steps: collision and streaming. In the first step, collision operator of Eq.~\ref{eq:bgk-operator} dictates the distribution function at all the nodes. In the second step, information is streamed for the timestep to neighboring nodes according to Eq.~\ref{eq:lbm-evolution}.

The collision operator for the thermal lattice is that given by Guo, et al.17 The solver has two lattices overlaid upon each other. The first is used to solve for density and velocity. The second lattice uses the velocity at each node and solves for the passive temperature scalar. 

For our model we represented the pebbles with a resolution of 10 nodes per diameter. For the system analyzed, this resulted in two lattices that have nodal sizes of 201×151×501; requiring, in a total, about 30 million nodes to be updated at each time step.

In the DEM-LBM approach, DEM is used only to determine packing structure and contact forces. When the bed is settled, a snapshot of the structure is discretized and loaded into the LBM solver which then calculates temperature and velocity fields of both solid and fluid phases. There is no cross-communication in this technique as the packing structure is effectively frozen during the LBM calculations.

\input{chapters/sections/lbm-benchmark.tex}
%%%%%%%%%%%%%%%%%%%%%%%%%%%%%%%%%%%%%%%%%%